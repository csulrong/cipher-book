\startcomponent ecc

\startsection[title={椭圆曲线密码学},reference=sec:ecc]
\index{sec:ecc}

{\it 椭圆曲线密码学}(Elliptic Curve Cryptography,ECC)是一种基于椭圆曲线数学的公开密钥加密算法。
椭圆曲线在密码学中的使用是在1985年由Neal Koblitz和Victor Miller分别独立提出的。ECC的主要优势是它
相比RSA加密算法使用较小的密钥长度并提供相当等级的安全性。

\startsubsection[title={群},reference=subsec:group]

在介绍椭圆曲线之前,有必要先来了解一下{\it 群}(group)的概念。在数学上,{\it 群}是由一种集合
$\mathbb{G}$以及一个二元运算(比如用符号$+$表示的“加法”运算)所组成的,并且符合包含下述四个性质的
代数结构:

\startitemize[n]
\item {\it 封闭性}(closure):$a \in \mathbb{G}, b \in \mathbb{G} \Rightarrow a+b \in \mathbb{G}$。

\item {\it 结合律}(associativity):$a \in \mathbb{G}, b \in \mathbb{G}, c \in \mathbb{G} \Rightarrow (a+b)+c=a+(b+c)$。

\item {\it 单位元}(identity element):存在一个单位元用$0$表示,使得$a+0=0+a=a$。单位元是与任意
元素运算不改变其值的元素。

\item {\it 逆元}(inverse):对于$\mathbb{G}$中的每个元素$a$,存在$\mathbb{G}$中的一个元素$b$,
使得$a+b=b+a=0$。
\stopitemize

群运算的次序很重要,把元素$a$与元素$b$进行二元运算,所得到的结果不一定与把元素$b$与元素$a$进行二元运算
的结果相同,亦即,$a + b = b + a$({\it 交换律},commutativity)不一定恒成立。如果把交换律作为第5个
性质的话,我们把同时满足这五个性质的群称为{\it 阿贝尔群}(abelian group)。

从我们通常的加法概念来看,整数集$\mathbb{Z}$是一个阿贝尔群。自然数集$\mathbb{N}$不是一个群,因为它不
满足第4条性质。
\stopsubsection

\startsubsection[title={实数域上的椭圆曲线},reference=subsec:ec]
一条椭圆曲线就是一组由如下形式的方程定义的点集。

\startformula
y^{2}=x^{3}+ax+b, \quad 4a^3 + 27b^2 \neq 0
\stopformula

椭圆曲线的定义要求曲线是{\it 非奇异}的。几何上来说,这意味着图像里面没有尖点、自相交或孤立点。其中,
$4a^3 + 27b^2 \neq 0$这个限定条件是为了保证曲线不包含{\it 奇点}(singularity)。

\in{图}[fig:ecc-curves]给出了$a$从$-2$到$0$,$b$从$-1$到$3$变化时对椭圆曲线形状的影响。从图中,
可以看到随着$a$和$b$的变化,椭圆曲线也会在平面上呈现出不同的形状,但有一点是很容易辨认的,椭圆曲线始终
是关于$x$轴对称的。我们观察到,当$a=0, b=0$时,曲线存在尖点,不满足椭圆曲线规定的限定条件。另外,我们
还可以找出曲线自相交和存在孤立点的情况,如\in{图}[fig:ecc-singular]。这些特殊点,我们都称之为奇点,
此时$a$和$b$的值不满足椭圆曲线的限定条件。

另外,我们还需要一个无穷处的点(point at infinity/ideal point)作为曲线的一部分,从现在开始,我们
将用$0$这个符号表示无穷处的点。如果我们将无穷处的点也考虑进来的话,那么椭圆曲线的表达式精炼为:

\startformula
\left\{(x, y) \in \mathbb{R}^2 \mid y^{2}=x^{3}+ax+b, 4a^3 + 27b^2 \neq 0\right\} \cup \{0\}
\stopformula

\startplacefigure[title={椭圆曲线形状随参数$a$、$b$的变化},reference=fig:ecc-curves]
\switchtobodyfont[8pt]
\pgfplotsset{
    xmin=-3,xmax=5,ymin=-6,ymax=6,
    xlabel={$x$},ylabel={$y$},
    xticklabel={\empty},yticklabel={\empty},
    scale only axis=false,
    axis lines=middle,
    width=150pt,
    samples=801,
    unbounded coords=jump,
    clip=false,
    axis equal image=true,
}
\setupcombination[distance=5mm,after=]
\startcombination[5*3]
% first row
{
\starttikzpicture
% a = -2, b = -1
\startaxis[
    title={$a=-2$, $b=-1$},
    domain=-1:3,
    smooth,
]
\addplot [red!62!black,thick] {sqrt(x^3-2*x-1)};
\addplot [red!62!black,thick] {-sqrt(x^3-2*x-1)};
\stopaxis
\stoptikzpicture
}{}
{
\starttikzpicture
% a = -2, b = 0
\startaxis[
    title={$a=-2$, $b=0$},
    domain=-1.41421:3,
    smooth,
]
\addplot [red!62!black,thick] {sqrt(x^3-2*x)};
\addplot [red!62!black,thick] {-sqrt(x^3-2*x)};
\stopaxis
\stoptikzpicture
}{}
{
\starttikzpicture
% a = -2, b = 1
\startaxis[
    title={$a=-2$, $b=1$},
    domain=-1.61803:3,
    smooth,
]
\addplot [red!62!black,thick] {sqrt(x^3-2*x+1)};
\addplot [red!62!black,thick] {-sqrt(x^3-2*x+1)};
\stopaxis
\stoptikzpicture
}{}
{
\starttikzpicture
% a = -2, b = 2
\startaxis[
    title={$a=-2$, $b=2$},
    domain=-1.76929:3,
    smooth,
]
\addplot [red!62!black,thick] {sqrt(x^3-2*x+2)};
\addplot [red!62!black,thick] {-sqrt(x^3-2*x+2)};
\stopaxis
\stoptikzpicture
}{}
{
\starttikzpicture
% a = -2, b = 3
\startaxis[
    title={$a=-2$, $b=3$},
    domain=-1.893289:3,
    smooth,
]
\addplot [red!62!black,thick] {sqrt(x^3-2*x+3)};
\addplot [red!62!black,thick] {-sqrt(x^3-2*x+3)};
\stopaxis
\stoptikzpicture
}{}

% second row
{
\starttikzpicture
% a = -1, b = -1
\startaxis[
    title={$a=-1$, $b=-1$},
    domain=1.32472:3,
    smooth,
]
\addplot [red!62!black,thick] {sqrt(x^3-x-1)};
\addplot [red!62!black,thick] {-sqrt(x^3-x-1)};
\stopaxis
\stoptikzpicture
}{}
{
\starttikzpicture
% a = -1, b = 0
\startaxis[
    title={$a=-1$, $b=0$},
    domain=-1:3,
    smooth,
]
\addplot [red!62!black,thick] {sqrt(x^3-x)};
\addplot [red!62!black,thick] {-sqrt(x^3-x)};
\stopaxis
\stoptikzpicture
}{}
{
\starttikzpicture
% a = -1, b = 1
\startaxis[
    title={$a=-1$, $b=1$},
    domain=-1.324718:3,
    smooth,
]
\addplot [red!62!black,thick] {sqrt(x^3-x+1)};
\addplot [red!62!black,thick] {-sqrt(x^3-x+1)};
\stopaxis
\stoptikzpicture
}{}
{
\starttikzpicture
% a = -1, b = 2
\startaxis[
    title={$a=-1$, $b=2$},
    domain=-1.5213797:3,
    smooth,
]
\addplot [red!62!black,thick] {sqrt(x^3-x+2)};
\addplot [red!62!black,thick] {-sqrt(x^3-x+2)};
\stopaxis
\stoptikzpicture
}{}
{
\starttikzpicture
% a = -1, b = 3
\startaxis[
    title={$a=-1$, $b=3$},
    domain=-1.671699:3,
    smooth,
]
\addplot [red!62!black,thick] {sqrt(x^3-x+3)};
\addplot [red!62!black,thick] {-sqrt(x^3-x+3)};
\stopaxis
\stoptikzpicture
}{}

% third row
{
\starttikzpicture
% a = 0, b = -1
\startaxis[
    title={$a=0$, $b=-1$},
    domain=1:3,
    smooth,
]
\addplot [red!62!black,thick] {sqrt(x^3-1)};
\addplot [red!62!black,thick] {-sqrt(x^3-1)};
\stopaxis
\stoptikzpicture
}{}
{
\starttikzpicture
% a = 0, b = 0
\startaxis[
    title={$a=0$, $b=0$},
    domain=0:3,
    smooth,
]
\addplot [red!62!black,thick] {sqrt(x^3)};
\addplot [red!62!black,thick] {-sqrt(x^3)};
\stopaxis
\stoptikzpicture
}{}
{
\starttikzpicture
% a = 0, b = 1
\startaxis[
    title={$a=0$, $b=1$},
    domain=-1:3,
    smooth,
]
\addplot [red!62!black,thick] {sqrt(x^3+1)};
\addplot [red!62!black,thick] {-sqrt(x^3+1)};
\stopaxis
\stoptikzpicture
}{}
{
\starttikzpicture
% a = 0, b = 2
\startaxis[
    title={$a=0$, $b=2$},
    domain=-1.259921:3,
    smooth,
]
\addplot [red!62!black,thick] {sqrt(x^3+2)};
\addplot [red!62!black,thick] {-sqrt(x^3+2)};
\stopaxis
\stoptikzpicture
}{}
{
\starttikzpicture
% a = 0, b = 3
\startaxis[
    title={$a=0$, $b=3$},
    domain=-1.442249:3,
    smooth,
]
\addplot [red!62!black,thick] {sqrt(x^3+3)};
\addplot [red!62!black,thick] {-sqrt(x^3+3)};
\stopaxis
\stoptikzpicture
}{}
\stopcombination
\stopplacefigure


\startplacefigure[title={当限定条件不满足时,椭圆曲线存在尖点、自相交或孤立点},reference=fig:ecc-singular]
\switchtobodyfont[8pt]
\pgfplotsset{
    xmin=-3,xmax=5,ymin=-6,ymax=6,
    xlabel={$x$},ylabel={$y$},
    xticklabel={\empty},yticklabel={\empty},
    scale only axis=false,
    axis lines=middle,
    width=180pt,
    samples=1001,
    unbounded coords=jump,
    clip=false,
    axis equal image=true,
}
\setupcombination[distance=1.25cm,after=]
\startcombination[3*1]
{
\starttikzpicture
% a = 0, b = 0
\startaxis[
    title={$a=0$, $b=0$},
    domain=0:3,
    smooth,
]
\addplot [red!62!black,thick] {sqrt(x^3)};
\addplot [red!62!black,thick] {-sqrt(x^3)};
\stopaxis
\stoptikzpicture
}{\darkred 尖点}
{
\starttikzpicture
% a = -3, b = 2
\startaxis[
    title={$a=-3$, $b=2$},
    domain=-2:3,
    smooth,
]
\addplot [red!62!black,thick] {sqrt(x^3-3*x+2)};
\addplot [red!62!black,thick] {-sqrt(x^3-3*x+2)};
\stopaxis
\stoptikzpicture
}{\darkred 自相交}
{
\starttikzpicture
% a = -3, b = -2
\startaxis[
    title={$a=-3$, $b=-2$},
    domain=-2:3,
]
\addplot [red!62!black,thick] {sqrt(x^3-3*x-2)};
\addplot [red!62!black,thick] {-sqrt(x^3-3*x-2)};
\addplot [scatter,only marks,mark size=1pt,
    scatter/use mapped color={
        draw=red!62!black,
        fill=red!62!black
    }] coordinates {
    (-1,0)
};
\stopaxis
\stoptikzpicture
}{\darkred 孤立点}
\stopcombination
\stopplacefigure


\startsubsubject[title={椭圆曲线上的群论},reference=subsubsec:ecc-group]

我们可以在椭圆曲线上定义一个群:

\startitemize
\item 群中的元素就是椭圆曲线上的点。

\item 单位元就是无穷处的点$0$。

\item 点$P$的逆元是关于X轴对称的另一边的点,记作$-P$。

\item 二元运算规则定义如下:取一条直线和椭圆曲线相交的三点$P$、$Q$、$R$(皆非单位元),他们的总和等于
单位元$0$,即$P+Q+R=0$。
\stopitemize

请注意最后一条规则,我们仅仅说了需要三个在一条直线上的点,并没有规定它们的顺序。这就意味着,如果$P$、$Q$、$R$在
一条直线上的话,它们满足

\startformula
P+(Q+R)=Q+(P+R)=R+(P+Q)=\cdots=0
\stopformula

这样,我们可以直观地证明:定义在椭圆曲线上的加法运算是符合交换律和结合律的,这是一个阿贝尔群。

\stopsubsubject

\startsubsubject[title={几何加法},reference=subsubsec:ecc-geo]

由于椭圆曲线的点集属于一个阿贝尔群,所以,我们可以将$P+Q+R=0$写成$P+Q=−R$。这个方程式让我们派生出了一个用几何方法
去计算两个点$P$和$Q$的和:当我们画一条直线通过$P$、$Q$,这条线将会和椭圆曲线相交于第三个点$R$(这就意味着
$P$、$Q$、$R$三点是在一条直线上的)。如果我们取它相反的点$-R$,我们就可以找到$P+Q$的结果,如\in{图}[fig:ecc-geo]所示。


\startplacefigure[title={穿过$P$和$Q$的直线与曲线相交的第三点$R$关于$x$轴对称的点$-R$就是$P+Q$的结果},reference=fig:ecc-geo]
\switchtobodyfont[8pt]
\pgfplotsset{
    xmin=-3,xmax=4,ymin=-3,ymax=3,
    xlabel={$x$},ylabel={$y$},
    xticklabel={\empty},yticklabel={\empty},
    scale only axis=false,
    axis lines=none,
    width=150pt,
    samples=801,
    unbounded coords=jump,
    clip=false,
    axis equal image=true,
}
\midaligned{
\starttikzpicture
% P+Q+R=0
\startaxis[
    domain=-1.76929:3,
    smooth,
]
\addplot [red!62!black,thick] {sqrt(x^3-2*x+2)};
\addplot [red!62!black,thick] {-sqrt(x^3-2*x+2)};

\addplot [
    nodes near coords,
    only marks,
    mark=*,
    mark size=1.6pt,
    mark options={draw=green!62!black,fill=green!62!black},
    point meta=explicit symbolic] coordinates {
    (1.76929,  2) [$Q$]
    (-1.73461, 0.5) [$P$]
    (0.14858,  1.30619) [$R$]
};

\addplot [
    nodes near coords,
    every node near coord/.append style={anchor=north},
    only marks,
    mark=*,
    mark size=1.6pt,
    mark options={draw=yellow!62!black,fill=yellow!62!black},
    point meta=explicit symbolic] coordinates {
    (0.14858,  -1.30619) [$-R$]
};

\addplot [yellow!62!black,thick,dashed] coordinates {
    (0.14858, 1.30619)
    (0.14858, -1.30619)
};
\addplot [domain=-3:4,green!62!black,thick] {0.4280944*(x+1.73461)+0.5};
\stopaxis
\stoptikzpicture
}
\stopplacefigure


这个几何方法非常有用但是还需要再考虑以下几种情况(\in{图}[fig:ecc-cases]画出了每种情况的一个例子):

\startitemize
\head {\bf 情形1:$P=-Q$}

在这种情况下,穿过两点的直线是和$x$轴垂直的,和曲线没有相交的第三个点。此时,$Q$是$P$的逆元,从逆元的定义可以得到$P+Q=P+(-P)=0$。

\head {\bf 情形2:$P=Q$}

在这种情况下,有无数条线会经过这个点。我们假设一个点$Q' \neq P$. 当$Q'$越来越接近$P$并和$P$重合的时候,穿过$P$和$Q'$两点
的这条线最终会成为曲线的一条切线,这条切线与曲线相交的另一点就是$R$,也就有$P+P+R=0$,或者写成$P+P=−R$,$R$是曲线和切线的
交点,$P$是切点。

\head {\bf 情形3:$P \neq Q$但是经过$P$和$Q$的直线与椭圆曲线没有第三个交点}

这种情况与上一种情况非常相似。事实上,这种情况就是一条直线穿过$P$和$Q$与曲线相切。我们可以假设$P$是切点,在上一个情况下,我们已经
说明了$P+P=−Q$,这个方程现在可以写成:$P+Q=−P$。

\head {\bf 情形4:$P=Q$且$P=-Q$}

这种情况是上述情形2或情形3的一个特例。此时,$P$为曲线和$x$轴的交点。经过$P$点的切线垂直于$x$轴和曲线没有相交的其他点。$P$的
逆元为其自身,同时,$P+Q=P+P+0=0$。

\head {\bf 情形5:$P=0$或者$Q=0$}

很明显,这样我们是画不出线的,无穷远点$0$不在$xy$平面上。但是我们已经定义了$0$作为单位元。$P+0=P$和$Q+0=Q$,对于
任意的$P$和$Q$都适用,单位元的作用就是与任意元素运算不改变其值的元素。
\stopitemize

\startplacefigure[title={椭圆曲线上几何加法的几种特殊情况},reference=fig:ecc-cases]
\switchtobodyfont[8pt]
\pgfplotsset{
    xmin=-3,xmax=3,ymin=-3,ymax=3,
    xlabel={$x$},ylabel={$y$},
    xticklabel={\empty},yticklabel={\empty},
    scale only axis=false,
    axis lines=none,
    width=140pt,
    samples=801,
    unbounded coords=jump,
    clip=false,
    axis equal image=true,
}
\setupcombination[distance=1.25cm,after=]
\startcombination[4*1]
{
% P+Q+0=0
\starttikzpicture
\startaxis[
    domain=-1.76929:3,
    smooth,
]
\addplot [red!62!black,thick] {sqrt(x^3-2*x+2)};
\addplot [red!62!black,thick] {-sqrt(x^3-2*x+2)};

\addplot+ [
    nodes near coords,
    every node near coord/.append style={anchor=east},
    only marks,
    mark=*,
    mark size=1.6pt,
    mark options={draw=green!62!black,fill=green!62!black},
    point meta=explicit symbolic] coordinates {
    (0.14858, 1.30619) [$P$]
    (0.14858, -1.30619) [$Q$]
};

\addplot [green!62!black,thick] coordinates {
    (0.14858, 4)
    (0.14858, -4)
};
\stopaxis
\stoptikzpicture
}{\darkred (1) $P+Q+0=0$}
{
% P+P+R=0
\starttikzpicture
\startaxis[
    domain=-1.76929:3,
    smooth,
]
\addplot [red!62!black,thick] {sqrt(x^3-2*x+2)};
\addplot [red!62!black,thick] {-sqrt(x^3-2*x+2)};

\addplot [
    nodes near coords,
    only marks,
    mark=*,
    mark size=1.6pt,
    mark options={draw=green!62!black,fill=green!62!black},
    point meta=explicit symbolic] coordinates {
    (-1.1, 1.69381) [$P$]
    (2.43152, 3.39305) [$R$]
    (1.70452, 1.88239) [$R'$]
};
\addplot [
    nodes near coords,
    every node near coord/.append style={anchor=north},
    only marks,
    mark=*,
    mark size=1.6pt,
    mark options={draw=green!62!black,fill=green!62!black},
    point meta=explicit symbolic] coordinates {
    (-0.6, 1.72743) [$Q'$]
};
\addplot [
    nodes near coords,
    every node near coord/.append style={anchor=north east},
    only marks,
    mark=*,
    mark size=1.6pt,
    mark options={draw=yellow!62!black,fill=yellow!62!black},
    point meta=explicit symbolic] coordinates {
    (2.43152, -3.39305) [$-R$]
    (1.70452, -1.88239) [$-R'$]
};
\addplot [yellow!62!black,thick,dashed] coordinates {
    (2.43152, 3.39305)
    (2.43152, -3.39305)
};
\addplot [yellow!62!black,thick,dashed] coordinates {
    (1.70452, 1.88239)
    (1.70452, -1.88239)
};
\addplot [domain=-3:3,green!62!black,thick] {0.481163918*(x+1.1)+1.69381};
\addplot [domain=-3:3,green!62!black,thick,dashed] {0.06724*(x+1.1)+1.69381};

\stopaxis
\stoptikzpicture
}{\darkred (2) $P+P+R=0$}
{
% P+Q+P=0
\starttikzpicture
\startaxis[
    domain=-1.76929:3,
    smooth,
]
\addplot [red!62!black,thick] {sqrt(x^3-2*x+2)};
\addplot [red!62!black,thick] {-sqrt(x^3-2*x+2)};

\addplot+ [
    nodes near coords,
    only marks,
    mark=*,
    mark size=1.6pt,
    mark options={draw=green!62!black,fill=green!62!black},
    point meta=explicit symbolic] coordinates {
    (-1, 1.73205) [$P$]
    (2.08333, 2.62213) [$Q$]
};

\addplot [
    nodes near coords,
    every node near coord/.append style={anchor=north},
    only marks,
    mark=*,
    mark size=1.6pt,
    mark options={draw=yellow!62!black,fill=yellow!62!black},
    point meta=explicit symbolic] coordinates {
    (-1, -1.73205) [$-P$]
};

\addplot [yellow!62!black,thick,dashed] coordinates {
    (-1, 1.73205)
    (-1, -1.73205)
};
\addplot [domain=-3:3,green!62!black,thick] {0.2886749*(x+1)+1.73205};
\stopaxis
\stoptikzpicture
}{\darkred (3) $P+Q+P=0$}
{
% P+P+0=0
\starttikzpicture
\startaxis[
    domain=-1.76929:3,
    smooth,
]
\addplot [red!62!black,thick] {sqrt(x^3-2*x+2)};
\addplot [red!62!black,thick] {-sqrt(x^3-2*x+2)};

\addplot+ [
    nodes near coords,
    every node near coord/.append style={anchor=east},
    only marks,
    mark=*,
    mark size=1.6pt,
    mark options={draw=green!62!black,fill=green!62!black},
    point meta=explicit symbolic] coordinates {
    (-1.76929, 0) [$P$]
};

\addplot [green!62!black,thick] coordinates {
    (-1.76929, 4)
    (-1.76929, -4)
};
\stopaxis
\stoptikzpicture
}{\darkred (4) $P+P+0=0$}
\stopcombination
\stopplacefigure

\stopsubsubject

\startsubsubject[title={代数加法},reference=subsubsec:ecc-algebra]

如果我们想要一台计算机能够运行点的加法运算,那我们就需要把几何方法转换成代数方法。将一些规则转换成一系列的方程式看上去是
非常直观的,但是实际上是很枯燥的,因为要算三次方程。出于这个原因,这里我只放结果。

首先,我们先去掉一些特殊情况,只会考虑两个非无穷处点$P (x_P, y_P)$、$Q (x_Q, y_Q)$。我们针对$P$和$Q$是否对称这两种
情况分别考虑。

\startitemize
\head 先假设$P$和$Q$不对称,即$x_P \neq x_Q$

此时,经过$P$和$Q$的直线的斜率为

\startformula
k=\frac{y_P - y_Q}{x_P - x_Q}
\stopformula

令该直线的方程为$y=kx+d$,直线与椭圆曲线相交,则有:

\startformula
(kx+d)^2=x^3+ax+b \Rightarrow x^3 - k^2 x^2 + (a-2kd)x + (b-d^2) = 0
\stopformula

因为直线与椭圆曲线相交于第三点$R$,$P$、$Q$、$R$为直线与曲线的交点,即上述方程的解,有:

\startformula
(x-x_P) (x-x_Q) (x-x_R) = x^3 - (x_P+x_Q+x_R)x^2 + (x_{P}x_{Q}+x_{P}x_{R}+x_{Q}x_{R})x - x_{P}x_{Q}x_{R}
\stopformula

替换$x^2$的系数后,得到$x_P + x_Q + x_R = k^2$,这样,我们便能求得$R$的横坐标:

\placeformula[formula:xr]
\startformula
x_R = k^2 - x_P - x_Q
\stopformula

进而,通过斜率,我们接下来求得$R$的纵坐标:

\startformula
y_R = y_P + k(x_R - x_P)
\stopformula

于是,$(x_P, y_P) + (x_Q, y_Q)$的结果为$(x_R, -y_R)$(请注意符号的变化,并且记住:$P+Q=−R$)。

\head 若$P$和$Q$对称,即$x_P = x_Q$

进一步考虑以下两种情况:
\startitemize[n]
\item 若$y_P = -y_Q$,即$P$和$Q$关于$x$轴对称,此时,$P+Q=0$。

\item 若$y_P = y_Q$,则$P$和$Q$重合,曲线在$P$点的切线斜率为下式的一阶导数。

\startformula
y_P=\pm \sqrt{x_P^3 + a x_P + b} 
\stopformula

根据一阶导数的计算方法,可以得到曲线在$P$点的切线斜率为:

\startformula
k = \frac{3x_P^2 + a}{2y_P}
\stopformula

因此,根据\in{公式}[formula:xr],可以计算得到:

\startformula
\startmathalignment[n=4,align=middle]
\NC x_R \NC = \NC k^2 - 2x_P           \NR
\NC y_R \NC = \NC y_P + k(x_R - x_P)   \NR
\stopmathalignment
\stopformula
\stopitemize

\stopitemize

\stopsubsubject

\startsubsubject[title={标量乘法和对数},reference=subsubsec:scalar-multi]

除了加法,我们还需定义另一个运算:标量乘法,如下:

\startformula
nP = \underbrace{P + P + \cdots + P}_{\mathrm{n~times}}
\stopformula

根据上述标量乘法的定义和几何加法的定义,$2P=P+P$的结果为经过$P$的切线与曲线的交点关于$x$轴的对称点,
$3P=P+2P$为经过$P$和$2P$与曲线的交点关于$x$轴的对称点,以此类推。\in{图}[fig:ecc-multi]给出了
标量乘法的过程图。

\startplacefigure[title={椭圆曲线上点的标量乘法},reference=fig:ecc-multi]
\switchtobodyfont[8pt]
\pgfplotsset{
    xmin=-2.5,xmax=2.5,ymin=-2.5,ymax=2.5,
    xlabel={$x$},ylabel={$y$},
    xticklabel={\empty},yticklabel={\empty},
    scale only axis=false,
    axis lines=none,
    width=180pt,
    samples=801,
    unbounded coords=jump,
    clip=false,
    axis equal image=true,
}
\midaligned{
\starttikzpicture
\startaxis[
    domain=-1.76929:2.0,
    smooth,
]
\addplot [red!62!black,thick] {sqrt(x^3-2*x+2)};
\addplot [red!62!black,thick] {-sqrt(x^3-2*x+2)};

\addplot [
    nodes near coords,
    % every node near coord/.append style={anchor=east},
    only marks,
    mark=*,
    mark size=1.6pt,
    mark options={draw=green!62!black,fill=green!62!black},
    point meta=explicit symbolic] coordinates {
    (1.2, 1.15239) [$P$]
    (-1.38675, 1.45144) [$2P$]
    (0.20011, -1.26799) [$3P$]
};

\addplot [
    nodes near coords,
    % every node near coord/.append style={anchor=east},
    only marks,
    mark=*,
    mark size=1.6pt,
    mark options={draw=green!62!black,fill=white!62!black},
    point meta=explicit symbolic] coordinates {
    (-1.38675, -1.45144)
    (0.20011, 1.26799)
};

\addplot [green!62!black,thick] {1.00660288*(x-1.2)+1.15239};
\addplot [green!62!black,thick] {-0.1156083889*(x-1.2)+1.15239};
\addplot [yellow!62!black,thick,dashed] coordinates {
    (-1.38675, -1.45144)
    (-1.38675, 1.45144)
};
\addplot [yellow!62!black,thick,dashed] coordinates {
    (0.20011, 1.26799)
    (0.20011, -1.26799)
};
\stopaxis
\stoptikzpicture
}
\stopplacefigure

从标量乘法的定义,也可以看出计算$nP$需要做$n$次加法运算。如果$n$有$k$位二进制的话,我们的算法时间复杂度是$O(2^k)$,
当$k$特别大的时候,这不是一个好的结果。幸好还有一个被称作{\it 快速幂算法}的方法,它的原理可以用如下例子解释。假设$n=151$,
二进制表示为$10010111_2$,这个二进制也可以表示成幂次加之和:

\startformula
\startmathalignment[n=4,align=middle]
\NC 115 \NC = \NC 1 \cdot 2^7 + 0 \cdot 2^6 + 0 \cdot 2^5 + 1 \cdot 2^4 + 0 \cdot 2^3 + 1 \cdot 2^2 + 1 \cdot 2^1 + 1\cdot 2^0 \NR
\NC     \NC = \NC 2^7 + 2^4 + 2^2 + 2^1 + 2^0 \NR
\stopmathalignment
\stopformula

由此,椭圆曲线上的点的标量乘法可以简化为:

\startformula
151 \cdot P = 2^7 \cdot P + 2^4 \cdot P + 2^2 \cdot P + 2^1 \cdot P + 2^0 \cdot P
\stopformula

快速幂算法告诉我们的是,在该例中,只需要7次倍乘和4次加法操作就可以计算出$151P$。7次倍乘依次计算出$2^1 \cdot P$、
$2^2 \cdot P$、$\cdots$、$2^7 \cdot P$。每次倍乘都在前一次倍乘结果的基础上乘以2。最后,将不需要的中间倍乘结果
舍弃掉,把剩下来需要的倍乘结果通过4次加法操作就可以计算出$151P$的最终结果。

倍乘和加法都是时间复杂度为常数的运算,那么,这个算法的时间复杂度是$O(\log{n})$。(或者是$O(k)$,如果我们考虑到比特
长度的话),这个结果还是不错的。

对于给定的$n$和$P$,我们现在至少可以利用快速幂算法在$O(\log{n})$的时间复杂度级别计算出$Q=nP$。那么反过来呢?如果
我们已知$Q$和$P$,如何找到$n$呢?这个问题就是著名的对数问题(logarithm problem)。到目前为止,没有发现比穷举试探
方法快太多的算法,于是椭圆曲线加密所依赖的数学难题就这么诞生了。
\stopsubsubject
\stopsubsection

\startsubsection[title={有限域的椭圆曲线},reference=subsec:ecc-fp]

前面,我们已经了解了在实数域$\mathbb{R}$的椭圆曲线可以用来定义一个群。我们在实数域上面的椭圆曲线定义了一个点的加法
运算,并对加法运算的几何方法和代数方法进行了详细的阐述。

接下来,我们将椭圆曲线限定在有限域内,然后看看会有什么变化。我们对有限域的概念应该不感到陌生了,在介绍AES算法的
\type{MixColumn}运算的时候,我们曾经介绍过有限域(或伽罗瓦域)的概念。这里,我们用$\mathbb{F}_p$表示有$p$个
元素的有限域。回顾一下有限域上的加法和乘法运算,两者都满足封闭性、交换律和结合律,乘法相对加法还满足分配律,即
$x \times (y + z) = x \times y + x \times z$。

现在,我们对椭圆曲线在有限域$\mathbb{F}_p$上的定义如下:

\startformula
E = \left\{(x,y) \in (\mathbb{F}_p)^2 \vert y^2 \equiv x^3+ax+b \pmod{p}, \quad 4a^3+27b^2 \nequiv 0 \pmod{p} \right\}  \cup \{0\}
\stopformula

这里的$0$仍然是无穷处的点,$a$和$b$是$\mathbb{F}_p$上的两个整数。\in{图}[fig:ecc-fp]给出了椭圆曲线
$y^2=x^3-3x+10$在有限域$\mathbb{F}_{19}$、$\mathbb{F}_{97}$和$\mathbb{F}_{127}$上的图形。从几何的角度,
图形则从连续的曲线变成$xy$平面上的离散点的集合。即便对于定义域进行了限制,$\mathbb{F}_p$域上的椭圆曲线依然可以
组成一个阿贝尔群。

\startplacefigure[title={有限域上的椭圆曲线$y^2=x^3-3x+10$},reference=fig:ecc-fp]
\switchtobodyfont[8pt]
\pgfplotsset{
    grid=major,
    scale only axis=false,
    width=180pt,
    unbounded coords=jump,
    clip=false,
    axis equal image=true,
}
\startcombination[3*1]
{
\starttikzpicture
\startaxis[
    xmin=0,xmax=18,ymin=0,ymax=18,
    xtick distance=3,
    ytick distance=3,
]
\addplot [
    only marks,
    mark=*,
    mark size=1pt,
    mark options={draw=red!62!black,fill=white!62!black},
] table {programs/cipher-asymmetric/ecc-fp19.dat};
\stopaxis
\stoptikzpicture
}{\darkred $\mathbb{F}_{19}$}
{
\starttikzpicture
\startaxis[
    xmin=0,xmax=96,ymin=0,ymax=96,
    xtick distance=16,
    ytick distance=16,
]
\addplot [
    only marks,
    mark=*,
    mark size=1pt,
    mark options={draw=red!62!black,fill=white!62!black},
] table {programs/cipher-asymmetric/ecc-fp97.dat};
\stopaxis
\stoptikzpicture
}{\darkred $\mathbb{F}_{97}$}
{
\starttikzpicture
\startaxis[
    xmin=0,xmax=126,ymin=0,ymax=126,
    xtick distance=21,
    ytick distance=21,
]
\addplot [
    only marks,
    mark=*,
    mark size=1pt,
    mark options={draw=red!62!black,fill=white!62!black},
] table {programs/cipher-asymmetric/ecc-fp127.dat};
\stopaxis
\stoptikzpicture
}{\darkred $\mathbb{F}_{127}$}
\stopcombination
\stopplacefigure

\startsubsubject[title={有限域上椭圆曲线的点加法},reference=subsubsec:ec-fp-add]

显然,为了使得点加法在$\mathbb{F}_p$域上依然有效,我们需要对$\mathbb{F}_p$域上三点共线的定义作一些小小的修改。
在实数域,三点共线意味着能够找到一条直线将三个点连在一起。当然,在$\mathbb{F}_p$域中的直线,与实数域中的是有所不同的。
不太严谨地说,$\mathbb{F}_p$中的直线是满足方程$y \equiv kx+d \pmod{p}$的点$(x,y)$的集合。


\startplacefigure[title={椭圆曲线$y^2=x^3-3x+10$在有限域$\mathbb{F}_{127}$上的点加法},reference=fig:ecc-fp-add]
\switchtobodyfont[8pt]
\pgfplotsset{
    grid=major,
    scale only axis=false,
    width=240,
    unbounded coords=jump,
    clip=false,
    axis equal image=true,
}
\midaligned{
\starttikzpicture
\startaxis[
    xmin=0,xmax=126,ymin=0,ymax=126,
    xtick distance=21,
    ytick distance=21,
]
\addplot[
    only marks,
    mark=*,
    mark size=1pt,
    mark options={draw=red!62!black,fill=white!62!black},
] table {programs/cipher-asymmetric/ecc-fp127.dat};

% \addplot[green!62!black,thick,domain=0:10,samples=11] {mod(4*x+83,127)};
\addplot[green!62!black,thick] coordinates {(0,36) (12,0)};
\addplot[green!62!black,thick] coordinates {(12.333333,126) (54.333333,0)};
\addplot[green!62!black,thick] coordinates {(54.666667,126) (96.666667,0)};
\addplot[green!62!black,thick] coordinates {(97,126) (126,39)};

\addplot[
    nodes near coords,
    every node near coord/.append style={anchor=north},
    only marks,
    mark=*,
    mark size=1.6pt,
    mark options={draw=green!62!black,fill=white!62!black},
    point meta=explicit symbolic,
] coordinates {
    (13,124) [$P$]
    (80,50)  [$Q$]
    (43,34)  [$R$]
};

\addplot[
    nodes near coords,
    every node near coord/.append style={anchor=south},
    only marks,
    mark=*,
    mark size=1.6pt,
    mark options={draw=blue!62!black,fill=white!62!black},
    point meta=explicit symbolic,
] coordinates {
    (43,93) [$-R$]
};

\addplot[blue!62!black,thick,dashed] coordinates {(43,34) (43,93)};
\stopaxis
\stoptikzpicture
}
\stopplacefigure

有限域上的椭圆曲线点加法保留了所有我们已知的特性:

\startitemize
\item 单位元的定义:$P + 0 = 0 + P = P$。

\item 逆元的定义:$P + (-P) = 0$。

\item 对于一个非$0$的点$P$,逆元$-P$是横坐标相同但是纵坐标相反的点。或者还有一种方式,$-P=(x_P,-y_P \pmod{p})$。
举个例子,如果曲线在${\mathbb F}_{127}$上有一个点$P=(2,5)$,逆元是$-P=(2,-5 \pmod{127})=(2,122)$。
\stopitemize

\in{图}[fig:ecc-fp-add]展示了有限域$\mathbb{F}_{127}$上$y^2 \equiv x^3-3x+10 \pmod{127}$的所有点。请注意,
连接点$P=(13,124)$和$Q=(80,50)$的直线$y \equiv -3x+36 \pmod{127}$在图中多次重复,这是因为对127取模的原因。
方程将$P$、$Q$连接上之后,和域中的$R=(43,34)$“相交”,$R$的逆元为$-R=(43,-34 \pmod{127})=(43,93)$,
也就有$(13,124) + (80,50) = (43,93)$。

\stopsubsubject

\startsubsubject[title={有限域上椭圆曲线的代数加法},reference=subsubsec:ec-fp-algebra]

除了在每一个表达式后面加上一个$\pmod{p}$的操作以外,其它与\in[subsubsec:ecc-algebra]{代数加法}一节中所描述的步骤都
相同。因此,令$P=(x_P,y_P)$、$Q=(x_Q,y_Q)$、$R=(x_R,y_R)$,我们可以按如下方程计算$P+Q=-R$:

\startformula
\startmathalignment[n=4,align=middle]
\NC x_R \NC \equiv \NC (k^2-x_P-x_Q) \pmod{p}        \NR
\NC y_R \NC \equiv \NC [y_P + k(x_R - x_P)] \pmod{p} \NR
\NC     \NC \equiv \NC [y_Q + k(x_R - x_Q)] \pmod{p} \NR
\stopmathalignment
\stopformula

\startitemize
\item 如果$P \neq Q$,斜率$k$的形式如下:

\startformula
k \equiv (y_P - y_Q)(x_P - x_Q)^{-1} \pmod{p}
\stopformula

这里,$(x_P - x_Q)^{-1}$为$x_P-x_Q$关于$p$的模反元素。模反元素的计算可以使用扩展欧几里得算法求得。

\item 如果$P=Q$,斜率$k$为:
\startformula
k \equiv (3x_P^2+a)(2y_P)^{-1} \pmod{p}
\stopformula

\stopitemize

这里,你会发现数学的美妙之处:把椭圆曲线从实数域转换到有限域之后,点的几何加法和代数加法的公式表现出惊人的相似。
\stopsubsubject

\startsubsubject[title={数乘和循环子群},reference=subsubsec:ec-cylic]

在有限域$\mathbb{F}_p$上的椭圆曲线的乘法有个很有意思的属性。取一个曲线:$y^2 \equiv x^3+2x+3 \pmod{97}$和
点$P=(3,6)$,现在来计算$P$的所有倍数:

\startformula
\startmathalignment[n=4,align=middle]
\NC 0P \NC = \NC 0       \NR
\NC 1P \NC = \NC (3,6)   \NR
\NC 2P \NC = \NC (80,10) \NR
\NC 3P \NC = \NC (80,87) \NR
\NC 4P \NC = \NC (3,91)  \NR
\NC 5P \NC = \NC 0       \NR
\NC 6P \NC = \NC (3,6)   \NR
\NC 7P \NC = \NC (80,10) \NR
\NC \cdots \NC \NC       \NR
\stopmathalignment
\stopformula

\startplacefigure[title={$P=(3,6)$的所有倍乘的取值只有5个点然后不停地循环重复},reference=fig:p-multiple-cylic]
\midaligned{
\starttikzpicture
[pre/.style={<-,shorten <=1pt,>=stealth',very thick},
post/.style={->,shorten >=1pt,>=stealth',very thick},
dot/.style={fill=red!62!black,circle,minimum size=3pt,inner sep=0},
label/.style={black,font=\switchtobodyfont[8pt]}]

\draw[pre,rotate=90,red!62!black] (5:2.2cm) arc[start angle=5,end angle=355,radius=2.2cm];
\node [dot] () at (90:2.2cm) {};

\draw[thick,red!62!black] (90:1.4cm) -- +(90:2mm) node[label] at (90:1.8cm) {$0P$} 
node[label,anchor=south] at (90:2.3cm) {$0$};
\draw[thick,red!62!black] (18:1.4cm) -- +(18:2mm) node[label] at (18:1.8cm) {$1P$} 
node[label,anchor=south west] at (18:2.3cm) {$(3,6)$};
\draw[thick,red!62!black] (306:1.4cm) -- +(306:2mm) node[label] at (306:1.8cm) {$2P$} 
node[label,anchor=north west] at (306:2.3cm) {$(80,10)$};
\draw[thick,red!62!black] (234:1.4cm) -- +(234:2mm) node[label] at (234:1.8cm) {$3P$} 
node[label,anchor=north east] at (234:2.3cm) {$(80,87)$};
\draw[thick,red!62!black] (162:1.4cm) -- +(162:2mm) node[label] at (162:1.8cm) {$4P$} 
node[label,anchor=south east] at (162:2.3cm) {$(3,91)$};
\stoptikzpicture
}
\stopplacefigure

到此,我们发现了两个规律:(1)$P$的倍乘只有5个取值,永远不会出现第6个,如\in{图}[fig:p-multiple-cylic]。
(2)这些取值们是循环重复着的。我们可以写成这样:

\startformula
\startmathalignment[n=3,align={right,middle,left}]
\NC 5kP     \NC = \NC 0  \NR
\NC (5k+1)P \NC = \NC P  \NR
\NC (5k+2)P \NC = \NC 2P \NR
\NC (5k+3)P \NC = \NC 3P \NR
\NC (5k+4)P \NC = \NC 4P \NR
\NC \cdots  \NC   \NC    \NR
\stopmathalignment
\stopformula

也可以把这5个式子“压缩”成一个:$kP = (k \pmod{5})P$。这个规则同样适用于所有的点,不仅仅是对$P=(3,6)$。
事实上,对于任意的$P$:

\startformula
nP + mP = \underbrace{P + P + \cdots + P}_{\mathrm{n~times}} + \underbrace{P + P + \cdots + P}_{\mathrm{m~times}} = (n+m)P
\stopformula

这意味着:如果我们将$P$的倍乘进行相加,我们获得的仍然是$P$的倍数。这个性质非常重要,它证明了{\bf $nP$的集合是
椭圆曲线形成的群里的一个具有循环性质的子群}。这里的点$P$叫做循环子群的{\it 基点}(base point)。

\stopsubsubject

\startsubsubject[title={子群的阶},reference=subsubsec:ec-group-order]

首先,我们要定义一下在一个群有多少个点就叫做这个群的“阶”(order)。穷举椭圆曲线在有限域${\mathbb F}_p$中所有可能的值
的时间复杂度为$O(p)$。当$p$很大的时候,这算下来就很慢很慢,有些不太可行。好在有一个更快的算法来计算阶---Schoof算法。
这里,我们不对Schoof算法的细节进行展开,只需要知道它的复杂度是多项式时间。

在循环的子群里我们可以下一个新的、与前面的定义相等的定义,由$P$生成子群的阶是满足条件$nP=0$的最小的正整数$n$。前面的
例子中,$5P=0$,那由$P=(3,6)$生成的子群的阶就等于5。由$P$生成的子群的阶不能使用Schoof的算法,因为这个算法只能在整个
椭圆曲线上生效,在子群上无效。

由$P$生成的子群的阶和椭圆曲线是有联系的,{\it 群论中的拉格朗日定理告诉我们,子群的阶是父群的阶的因子}。换句话说,如果
一个椭圆曲线包含$N$个点,它的一个子群包含$n$个点,那么$n$是$N$的因子。

举个例子,在$\mathbb{F}_{37}$上的曲线$y^2=x^3-x+3$的阶是$N=42$。它的子群的阶则可能是$n=1,2,3,6,7,14,21,42$中
的一个。如果我们代入曲线上的点$P=(2,3)$,我们可以发现$P \neq 0$、$2P \neq 0$、$\cdots$、$7P=0$。因此,由$P$生成
的子群的阶是7。

\stopsubsubject

\startsubsubject[title={找基点和离散对数},reference=subsubsec:ec-base-point]

在ECC算法中,我们想找到一个阶数比较大的子群。所以通常呢,我们会选择一条椭圆曲线,然后去计算它的阶$N$,选择一个以较大的
因子作为子群的阶$n$,最终,依此找到一个合适的基点。也就是说,我们的计算步骤并不是先选择一个基点然后去计算它的阶,而是
反过来操作的。

首先,根据群论上的拉格朗日定理,我们知道,$h=N/n$里的$h$永远是一个整数(因为$n$整除$N$)。我们把$h$叫做{\it 辅因子}
(cofactor of the subgroup)。

现在,思考一下对于椭圆曲线中的每一个点,我们有$NP=0$,且$N=hn$。因此,我们可以写成:$n(hP)=0$。假设$n$是质数,这个
方程式告诉我们:如果$G=hP \neq 0$,则生成了一个阶为$n$的子群。若$G=hP=0$,则子群的阶是1,需要重新另外选择一个$P$。

现在我们总结一下寻找基点的算法:

\startitemize[n]
\item 计算椭圆曲线的阶$N$。
\item 选择一个阶为$n$的子群,$n$必须是质数且必须是$N$的因子。
\item 计算辅因子$h=N/n$。
\item 在曲线上选择一个随机的点$P$。
\item 计算$G=hP$。
\item 若$G=0$,那么回到步骤4。否则,我们已经找到了阶为$n$和辅因子是$h$的子群的生成器或基点。
\stopitemize

请注意,上面这个算法仅仅适用于$n$是质数的情况下。如果$n$不是质数,那么$G$的阶可以是$n$的任何一个因子。

在\about[subsubsec:scalar-multi]一节中,我们已经引出了椭圆曲线的对数问题。现在,回到有限域上的椭圆曲线,我们
可以提出相同的问题:如果我们已知$P$和$G$,怎么计算$h$呢?这个问题,就是有限域上椭圆曲线的{\it 离散对数问题}。到
目前为止,没有找到一个能在多项式时间内解出来的算法。这样一来,这个数学难题就奠定了椭圆曲线加密的安全基础。

我们知道在有限域上计算点的数乘是一个容易的过程,但是离散对数问题却是非常难的,接下来,我们就来看看这些理论是如何应用
在密码学上的。
\stopsubsubject
\stopsubsection

\startsubsection[title={椭圆曲线加密算法},reference=sec:ecc-crypto]

椭圆曲线加密算法建立在有限域上的椭圆曲线所形成的循环子群上,因此,我们的算法需要以下几个参数:

\startitemize
\item 质数$p$:用于确定有限域的范围;
\item 椭圆曲线方程中的$a$和$b$;
\item 用于生成子群的的基点$G$;
\item 子群的阶$n$;
\item 子群的辅助因子$h$。
\stopitemize

通常,我们使用六元组$(p, a, b, G, n, h)$来定义这些参数。

\startsubsubject[title={密钥对的生成}]

\startitemize[n]
\item 接收方选取椭圆曲线的参数$p$、$a$、$b$,并寻找椭圆曲线上一点作为基点$G$。
\item 接收方在$\{1,\ldots,n-1\}$范围内随机选择整数$d$作为私钥。
\item 接收方计算$H=dG$,$H$即为公钥。
\item 接收方把椭圆曲线参数$p$、$a$、$b$,基点$G$,以及公钥$H$传给发送方。
\stopitemize

接收方知道了私钥$d$和基点$G$(还有主要参数中的其他参数),求得公钥$H$是很容易的。相反,发送方知道公钥$H$和基点$G$,
想要求得私钥$d$却是很困难的,因为这要求解决离散对数问题。
\stopsubsubject

\startsubsubject[title={加解密过程}]

\startitemize[n]
\item 发送方选择随机数$r$,将明文消息编码到椭圆曲线上的点$M$。
\item 发送方生成密文$C$,该密文是一个点对,$C=\{rG, M+rH\}$。
\item 发送方将密文$C$传给接收方。
\item 接收方收到密文$C$后进行解密计算:$M+rH-d(rG)=M+r(dG)-d(rG)=M$。
\item 接收方对$M$解码还原出明文消息。
\stopitemize
\stopsubsubject

\stopsubsection
\stopsection

\stopcomponent


% https://zhuanlan.zhihu.com/p/66794410
% https://cdn.rawgit.com/andreacorbellini/ecc/920b29a/interactive/reals-add.html
% https://andrea.corbellini.name/2015/05/17/elliptic-curve-cryptography-a-gentle-introduction/
% https://ehds.github.io/uploads/papers/ecc.pdf
% https://www.secg.org/
