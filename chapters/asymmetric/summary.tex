\startcomponent summary

\startsection[title={本章小结}]
\index{summary}

在这一章,我们首先引出了对称加密算法的密钥配送问题,紧接着讨论了解决密钥配送问题的方法,并详细介绍了DH密钥交换算法、
以及RSA和ECC两种非对称加密算法(公钥加密算法)。对于公钥加密算法中所涉及的数学数论知识和椭圆曲线进行了详细的解释。
通过本章的学习,我们了解到:

\startitemize[n,packed,broad]
\item DH密钥交换算法能够让通信双方在不安全的信道上创建出相同的共享密钥,这个共享密钥可以作为对称加密算法的密钥来加密
通信内容,解决了密钥传送问题。

\item 受DH密钥交换算法的启发,RSA开创了非对称加密算法的先河,这种新的加密模式有两把密钥,一把是公钥,还有一把是私钥。
公钥和私钥一一对应,有一把公钥就必然有一把与之对应的、独一无二的私钥,反之亦成立。公钥是公开的,任何人都可以拿到公钥,
但是私钥需要妥善保管。通常,信息发送方用公钥对消息进行加密,接收方用私钥可以解开公钥加密的消息。由于只有接收方有私钥,
这样一来,就不存在密钥配送的问题了。RSA加密算法的安全性建立在大数的质因数分解的困难性之上。

\item ECC是一种基于椭圆曲线数学的非对称加密算法。我们首先对实数域上的椭圆曲线进行了详细的描述,然后扩展到有限域上的
椭圆曲线,并讨论了有限域上的计算点的数乘,随后引出了离散对数问题。离散对数问题奠定了椭圆曲线安全性的基础。和RSA相比,
ECC的主要优势是可以使用较小的密钥长度提供相当等级的安全性。基于椭圆曲线加密的算法有ECDH、ECMQV、ECDSA等。
\stopitemize

\stopsection

\stopcomponent