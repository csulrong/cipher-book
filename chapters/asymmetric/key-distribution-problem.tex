\startcomponent key-distribution-problem

\startsection[title={密钥配送问题},reference=sec:key-distribution-problem]
\index{sec:key-distribution-problem}

我们先假设一个情景:发送方需要将一封非常重要的秘密邮件发送给接收方。发送方想到使用某个
对称加密算法对信件内容进行加密后发送给发送方。但是,他立马意识到一个问题,接收方必须知道
加密算法的密钥才能解密出邮件的真实内容。于是,怎么把密传递给接收方就变成了一个必须直面
解决的问题。但是,怎么才能安全地将密钥从发送方传递给接收方呢?下面我们介绍一下解决密钥配
送问题的几个方法。

\startitemize

\head {\bf 事先共享密钥}

解决密钥配送问题的最简单方法就是事先共享密钥,也就是发送方提前将密钥告诉接收方。如果他们
两个离得很近,那没有问题,直接线下见面沟通就可以了。

如果他们分隔两地那就麻烦了。因为邮寄或者远程传输的过程中,密钥可能会被劫持。退一步说,
即便能够安全有效地共享密钥,也会存在一个密钥保存的问题,因为每两个人之间进行通信都需要
一个完全不同的密钥。假设邮件是发送方发给他的客户的商务合同,当客户数量巨大时,则需要
保存一个相当大数量的密钥。实际操作起来非常不便,也容易出错。

\head {\bf 密钥中心分配密钥}

为了解决保存大数量的密钥的问题。可以考虑采用密钥中心来对密钥进行集中管理。我们可以将密钥
中心看成是一个服务器,它里面保存了每一个人的密钥信息,下面我们看一下具体的通信流程:

  \startitemize[n,packed]
  \item 发送方和接收方需要进行通信;
  \item 密钥中心随机生成一个密钥,这个密钥将会是发送方和接收方本次通信要使用的临时密钥;
  \item 密钥中心取出保存好的本次通信要使用的密钥;
  \item 密钥中心将临时密钥使用发送方的密钥加密后,发给发送方;
  \item 密钥中心将临时密钥使用接收方的密钥加密后,发给接收方;
  \item 发送方收到加密后的数据,使用自己的密钥解密后,得到临时密钥;
  \item 接收方收到加密后的数据,使用自己的密钥解密后,也得到临时密钥;
  \item 发送方和接收方可以使用这个临时密钥自由通信了。
  \stopitemize

特别要注意的是,这里的临时密钥的使用方法很巧妙,后面我们讲到大家最常用的HTTPS通信协议时,
会再次领会到这个临时密钥的巧妙使用。

密钥中心很好,但是也有缺点,首先,密钥中心的密钥是集中管理的,一旦被攻破,所有人的密钥都会
暴露。其次,所有的通信都要经过密钥中心,可能会造成性能瓶颈。

% https://en.wikipedia.org/wiki/Diffie%E2%80%93Hellman_key_exchange
\head {\bf 使用Diffie-Hellman密钥交互}

Diffie-Hellman通过交互一些信息,双方来生成相同的密钥。具体的细节我们后在后面的章节中讲到。

\head {\bf 使用非对称加密中的私钥}

密码配送的原因就在于对称加密使用的密钥是相同的。如果有种加密算法能够使用不同的密钥加密和解密,
这个问题是不是就解决了呢?

在非对称加密算法中,需要生成两个密钥,一个是{\it 公开密钥}(public key),另外一个是{\it 私有密钥}
(private key)。公钥用作加密,私钥用作解密。使用公钥对明文加密后的密文,只能用相对应的私钥
才能解密还原出真实的明文。由于加密和解密需要两个不同的密钥,故被称为非对称加密。不同于对称加密
中加密和解密都使用同一个密钥,非对称加密算法中的公钥可以公开发布,但私钥必须由用户自行秘密保管,
因此,又被称为公钥加密算法。

回到刚才发送秘密邮件的问题,如果接收方事先用非对称加密算法生成了公钥和私钥,并把公钥发给了
发送方,则发送方可以将邮件使用公钥进行加密,然后发给接收方,这个邮件只有接收方才能解密。即使
公钥和加密后的邮件被任何第三方截获了也无法解密出邮件内容。

当然这里也有一个问题,就是邮件发送方要确保生成的公钥的确是邮件接收方发出来的。这个问题的解决
方法我们会在后面的章节再深入讨论。非对称加密还有一个问题就是加密性能大约只有对称加密算法的
几百分之一,这使得它不太适合大批量数据的加密。

基于公开密钥加密的特性,它还能提供数字签名的功能,使电子文件可以得到如同在纸本文件上亲笔签署
的效果。关于数字签名,我们将在后面的章节中进行详细的阐述。

\stopitemize

\stopsection

\stopcomponent
