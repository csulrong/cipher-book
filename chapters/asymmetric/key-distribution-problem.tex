\startcomponent key-distribution-problem

\startsection[title={密钥配送问题},reference=sec:key-distribution-problem]
\index{sec:key-distribution-problem}

我们先假设一个情景:发送方需要将一封非常重要的秘密邮件发送给接收方。发送方想到使用某个
对称加密算法对信件内容进行加密后发送给发送方。但是,他立马意识到一个问题,接收方必须知道
加密算法的密钥才能解密出邮件的真实内容。于是,怎么把密钥传递给接收方就变成了一个必须直面
解决的问题。但是,怎么才能安全地将密钥从发送方传递给接收方呢?下面我们介绍一下解决密钥配
送问题的几个方法。

\startitemize

\head {\bf 事先共享密钥}

解决密钥配送问题的最简单方法就是事先共享密钥,也就是发送方提前将密钥告诉接收方。如果他们
两个离得很近,那没有问题,直接私下见面沟通就可以了。倘若他们分隔两地那就麻烦了。因为邮寄
或者远程传输的过程中,密钥可能会被劫持。

退一步说,即便能够安全有效地共享密钥,也会存在一个密钥保存的问题,因为每两个人之间进行
通信都需要一个完全不同的密钥。假设邮件是发送方发给他的客户的商务合同,当客户数量巨大时,
则需要保存一个相当大数量的密钥库。实际操作起来非常不便,也容易出错。

\head {\bf 密钥中心分配密钥}

为了解决保存大数量的密钥的问题。可以考虑采用密钥中心来对密钥进行集中管理。我们可以将密钥
中心看成是一个服务器,它里面保存了每一个人的密钥信息,下面我们看一下具体的通信流程:

  \startitemize[n]
  \item 发送方和接收方需要进行通信;
  \item 密钥中心随机生成一个密钥,这个密钥将会是发送方和接收方本次通信要使用的临时密钥;
  \item 密钥中心取出保存好的本次通信要使用的密钥;
  \item 密钥中心将临时密钥使用发送方的密钥加密后,发给发送方;
  \item 密钥中心将临时密钥使用接收方的密钥加密后,发给接收方;
  \item 发送方收到加密后的数据,使用自己的密钥解密后,得到临时密钥;
  \item 接收方收到加密后的数据,使用自己的密钥解密后,也得到临时密钥;
  \item 发送方和接收方可以使用这个临时密钥自由通信了。
  \stopitemize

特别要注意的是,这里的临时密钥的使用方法很巧妙,后面我们讲到大家最常用的HTTPS通信协议时,
会再次领会到这个临时密钥的巧妙使用。

密钥中心很好,但是也有缺点,首先,密钥中心的密钥是集中管理的,一旦被攻破,所有人的密钥都会
暴露。其次,所有的通信都要经过密钥中心,可能会造成性能瓶颈。

% https://en.wikipedia.org/wiki/Diffie%E2%80%93Hellman_key_exchange
\head {\bf 使用DH密钥交换}

DH(Diffie-Hellman)密钥交换算法能够让通信双方在不安全的信道上通过交互一些信息来生成相同的
密钥。算法的具体细节我们将在\about[sec:dh]一节中讲到。

\head {\bf 使用非对称加密中的私钥}

密码配送的原因就在于对称加密使用的密钥是相同的。如果有种加密算法能够使用不同的密钥加密和解密,
这个问题是不是就解决了呢?

在非对称加密算法中,需要生成两个密钥,一个是{\it 公开密钥}(public key),另外一个是{\it 私有密钥}
(private key)。公钥用作加密,私钥用作解密。使用公钥对明文加密后的密文,只能用相对应的私钥
才能解密还原出真实的明文。由于加密和解密需要两个不同的密钥,故被称为非对称加密。不同于对称加密
中加密和解密都使用同一个密钥,非对称加密算法中的公钥可以公开发布,但私钥必须由用户自行秘密保管,
因此,又被称为{\it 公钥加密算法}。

回到刚才发送秘密邮件的问题,如果接收方事先用非对称加密算法生成了公钥和私钥,并把公钥发给了
发送方,则发送方可以将邮件使用公钥进行加密,然后发给接收方,这个邮件只有接收方才能解密。即使
公钥和加密后的邮件被任何第三方截获了也无法解密出邮件内容。

当然这里也有一个问题,就是邮件发送方要确保生成的公钥的确是邮件接收方公开发布的。这个问题的解决
方法我们会在后面的章节再深入讨论。非对称加密还有一个问题就是加密性能大约只有对称加密算法的
几百分之一,这使得它不太适合大批量数据的加密。

基于公开密钥加密的特性,它还能提供数字签名的功能,使电子文件可以得到如同在纸本文件上亲笔签署
的效果。关于数字签名,我们将在后面的章节中进行详细的阐述。

\stopitemize

\stopsection

\stopcomponent
