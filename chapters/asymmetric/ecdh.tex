\startcomponent ecdh

\startsection[title={ECDH},reference=sec:ecdh]
\index{sec:ecdh}

ECDH是\about[sec:dh]在使用椭圆曲线加密方法后的变种。DH算法由于计算性能不佳,因为需要做大量的乘法,为了提升DHE算法
的性能,所以就出现了现在广泛用于密钥交换算法---ECDH算法。ECDH算法是在DH算法的基础上利用了ECC椭圆曲线特性,可以用更少
的计算量计算出公钥,以及最终的会话密钥。甲乙通信双方使用ECDH密钥交换算法的过程:

\startitemize
\item 甲乙双方事先确定好椭圆曲线的参数$a$、$b$和有限域参数$p$,以及曲线上的基点$G$,这些参数都是公开的。

\item 甲乙双方各自生成自己的私钥和公钥。甲方的私钥为$d_A$,公钥为$H_A=d_A G$;乙方的私钥为$d_B$,公钥为$H_B=d_B G$。

\item 甲乙双方交换各自的公钥$H_A$和$H_B$。

\item 最后,甲方计算$S=d_A H_B$(用自己的私钥点乘乙的公钥);同样,乙方计算$S=d_B H_A$(用自己的私钥点乘甲的公钥)。
双方求得的$S$是一致的,因为$d_A H_B = d_A (d_B G) = d_B (d_A G) = d_B H_A$。
\stopitemize

这个过程中,如果双方的私钥都是随机、临时生成的,我们将之称为{\it ECDHE算法},E表示短暂的(ephemeral),指的是交换的
密钥是暂时的动态的,而不是固定的静态的。

静态的DH或者ECDH算法里有一方的私钥在每次密钥协商的时候都是固定不变的,通常是服务器方固定私钥不变,客户端的私钥则是随机
生成的。于是,在静态模式下,DH或ECDH交换密钥时就只有客户端的公钥是变化,而服务端公钥是不变的,那么随着时间延长,黑客就
会截获海量的密钥协商过程的数据,黑客就可以依据这些数据暴力破解出服务器的私钥,然后就可以计算出会话密钥了。这样一来所有
的历史会话中传输的数据以及后续会话中传输的数据都将被黑客破解。所以,静态模式的DH或ECDH算法不具备{\it 前向安全性}。


现在常用的是ECDHE交换算法。在这种模式下,每次会话的的私钥随机生成的,没有任何关系。黑客所截获的密钥协商过程的数据只在某
一次会话中使用,他只能通过解决离散对数问题来破解私钥。退一步说,即便有个厉害的黑客破解了某一次通信过程的私钥,其他通信过程
的私钥仍然是安全的,因为每个通信过程的私钥都是独立的,这样就保证了{\it 前向安全}。
\stopsection