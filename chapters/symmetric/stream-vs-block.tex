\startcomponent stream-vs-block

\startsection[title={流密码和分组密码},reference=sec:stream-vs-block]
\index{sec:stream-vs-block}

在对称加密算法中,按照加密方式的不同,可以分为流密码和分组密码。

在流密码加密方式下,明文称为明文流,以比特序列的方式表示。加密时候,先由种子密钥生成一个密钥流。然后利用加密算法
把明文流和密钥流进行加密,产生密文流。流密码一般以1比特、8比特、或32比特为单位进行加密和解密。加密过程所需要的
密钥流由种子密钥通过密钥流生成器产生。

流密码的主要原理是通过随机数发生器产生性能优良的伪随机序列,使用该序列加密明文流得到密文流。若密钥流是无周期、
无限长随机序列,则每一个明文都对应一个随机的加密密钥,理论上,流密码是“一次一密”密码体制,也就是绝对安全的。
实际应用中密钥流都用有限存储和复杂逻辑的电路产生,此时它的生成器只有有限个状态,这样,它早晚要回到初始状态而呈现
出一定长度的周期,其输出也就是周期序列。所以,实际应用中的流密码不会实现“一次一密”密码体制,但若生成的密钥流周期
够长,随机性好,其安全强度还是能保证的。因此,密钥流生成器的设计是流密码的核心,流密码的安全强度取决于密钥流的周期、
复杂度、随机(伪随机)性等。流密码涉及许多理论知识,提了很多设计原理,得到了广泛分析,但很多研究成果并没有全部
公开,可能是因为目前流密码主要用于军事和外交。日前,公开的流密码算法主要有RC4、SEAL等。

不同于流密码加密方式,分组密码(又称块密码)在把明文变成二进制比特序列后,将其划分成若干个{\it 固定长度}的组,
不足位用0补全。分组长度通常是64比特、128比特、192比特和256比特等。分组密码对每个分组逐个依次进行加密操作。
分组长短影响密码强度,分组长度不能太短也不能太长。既要便于操作与运算,又要保证密码的安全性。本章将详细介绍几种
分组密码算法:DES、3DES、AES和Blowfish。分组密码算法一次只能处理固定长度的分组,但是我们需要加密的明文长度
可能会超过分组密码的分组长度,这时就需要对分组密码算法进行迭代,以便将一段很长的明文全部加密,迭代的方法就称为
分组密码的分组模式 (mode)。我们将把分组密码的模式留在下一章进行讲解。

\stopsection

\stopcomponent