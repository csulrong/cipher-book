\startcomponent summary

\startsection[title={本章小结}]

本章,我们介绍了对称加密,同时介绍了对流密码和分组密码两种加密方式。进而,详细阐述了RC4、DES、3DES、AES和
ChaCha20密码算法的加解密原理及过程。

然而,用对称加密进行通信时,也有着其无法逾越的弱点,也就是密钥的配送问题,即如何将密钥安全地发送给通信对方。
非对称加密(即公钥密码)的技术就完美地解决了密钥配送问题。我们将在下一章对密钥配送问题和非对称加密进行详细的
探讨。

另外,本章所介绍的分组密码算法,都需要对明文按照固定的长度划分的分组进行加密。当需要加密的明文长度超过分组长度
时,就需要对多个分组进行迭代,我们将在第?章探讨分组密码的分组模式。当明文长度不为分组长度的倍数时,对明文按照
分组长度划分数据块之后,最后一个数据块的长度就不够分组的长度,我们也将在第?章讨论分组密码的填充模式。

\stopsection

\stopcomponent