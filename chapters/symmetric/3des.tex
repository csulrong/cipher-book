
\startcomponent 3des

\startsection[title={3DES加密算法},reference=sec:3des]
\index{sec:3des}

由于计算机运算能力不断增强,DES密码的密钥长度(56位)变得容易被暴力破解。3DES (Triple DES, 三重DES) 
为了增强DES密码的强度,将DES重复三次来增加DES的密钥长度以避免被暴力破解。因此,3DES并不是一种全新的
分组密码算法。

\startsubsection[title={3DES的总体结构},reference=subsec:3des-structure]

3DES加密算法的总体结构如\in{图}[fig:3des]所示。在加密和解密过程中,都需要经过3次DES处理。3DES使用
“密钥包”,其包含3个DES密钥,K1,K2和K3,均为56位(除去奇偶校验位)。

\startplacefigure[title={3DES密码算法总体结构},reference=fig:3des]
\midaligned{
\starttikzpicture
[pre/.style={<-,shorten <=1pt,>=stealth',semithick},
 post/.style={->,shorten >=1pt,>=stealth',semithick},
 labelnode/.style={inner sep=0,font=\Tiny},
 elemnode/.style={draw=red!62.5!black,thick,fill=white!62.5!black,
 minimum width=4em,minimum height=2em,font=\Tiny}]

\matrix[row sep=1cm,column sep=8mm] {
% row 1
 &
\node [elemnode] (c-des1) {DES}; &
\node [elemnode] (c-des2) {DES$^{-1}$}; &
\node [elemnode] (c-des3) {DES}; &
  \\
% row 2
\node [elemnode] (plaintext) {明文分组}; &
\node [elemnode] (k1) {密钥K1}; &
\node [elemnode] (k2) {密钥K2}; &
\node [elemnode] (k3) {密钥K3}; &
\node [elemnode] (ciphertext) {密文分组}; \\
% row 3
&
\node [elemnode] (d-des1) {DES$^{-1}$}; &
\node [elemnode] (d-des2) {DES}; &
\node [elemnode] (d-des3) {DES$^{-1}$}; &
\\
};

\draw [post] (plaintext.north) |- (c-des1.west);
\draw [post] (c-des1.east) -- (c-des2.west);
\draw [post] (c-des2.east) -- (c-des3.west);
\draw [post] (c-des3.east) -| (ciphertext.north);


\draw [post] (ciphertext.south) |- (d-des3.east);
\draw [post] (d-des3.west) -- (d-des2.east);
\draw [post] (d-des2.west) -- (d-des1.east);
\draw [post] (d-des1.west) -| (plaintext.south);

\draw [post] (k1.north) -- (c-des1.south);
\draw [post] (k1.south) -- (d-des1.north);
\draw [post] (k2.north) -- (c-des2.south);
\draw [post] (k2.south) -- (d-des2.north);
\draw [post] (k3.north) -- (c-des3.south);
\draw [post] (k3.south) -- (d-des3.north);

\startscope[on background layer]
\node [fill=white!62.5!black, draw=red!62.5!black, very thick, inner xsep=3mm, inner ysep=5mm, 
 rounded corners, fit=(c-des1) (c-des3)] (encrypt) {};
\node [labelnode,below=2pt of encrypt.north] {加密}; 
\node [fill=white!62.5!black, draw=red!62.5!black, very thick, inner xsep=3mm, inner ysep=5mm,
 rounded corners, fit=(d-des1) (d-des3)] (decrypt) {};
\node [labelnode,above=2pt of decrypt.south] {解密};
\stopscope
\stoptikzpicture
}
\stopplacefigure

在加密过程中,明文分组经过3次DES处理,即DES $\rightarrow$ DES$^{-1}$ $\rightarrow$ DES,这里,
我们用DES$^{-1}$表示DES的解密。所以,在3DES加密过程中,实际的运算是:DES加密 $\rightarrow$ DES解密 
$\rightarrow$ DES加密。这看起来有点不可思议,但实际上却是有意为之。这个设计是IBM提出来的,目的是为了
让3DES能够向下兼容DES。当3DES的所有密钥K1、K2、K3都相同时,3DES也就等同于普通的DES了。这是因为前两次
DES处理(DES加密 $\rightarrow$ DES解密)完成后的结果就是最初的明文分组。

用$E$表示DES加密函数,$D$表示DES解密函数,那么3DES的加密算法可以表示为:

\startformula
ciphertext = E_{K3}(D_{K2}(E_{K1}(plaintext)))
\stopformula

也就是说,使用密钥K1进行DES加密,再用密钥K2进行DES“解密”,最后以密钥K3进行DES加密。

而解密算法则和加密过程正好相反,是以密钥K3、密钥K2、密钥K1的顺序执行的,形式上可以表示为:

\startformula
plaintext = D_{K1}(E_{K2}(D_{K3}(ciphertext)))
\stopformula

即以密钥K3解密,以密钥K2“加密”,最后以密钥K1解密。无论是加密还是解密,中间一步都是前后两步的逆。

\stopsubsection

\startsubsection[title={密钥选项},reference=subsec:3des-keys]

3DES标准定义了三种密钥选项:

\startitemize[1,packed,broad]
\item 密钥选项1:三个密钥是独立的,即$K1 \neq K2 \neq K3$。
\item 密钥选项2:$K1$和$K2$是独立的,而$K3=K1$。
\item 密钥选项3:三个密钥均相等,即$K1=K2=K3$。
\stopitemize

密钥选项1的强度最高,拥有$3 \times 56=168$个独立的密钥位。这种3DES密码算法称为DES-EDE3,EDE是
加密 (Encryption) -- 解密 (Decrpytion) -- 加密 (Encryption) 3个单词首字母的缩写。

密钥选项2的安全性稍低,拥有$2 \times 56=112$个独立的密钥位。该选项比简单的应用DES两次的强度较高,
即使用K1和K2,因为它可以防御中途相遇攻击 (Meet-in-the-Middle Attack)。这种3DES密码算法称为DES-EDE2。

% 介绍中途相遇攻击:https://zh.wikipedia.org/wiki/%E4%B8%AD%E9%80%94%E7%9B%B8%E9%81%87%E6%94%BB%E6%93%8A

密钥选项3等同于普通DES,只有56个密钥位。这个选项提供了与DES的兼容性,因为第1和第2次DES操作相互抵消了。

\stopsubsection

\startsubsection[title={3DES密码现状},reference=subsec:3des-status-quo]

银行机构目前还在普遍使用3DES,并持续开发和宣传基于3DES的标准,例如EMV。EMV三个字母分别代表Europay、
MasterCard与Visa,是制定该标准最初的三家公司。EMV是国际金融业界对于智慧支付卡与可使用晶片卡的POS终端机
及自动柜员机 (ATM) 等所制定的标准。EMV智慧卡(也称为IC卡)的信息储存在集成电路中而非过去的磁条里,但
大部分EMV卡背也有可以向下兼容的磁条。芯片可以和插入式读卡器交换数据,非接触式智能卡还可以使用射频识别
(RFID)技术在一定距离范围内交换数据。符合EMV标准的支付卡叫做芯片卡。

此外,Microsoft OneNote和Microsoft Outlook 2007使用3DES密码保护用户数据。

尽管3DES还有在使用,但是其两个独立密钥的变种DES-EDE2已经在2015年正式退出。在经过一系列安全分析和针对
3DES的实例攻击演示之后,2017年11月,NIST建议应当限制3DES用于对数量在$2^{20}$个以上的64位分组(8MB)
的数据进行加密,因此,TLS、IPSec和大文件加密不应该再使用3DES。目前,NIST正在推动2030年彻底废除3DES的
使用,但是,随着密码研究揭露的越来越多的漏洞,3DES可能会被加速退出历史舞台。

\stopsubsection

\stopsection

\stopcomponent