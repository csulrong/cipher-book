
\startcomponent rc4

\startsection[title={RC4流式加密算法},reference=sec:rc4]
\index{sec:rc4}

RC4 (Rivest Cipher 4) 是由罗纳德·李维斯特 (Ron Rivest) 在1987年开发出来的,RC4开始时是商业密码,没有
公开发表出来,但是在1994年9月份的时候,它被人匿名公开在了密码朋克 (Cypherpunks) 邮件列表上,随后又传播到了
互联网的许多站点被公开,RC4也就不再是商业秘密了,只是它的名字“RC4”仍然是一个注册商标。RC4已经成为一些常用的
协议和标准的一部分,如1997年的WEP和2003/2004年无线卡的WPA;1995年的SSL,以及后来1999年的TLS。让它如此
广泛分布和使用的主要因素是它不可思议的简单和速度,不管是软件还是硬件,实现起来都十分容易。

RC4算法的原理很简单,包括{\it 初始化算法}(Key Scheduling Algorithm, KSA)和{\it 伪随机子密码生成算法}
(Pseudo Random Generation Algorithm, PRGA) 两大部分。


\startplacefigure[title={RC4初始化算法对S盒的初始化过程示意图}, reference=fig:rc4-init]
\midaligned{
\starttikzpicture
  [pre/.style={<-,shorten <=1pt,>=stealth',semithick},
  post/.style={->,shorten >=1pt,>=stealth',semithick},
  textnode/.style={draw=red!62.5!black, thick, fill=white!62.5!black, 
  minimum width=1.8em, minimum height=1.4em, font=\Tiny},
  plainnode/.style={minimum width=1.8em,minimum height=1.4em, font=\Tiny},
  node distance=-0.2mm]

  % S: initialization
  \node [plainnode] (S0) {\type{S}};
  \node [textnode,right=2mm of S0] (S0-0) {0};
  \node [textnode,right=of S0-0] (S0-1) {1};
  \node [textnode,right=of S0-1] (S0-2) {2};
  \node [textnode,right=of S0-2] (S0-3) {3};
  \node [textnode,right=of S0-3] (S0-4) {4};
  \node [plainnode,right=4.8cm of S0-4] (S0-dots) {.........};
  \node [textnode,right=4.8cm of S0-dots] (S0-253) {253};
  \node [textnode,right=of S0-253] (S0-254) {254};
  \node [textnode,right=of S0-254] (S0-255) {255};

  % K: seed key
  \node [plainnode,below=1cm of S0] (K) {\type{K}};
  \node [textnode,right=2mm of K] (K-0) {};
  \node [textnode,right=of K-0] (K-1) {};
  \node [textnode,right=of K-1] (K-2) {};
  \node [textnode,right=of K-2] (K-3) {};
  \node [plainnode,right=of K-3] (K-dots) {...};
  \node [textnode,right=of K-dots] (K-4) {};
  \path (K-0.north west) -- (K-4.north east) node [midway,plainnode,above] (K-len) {密钥长度Length(K)};
  \draw [name path=left-y,semithick] (K-0.north west) -- +(0, 0.5cm);
  \path [name path=left-x] (K-len.west) -- +(-3em, 0);
  \draw [name intersections={of=left-y and left-x,by=L}]
    [<-,shorten <=1pt,>=stealth',semithick] (L) -- (K-len.west);
  \draw [name path=right-y,semithick] (K-4.north east) -- +(0, 0.5cm);
  \path [name path=right-x] (K-len.east) -- +(3em, 0);
  \draw [name intersections={of=right-y and right-x,by=R}]
    [->,shorten >=1pt,>=stealth',semithick] (K-len.east) -- (R);

  % T: initialized by K
  \node [plainnode,below=1.5cm of K] (T) {\type{T}};
  \node [textnode,right=2mm of T] (T-00) {};
  \node [textnode,right=of T-00] (T-01) {};
  \node [textnode,right=of T-01] (T-02) {};
  \node [textnode,right=of T-02] (T-03) {};
  \node [plainnode,right=of T-03] (T-0dots) {...};
  \node [textnode,right=of T-0dots] (T-04) {};
  
  \node [textnode,right=of T-04] (T-10) {};
  \node [textnode,right=of T-10] (T-11) {};
  \node [textnode,right=of T-11] (T-12) {T[i]};
  \node [textnode,right=of T-12] (T-13) {};
  \node [plainnode,right=of T-13] (T-1dots) {...};
  \node [textnode,right=of T-1dots] (T-14) {};

  \path [name path=T-x] (T-14.east) -- +(7.8cm,0);
  \path [name path=S0-255-y] (S0-255.south) -- +(0,-3.5cm);
  \node [name intersections={of=T-x and S0-255-y,by=X}][textnode] (T-24) at (X) {};
  \node [plainnode,left=of T-24] (T-2dots) {...};
  \node [textnode,left=of T-2dots] (T-23) {};
  \node [textnode,left=of T-23] (T-22) {};
  \node [textnode,left=of T-22] (T-21) {};
  \node [textnode,left=of T-21] (T-20) {};
  \path (T-14.east) -- (T-20.west) node [midway,plainnode] (Tdots) {......};

  % K -> T
  \foreach \i in {0,1,2,3,4} {
    \draw [->,shorten >=1pt,>=stealth',semithick] (K-\i.south) -- (T-0\i.north);
    \coordinate (X-\i) at ($ (K-\i.south) + (0, -\i*1.5mm-4mm) $);
    \draw [->,shorten >=1pt,>=stealth',semithick] (X-\i) -| (T-1\i.north);
    \coordinate (Y-\i) at ($ (X-\i) + (6cm, 0) $);
    \draw [->,shorten >=1pt,>=stealth',semithick] (Y-\i) -| (T-2\i.north);
  }

  % calculate j
  \node [plainnode,below=0.8cm of T-12] (j) {j := (j + S[i] + T[i]) mod 256};

  % S randomization
  \node [plainnode, below=2cm of T] (S1) {\type{S}};
  \node [textnode,right=2mm of S1] (S1-0) {};
  \node [textnode,right=of S1-0] (S1-1) {};
  \node [textnode,right=of S1-1] (S1-2) {};
  \node [textnode,right=of S1-2] (S1-3) {};
  \node [textnode,right=of S1-3] (S1-4) {};
  \node [plainnode,right=of S1-4] (S1-5) {};
  \node [plainnode,right=of S1-5] (S1-6) {...};
  \node [plainnode,right=of S1-6] (S1-7) {};
  \node [textnode,below=2cm of T-12] (S1-i) {S[i]};

  \node [textnode,below=2cm of T-24] (S1-255) {};
  \node [textnode,left=of S1-255] (S1-254) {};
  \node [textnode,left=of S1-254] (S1-253) {};
  \path (S1-i.east) -- (S1-253.west) node [midway,textnode] (S1-j) {S[j]};
  \path (S1-i.east) -- (S1-j.west) node [midway,plainnode] (S1-dots1) {...};
  \path (S1-j.east) -- (S1-253.east) node [midway,plainnode] (S1-dots2) {...};
  \draw [->,shorten >=1pt,>=stealth',semithick] (T-12.south) -- (j.north);
  \draw [->,shorten >=1pt,>=stealth',semithick] (S1-i.north) -- (j.south);
  \draw [->,shorten >=1pt,>=stealth',semithick] (j.east) -| (S1-j.north);
  \draw [<->,shorten >=1pt,>=stealth',semithick] 
    (S1-i.south) .. controls +(-45:1cm) and +(-135:1cm) .. (S1-j.south)
    node [plainnode,above,midway] {交换};
\stoptikzpicture
}
\stopplacefigure


\startsubsection[title={初始化算法},reference=subsec:ksa]

RC4初始化阶段,由种子密钥\type{K}对一个称之为\type{S}盒的数组进行初始化。我们用\type{Length(K)}来表示
种子密钥的长度,并假设\type{S}盒的长度为256,RC4初始化过程可以用\in{图}[fig:rc4-init]和如下伪代码来描述。

\startpseudocode[before={\starttextrule{\small\bf RC4初始化算法的伪代码描述}}]
for i from 0 to 255 do
    S[i] := i
    T[i] := K[i mod Length(K)]

j := 0
for i from 0 to 255 do
    j := (j + S[i] + T[i]) mod 256
    swap values of S[i] and S[j]
\stoppseudocode

第一个\type{for}循环对\type{S}盒和数组\type{T}进行初始化。\type{S}盒中按顺序依次填入索引序号,种子
密钥\type{K}中的字节序列依次循环填入数组\type{T}来完成对\type{T}的初始化。第二个\type{for}循环
将\type{S}盒中的数值搅乱,\type{i}确保\type{S}中的每个元素都得到处理,\type{j}的选择通过种子密钥
\type{K}来保证\type{S}的搅乱是随机的。

\stopsubsection

\startsubsection[title={伪随机密码生成算法},referenc=subsec:prga]

在RC4伪随机密码生成算法阶段,基于初始化好的\type{S}盒,对明文流加密生成密文流。算法是一个迭代的过程,
\in{图}[fig:rc4-prga]给出了一次迭代过程的示意图,整个过程用如下伪代码描述。


\startplacefigure[title={RC4伪随机密码生成算法产生密文流的过程示意图}, reference=fig:rc4-prga]
\midaligned{
\starttikzpicture
  [pre/.style={<-,shorten <=1pt,>=stealth',semithick},
  post/.style={->,shorten >=1pt,>=stealth',semithick},
  textnode/.style={draw=red!62.5!black, thick, fill=white!62.5!black, 
  minimum width=1.8em, minimum height=1.4em, font=\Tiny},
  plainnode/.style={minimum width=1.8em,minimum height=1.4em, font=\Tiny},
  node distance=-0.2mm]

  \node [plainnode] (S) {\type{S}};
  \node [textnode,right=2mm of S] (S0) {};
  \node [textnode,right=of S0] (S1) {};
  \node [textnode,right=of S1] (S2) {};
  \node [textnode,right=of S2] (S3) {};
  \node [textnode,right=of S3] (S4) {};
  \node [plainnode,right=0.5cm of S4] (dots1) {...};
  \node [textnode,right=0.5cm of dots1] (Si) {S[i]};
  \node [plainnode,right=1.5cm of Si] (dots2) {...};
  \node [textnode,right=1.5cm of dots2] (Sj) {S[j]};
  \node [plainnode,right=0.5cm of Sj] (dots3) {...};
  \node [textnode,right=0.5cm of dots3] (Sk) {S[k]};
  \node [plainnode,right=0.5cm of Sk] (dots4) {...};
  \node [textnode,right=0.5cm of dots4] (S253) {};
  \node [textnode,right=of S253] (S254) {};
  \node [textnode,right=of S254] (S255) {};
  \node [plainnode] (j) at ($ (Si.north east) + (45:1cm) $) {j := (j + S[i]) mod 256};
  \draw [->,shorten >=1pt,>=stealth',semithick] (Si.east) -| (j.south);
  \draw [->,shorten >=1pt,>=stealth',semithick] (j.east) -| (Sj.north);
  \node [plainnode,below=1cm of dots2] (Sij) {k := (S[i] + S[j]) mod 256};
  \draw [<->,shorten >=1pt,>=stealth',semithick] 
    (Si.south) .. controls +(-30:1cm) and +(-150:1cm) .. (Sj.south)
    node [plainnode,above,midway] {交换};
  \draw [->,shorten >=1pt,>=stealth',semithick] 
    ($ (Si.south) + (-2mm, 0) $) -- ($ (Sij.north) + (-2.5mm, 0) $);
  \draw [->,shorten >=1pt,>=stealth',semithick] 
    ($ (Sj.south) + (2mm, 0) $) -- ($ (Sij.north) + (2.5mm, 0) $);
  \draw [->,shorten >=1pt,>=stealth',semithick] (Sij.east) -| ($ (Sk.south) + (-2mm, 0) $);
  \node [draw=red!62.5!black,thick,fill=white!62.5!black,
    shape=circle,below=1.5cm of Sk] (xor) {\type{+}} edge [pre] (Sk);
  \node [textnode,left=1cm of xor] (input) {inputByte} edge [post] (xor);
  \node [textnode,right=1cm of xor] (output) {outputByte} edge [pre] (xor);
\stoptikzpicture
}
\stopplacefigure

\startpseudocode[before={\starttextrule{\small\bf RC4伪随机密码生成算法的伪代码描述}}]
i := 0
j := 0
while (inputByte = readByte(inStream)) != EOF:
    i := (i + 1) mod 256
    j := (j + S[i]) mod 256
    swap values of S[i] and S[j]
    k := (S[i] + S[j]) mod 256
    outputByte = inputByte ^ S[k]
    writeByte(outStream, outputByte)
\stoppseudocode

每次迭代从明文流中读入一个字节,迭代结束之后生成一个密文字节输出到密文流中。每次迭代过程中,
由\type{i}和\type{j}两个指针索引\type{S}盒中相应的字节值。指针\type{i}在每次迭代前递增,
确保\type{S}盒中的每个位置的元素都有机会参与密文字节的计算。指针\type{i}指向的\type{S}盒中的元素进一步用于
计算指针\type{j},用来“随机”挑选\type{S}盒中的另一个值。每次迭代,将\type{i}和\type{j}指向的\type{S}盒
中的两个元素相互交换来进一步打乱\type{S}盒,此步操作保证每256次迭代中\type{S}盒中的每个元素至少被交换过一次。
同时,用\type{i}和\type{j}指向的元素来计算另一个指针\type{k},用\type{k}指向的\type{S}盒中的元素和明文流
中的当前读入的字节进行异或计算得到密文字节,并输出到密文流中。

PRGA生成密文流的过程中,只采用了异或运算,由于异或运算具有对称性,即由$a \oplus b = x$,可以得到
$x \oplus b = a$和$x \oplus a = b$。因此,RC4解密也使用同一套算法。

\stopsubsection

\startsubsection[title={使用OpenSSL进行RC4加密},reference=subsec:openssl-rc4]

OpenSSL是密码工作者和开发人员的必备利器,它是一个开放源代码的软件库包,整个软件包大概可以分成三个主要的
功能部分:
\startitemize[packed]
\item openssl命令行工具
\item libencrypt加密算法库
\item libssl加密模块应用库
\stopitemize

作为一个基于密码学的安全开发包,OpenSSL提供的功能相当强大和全面,囊括了主要的密码算法、常用的密钥和
证书封装管理功能以及SSL协议,并提供了丰富的应用程序供测试或其它目的使用。在介绍完每项具体的现代密码技术之后,
本书会使用OpenSSL来展示具体的例子,帮助读者加深对密码技术的理解。

在对称加密算法方面,OpenSSL一共提供了8种对称加密算法,其中7种是分组加密算法,仅有的一种流加密算法就是RC4。
下面展示用OpenSSL命令行工具完成RC4的加密和解密。

首先,我们使用\type{openssl}对字符串\type{hello}进行加密,指定加密算法为RC4。

\startshellcode
root@localhost:~# echo -ne "hello" | openssl enc -rc4 -K "DEADBEEF" -e -nosalt -p -out hello.rc4
key=DEADBEEF000000000000000000000000
\stopshellcode

\type{echo}命令输出的字符串\type{hello}通过管道传给\type{openssl}作为输入的明文字符串被加密,\type{-ne}
选项控制\type{echo}输出的字符串最后不自动添加换行符。\type{openssl}的\type{enc}子命令用于加密,这里,
我们通过\type{-rc4}选项指定了RC4算法,并指定了密钥是\type{DEADBEEF},\type{openssl}会自动将密钥长度补齐
到128个比特位。\type{-e}是控制加密的选项参数,\type{-nosalt}表示不给密码加盐(后面,我们会详细讨论加盐密码)。
\type{-out}将加密后的密文输出到指定的文件中。

接下来,我们来看一下加密后的密文到底长得啥样。

\startshellcode
root@localhost:~# cat hello.rc4 | xxd
00000000: 7d5c e5fc 3b                             }\..;
\stopshellcode

我们将\type{hello.rc4}密文文件的内容读出来,并用\type{xxd}显示十六进制格式的密文字节,可以看到密文中有5个字节,
依次对应明文\type{hello}的5个字符。

最后,我们继续使用\type{openssl}命令对密文文件进行解密,并将解密后的结果输出到屏幕终端上。

\startshellcode
root@localhost:~# openssl enc -rc4 -K "DEADBEEF" -d -nosalt -p -in hello.rc4
key=DEADBEEF000000000000000000000000
hello
\stopshellcode

在命令行中,我们使用\type{-d}选项来表明解密操作,和加密操作一样,继续指定了RC4密码算法和密钥。并用\type{-in}选项
从文件\type{hello.rc4}中读入密文,最终还原出明文字符串\type{hello}。

\stopsubsection

\startsubsection[title={针对RC4的破解},reference=subsec:crack-rc4]

RC4算法是对称的加密算法,加密和解密的步骤都是众所周知的、固定不变的。唯一的保密性来自于种子密钥。理论上,
这个种子密钥只有通讯双方知道,但是如果第三方从某种途径获得了这个种子密钥,那么第三方可以毫不费力地用RC4来
解密他截获到的密文了。

理论上来说,RC4算法是很难被破解的。RC4中用到的种子密钥是长度在1到256之间可变的无类型字符序列,每个字节
有256种可能的取值,那么RC4算法的种子密钥的可能性就是
$256 + 256^2 + 256^3 + \cdots + 256^{256} \approx 256^{257}$
种可能性,量级在$10^{600}$以上。因此,针对RC4算法的暴力破解几乎是不可能的。

研究人员早前发现可以利用RC4中的统计偏差,导致可对加密信息中的一些伪随机字节能进行猜测。在2013年,科学家利用
这个漏洞设计了一次攻击实验:他们在2000小时内猜出一个基础身份认证cookie中包含的字符。后来技术改进后,研究人员
只需约75小时猜解就能得到$94\%$的准确率。

另外,新型针对WPA-TKIP网络的猜解攻击大概只需要花上1个小时的时间,攻击者可以任意注入、解密数据包。这项技术不仅
可以解密cookie和WiFi数据包,其他高速传输的加密数据流也有可能被解密。技术是通过向加密负载中注入数据,如每个认证
cookie或者WiFi数据包中的标准头部。攻击者会通过组合所有可能的值,通过使用统计偏差找出最有可能的组合。研究人员
表示,现在RC4加密已经不安全了,建议完全停止使用。在2015年由RFC~7465禁止在所有版本的TLS中使用。

\stopsubsection

\stopsection

\stopcomponent