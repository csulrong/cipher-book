\startcomponent yinfu

\startsection[title={中国古代人民怎么加密?}]

早期加密算法主要使用在军事中,中国历史上最早关于加密算法的记载出自于周朝兵书《六韬·龙韬》中的《阴符》和《阴书》。
其中《阴符》记载了:

\startquotation
{\it
太公曰:“主与将,有阴符,凡八等。有大胜克敌之符,长一尺。破军擒将之符,长九寸。降城得邑之符,长八寸。却敌报远之符,
长七寸。警众坚守之符,长六寸。请粮益兵之符,长五寸。败军亡将之符,长四寸。失利亡士之符,长三寸。诸奉使行符,稽留,
若符事闻,泄告者,皆诛之。八符者,主将秘闻,所以阴通言语,不泄中外相知之术。敌虽圣智,莫之能识。”
}
\stopquotation

简单来说,阴符是以八等长度的符来表达不同的消息和指令,属于密码学中的替代法(\in{图}[fig:cipher-yinfu]),在应用
中是把信息转变成敌人看不懂的符号,但知情者知道这些符号代表的含义。

\startplacefigure
  [title={阴符所蕴含的加密原理:替换法}, reference=fig:cipher-yinfu]

\midaligned{
\starttikzpicture
  [pre/.style={<-,shorten <=1pt,>=stealth',semithick},
  post/.style={->,shorten >=1pt,>=stealth',semithick},
  textnode/.style={draw=red!62.5!black, thick, fill=white!62.5!black, inner sep=2mm, font=\Tiny},
  node distance=3mm]

  \node [textnode] (dskd) {大胜克敌};
  \node [textnode, right=of dskd] (pjqj) {破军擒将};
  \node [textnode, right=of pjqj] (xcdy) {降城得邑};
  \node [textnode, right=of xcdy] (qdby) {却敌报远};
  \node [textnode, right=of qdby] (jzjs) {警众坚守};
  \node [textnode, right=of jzjs] (qlyb) {请粮益兵};
  \node [textnode, right=of qlyb] (bjwj) {败军亡将};
  \node [textnode, right=of bjwj] (slwt) {失利亡士}; 

  \node [textnode, below=1cm of dskd] (yichi)  {符长一尺} edge [pre] (dskd);
  \node [textnode, below=1cm of pjqj] (jiucun) {符长九寸} edge [pre] (pjqj);
  \node [textnode, below=1cm of xcdy] (bacun)  {符长八寸} edge [pre] (xcdy);
  \node [textnode, below=1cm of qdby] (qicun)  {符长七寸} edge [pre] (qdby);
  \node [textnode, below=1cm of jzjs] (liucun) {符长六寸} edge [pre] (jzjs);
  \node [textnode, below=1cm of qlyb] (wucun)  {符长五寸} edge [pre] (qlyb);
  \node [textnode, below=1cm of bjwj] (sicun)  {符长四寸} edge [pre] (bjwj);
  \node [textnode, below=1cm of slwt] (sancun) {符长三寸} edge [pre] (slwt);

  \startscope [on background layer]
  \node
  [fill=white!62.5!black, draw=red!62.5!black, very thick, inner sep=0.5cm, 
  rounded corners, 
  fit=(dskd) (sancun)] {};
  \stopscope
\stoptikzpicture
}
\stopplacefigure

阴符只能表述最关键的八种信号,无法表达丰富的含义和传递更具体的消息。所以,《阴书》又作了补充:

\startquotation
{
\it
武王问太公曰:“引兵深入诸侯之地,主将欲合兵,行无穷之变,图不测之利,其事烦多,符不能明;相去辽远,言语不通。为之
奈何?” 太公曰:“诸有阴事大虑,当用书,不用符。主以书遗将,将以书问主。书皆一合而再离,三发而一知。再离者,分书为
三部。三发而一知者,言三人,人操一分,相参而不相知情也。此谓阴书。敌虽圣智,莫之能识。”
}
\stopquotation

阴书作为阴符的补充,所有密谋大计,都应当用阴书,而不用阴符。国君用阴书向主将传达指示,主将用阴书向国君请示问题,这种
阴书都是一合而再离(把一封书信分为三个部分)、三发而一知(派三个人送信,每人负责其中的一部分)。阴书运用了文字拆分法
直接把一份文字拆成三份(\in{图}[fig:cipher-yinshu]),由三种渠道发送到目标方手中。敌人只有同时截获三份内容才可能
破解阴书上写的内容。

\startplacefigure
  [title={阴书所蕴含的加密原理:文字分拆法}, reference=fig:cipher-yinshu]

\midaligned{
\starttikzpicture
  [pre/.style={<-,shorten <=1pt,>=stealth',semithick},
  post/.style={->,shorten >=1pt,>=stealth',semithick},
  textnode/.style={draw=red!62.5!black, thick, fill=white!62.5!black, inner sep=2mm, font=\Tiny},
  node distance=1cm]

  \node [textnode] (mmdj) {密谋大计};
  \node [textnode, below=of mmdj]  (mmdj2) {密谋大计第2部分} edge [pre] (mmdj);
  \node [textnode, left=of mmdj2]  (mmdj1) {密谋大计第1部分} edge [pre] (mmdj);
  \node [textnode, right=of mmdj2] (mmdj3) {密谋大计第3部分} edge [pre] (mmdj);

  \startscope [on background layer]
  \node
  [fill=white!62.5!black, draw=red!62.5!black, very thick, inner sep=0.5cm, 
  rounded corners, 
  fit=(mmdj) (mmdj1) (mmdj2) (mmdj3)] {};
  \stopscope
\stoptikzpicture
}
\stopplacefigure

无论是阴符,还是阴书,都有着一定的局限性。一是有可能被对方截获而难以达到传递消息的目的,二是有可能被对方破译内容并被
对方将计就计加以利用。因此,并不是“敌虽圣智,莫之能识”。张献忠袭取襄阳就说明了这一点。

崇祯十三年七月,张献忠率领起义军突破明军防线,进入四川,杨嗣昌亦率明军十万尾随追击。面对强敌,张献忠挥师东进,于次年
二月进入湖北兴山、当阳。在东进途中,起义军活捉了由襄阳(今湖北襄樊市)回四川的杨嗣昌的军使。张献忠从其口中得知杨嗣昌
大营所在地襄阳城防空虚,决定奔袭襄阳。他杀掉使者,搜出所携带的兵符,挑选了二十八名起义军战士,换上明军的衣服,持兵符
先行。张献忠自己则亲率二千精骑,随后跟进,一昼夜急行三百里,直扑襄阳。伪装成明军的起义军士兵到达襄阳时正是夜间,他们
自称是督师杨嗣昌派来调运军械的,并出示兵符。守城明军用小筐吊上兵符,细心查验,完全吻合,才命开门放入。城门刚打开,
二十八名起义军战士一涌而入,挥刀砍杀守门明军,占领城门。张献忠率领的后续部队恰好赶到,顺利入城。一时杀声震天,明军
惊慌失措,被迫投降。起义军杀死襄王朱翊铭,降俘明军数千人,占领襄阳,杨嗣昌闻讯呕血而死。此战表明,无论是阴符还是阴书,
都不是万无一失的。

\stopsection

\stopcomponent