
\startsection[title={对称加密算法和非对称加密算法}]
\index{crypto}

在上面的例子中,如果保险箱在运输的过程中被坏人偷盗,坏人有可能使用物理破坏的方法撬开保险箱,取出里面的机密文件。那么,
显而易见,保险箱越坚固,坏人就越难破坏保险箱拿到里面的机密文件。所以,保险箱的坚固程度就决定了里面机密文件的安全性到底
有多高。

在网络通信领域,网络传输通道是极不可靠的,信息加密后的密文在传输的过程中,存在被窃听者窃取的风险,加密算法也要确保即使
密文被窃取也不能在现实的时间内被破解,这也正是加密算法的魅力所在。

从原理上,加密算法被分成两大类,即{\it 对称加密算法} (symmetric encryption algorithm) 和
{\it 非对称加密算法} (asymmetric encryption algorithm)。它们最主要的区别在于密钥 (key) 的使用方式不同。区别于
现实生活中的“钥匙”,密码算法中的密钥是一串很长的看起来非常杂乱无章的字符序列。

\startplacefigure
[title={对称加密和非对称加密},reference=fig:encrypt-decrypt]
\startcombination[1*2]
% 对称加密
{
\midaligned{
\starttikzpicture
  [pre/.style={<-,shorten <=1pt,>=stealth',semithick},
  post/.style={->,shorten >=1pt,>=stealth',semithick},
  textnode/.style={draw=red!62.5!black, thick, fill=white!62.5!black, inner sep=2mm, font=\Tiny},
  node distance=1cm]

  \node [textnode, label={\Tiny 明文}] (sender plaintext) 
    {\vbox{\hsize 10em 你好,明天下午2点奥体中心见。}};
  \node [textnode, rounded corners, right=of sender plaintext] (encrypt) {加密}
    edge [pre] (sender plaintext);
  \node [textnode, label={\Tiny 密文}, right=of encrypt] (ciphertext) {\vbox{\hsize 10em HÇÖòU¼ÔcáGATW¶\crlf n²8以®è¤ªwålâ¶}}
    edge [pre] (encrypt);
  \node [textnode, rounded corners, right=of ciphertext] (decrypt) {解密}
    edge [pre] (ciphertext);
  \node [textnode, label={\Tiny 明文}, right=of decrypt] (receiver plaintext) {\vbox{\hsize 10em 你好,明天下午2点奥体中心见。}}
    edge [pre] (decrypt);
  \node [textnode, above=of encrypt] (encrypt key) {\vbox{\hsize 16em MIIEvwIBADANBgkqhkiG9w0BAQ\crlf FAASCBKkwggSlAgEAAoIBAQDoi}}
    edge [post] (encrypt);
  \node [font=\Tiny, above=0mm of encrypt key] (encrypt key label) {加密密钥 \externalfigure[key][type=eps, height=1em]};
  \node [textnode, above=of decrypt] (decrypt key) {\vbox{\hsize 16em MIIEvwIBADANBgkqhkiG9w0BAQ\crlf FAASCBKkwggSlAgEAAoIBAQDoi}}
    edge [post] (decrypt);
  \node [font=\Tiny, above=0mm of decrypt key] (decrypt key label) {解密密钥 \externalfigure[key][type=eps, height=1em]};
\stoptikzpicture
}
}{\Tiny\darkred 对称加密算法中,加密密钥和解密密钥相同}

% 非对称加密
{
\midaligned{
\starttikzpicture
  [pre/.style={<-,shorten <=1pt,>=stealth',semithick},
  post/.style={->,shorten >=1pt,>=stealth',semithick},
  textnode/.style={draw=red!62.5!black, thick, fill=white!62.5!black, inner sep=2mm, font=\Tiny},
  node distance=1cm]

  \node [textnode, label={\Tiny 明文}] (sender plaintext) 
    {\vbox{\hsize 10em 你好,明天下午2点奥体中心见。}};
  \node [textnode, rounded corners, right=of sender plaintext] (encrypt) {加密}
    edge [pre] (sender plaintext);
  \node [textnode, label={\Tiny 密文}, right=of encrypt] (ciphertext) {\vbox{\hsize 10em ØN,L®HÇÖòU¼AL8\crlf 7ÔE­áG¶<90>zn²ä»}}
    edge [pre] (encrypt);
  \node [textnode, rounded corners, right=of ciphertext] (decrypt) {解密}
    edge [pre] (ciphertext);
  \node [textnode, label={\Tiny 明文}, right=of decrypt] (receiver plaintext) {\vbox{\hsize 10em 你好,明天下午2点奥体中心见。}}
    edge [pre] (decrypt);
  \node [textnode, above=of encrypt] (encrypt key) {\vbox{\hsize 16em 7NTBa+6BTthK30PJolCees1hvR\crlf ph2+FZrYSt1wJcMthyk5/jVWWr}}
    edge [post] (encrypt);
  \node [font=\Tiny, above=0mm of encrypt key] (encrypt key label) {加密密钥 \externalfigure[key1][type=eps, height=1em]};
  \node [textnode, above=of decrypt] (decrypt key) {\vbox{\hsize 16em SdUxi7wL1ugq1NXn9CjEyL0C1E\crlf gop5I6iAP36WNdDKNvV4iKnBqy}}
    edge [post] (decrypt);
  \node [font=\Tiny, above=0mm of decrypt key] (decrypt key label) {解密密钥 \externalfigure[key2][type=eps, height=1em]};
\stoptikzpicture
}
}{\Tiny\darkred 非对称加密算法中,加密密钥和解密密钥不同}
\stopcombination
\stopplacefigure


对称加密算法在加密和解密时使用了相同的密钥,因为加密方和解密方使用了相同的密钥,任何人拿到了密钥就能解密加密后的消息
从而获取到原始的数据。因此,对称加密算法的密钥必须在加密方和解密方两者同时妥善保管,不能泄露给任何未经授权的第三方。

相反,非对称加密算法则在加密和解密时使用了不同的密钥。而且,在非对称加密算法下,加密密钥通常被公开,但解密密钥需要由接收
者私自妥善保管。据此特点,非对称加密算法又常常被称为{\it 公钥加密算法} (public key encryption)。

非对称加密算法是在1976年,由狄菲(Whitfield Diffie)与赫尔曼(Martin Hellman)两位学者以单向函数与单向暗门函数为
基础,提出了“非对称密码体制即公开密钥密码体制”的概念,开创了密码学研究的新方向。现代计算机和互联网中的安全体系,很大程度
上都依赖于公钥加密算法。

\stopsection
