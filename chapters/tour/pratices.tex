
\startsection[title={密码与信息安全常识}]
\index{practices}

随着信息技术的飞速发展,计算机的计算能力和存储能力正在以惊人的速度不断提升,我们所熟知的摩尔定律到目前为止仍然成立。
大数据和物联网应用正在渗透到社会组织的每一个细胞,几乎对所有行业产生颠覆性和革命性的影响。产业的发展环境逐步成熟,网络
基础设施支撑能力大幅提升,网络通信的数据量正在呈现指数级爆炸式增长。在人们的生活越来越依赖互联网的时代,信息安全在网络
通信中发挥的作用尤为重要,密码技术为保障信息安全提供了全方位的技术支持。本小节从最佳实践的角度阐述我们应该怎么合理地利
用密码技术来保障信息的安全。

\startsubsection[title={任何时候不要尝试发明新的加密算法}]

刚接触密码技术的软件开发人员,经常会出现这样的想法:我自己设计一个不对外公开的密码算法不就可以保障信息的机密性了吗?
这种想法是绝对错误的。加密系统的保密性只应建立在对密钥的保密上,不应该取决于加密算法的保密,这是密码学中的金科玉律。
任何时候,我们都不要尝试自己去发明新的加密算法,因为对加密算法的保密是困难的。对手可以用窃取、购买的方法来取得算法、
加密器件或者程序。如果得到的是加密器件或者程序,可以对它们进行反向工程而最终获得加密算法。如果只是密钥失密,那么失密
的只是和此密钥有关的情报,日后通讯的保密性可以通过更换密钥来补救;但如果是加密算法失密,而整个系统的保密性又建立在
算法的秘密性上,那么所有由此算法加密的信息就会全部暴露。

\startsubsection[title={不要使用低强度的密码}]

很多人对密码的使用有这么一个误区:就算密码强度再低也比不用密码更安全吧。其实,这种想法是非常危险的。与其使用低强度的
密码,还不如从一开始就不使用密码。这主要源于用户容易通过“密码”这个词获得一种“错误的安全感”。“信息被加密了”这一事实
并不能和信息安全划上句号。攻击者使用暴力穷举(brute-force)等攻击方法就可能破解低强度的密码。

\startsubsection[title={信息安全也是一门社会性课题}]

有了密码技术,信息安全就能完全得到保证吗?答案是否定的。密码技术只是信息安全的一部分,在信息安全的背景下,社会工程学 
(social engineering)攻击是一种操纵相关人员泄露出机密信息的攻击方法,建立在使人决断产生认知偏差的基础上,有时候
这些偏差被称为“人类硬件漏洞”。犯罪分子利用社会工程学的手法进行诱骗,使受害者不会意识到被利用来攻击网络。当人们没有
意识到他们拥有的信息的价值的时候,并不会特意地保护他们所得知的信息,社会工程学正是利用了这一点。

本书不会详细讨论社会工程学攻击,但是为了让大家提高安全意识,防患于未然,特列举以下一些流行的社会工程学攻击:

\startitemize
\item 伪装:犯罪分子通过伪装成各种角色来骗取访问权限。例如,伪装成一个看门人、雇员或者客户来获取物理访问权限;冒充
贵宾、高层经理或者其他有权或进入计算机系统并察看文件的人。
\item 偷窥:通过偷窥方式在他人输入密码时收集他的密码。甚至寻找在垃圾箱中记录密码的纸、电脑打印的文件、快递信息等,
往往也可以找到有用的信息。
\item 钓鱼:钓鱼涉及虚假邮件、聊天记录或网站设计,模拟与捕捉真正目标系统的敏感数据。比如伪造一条上来自银行或其他金融
机构的需要“验证”您登陆信息的消息,来冒充一条合法的登陆页面来骗取你的登录密码。
\item 引诱:攻击者可能使用能勾起你欲望的东西引诱你去点击,可能是一场音乐会或一部电影的下载链接,也有可能是你“中奖”
需要兑换礼品的链接,或者是商品大力打折的促销链接。一旦点击了这些链接,你的计算机设备或网络就会感染恶意软件以便于犯罪
分子进入你的系统。
\stopitemize

上面提到的这些攻击手段,都与密码的强度毫无关系。信息安全是一个复杂的系统性工程,其安全程度往往取决于系统中最薄弱的环节。
通常,最薄弱的环节不是密码,而是人类自己。“道高一尺魔高一丈”,信息安全上的漏洞和人性上脆弱的环节也不断被不法分子发掘,
我们唯有不断的增强自己的安全意识和时刻保持清醒才能更好地防患于未然。
\stopsection