
\startsection[title={其他密码技术}]
\index{others}

加密算法为消息提供了机密性,但信息安全远不止于此,还有更多的问题需要解决。例如,如何保证数据的一致性,确保数据没有被恶意
篡改过;如何对信息的来源进行判断,能对伪造来源的信息进行甄别。本节,我们来初步了解一下密码学工具箱中,除加密算法以外的
的其他几种密码技术。

\startsubsection[title={单向散列函数}]

有时候,接收者希望能够验证消息在传递的过程中,没有被篡改过,即入侵者不会用假消息冒充合法消息而达到某些非法的目的。

我们经常会发现,在互联网上下载免费软件的时候,有安全意识的软件发布者会在发布软件的同时发布该软件的散列值 (hash)。散列值
就是用{\it 单向散列函数} (one-way hash function) 计算出来的。这样,下载该软件的人可以自行计算所下载文件的散列值与
发布者所发布的散列值进行比较。如果两个散列值一致,就说明下载的软件与发布者所发布的软件是相同的。软件发布者通过发布散列值
的方法,可以防止有人在软件里植入一些恶意程序来侵害下载该软件的人的计算机系统。

单向散列函数所保证的并不是机密性,而是完整性 (integrity)。散列值通常又称为哈希值、校验和 (checksum)、
指纹 (fingerprint) 或消息摘要 (message digest)。
\stopsubsection

\startsubsection[title={消息认证码}]
为了确认消息是否来源于所期望的对象,可以使用{\it 消息认证码} (message authentication code) 技术。 通过消息认证码,
不但能够确认消息是否被篡改,而且能够确认消息是否来自于所期望的通信对象。也就是说,消息认证码不仅能够保证完整性,还能够提供
认证机制。
\stopsubsection

\startsubsection[title={数字签名}]
我们先来看一个例子:供应商给采购方发来邮件,内容是“该商品的采购价格是10万元”。由于这封邮件涉及到数额巨大的交易,如果你
是采购人员,肯定会特别小心,一定要核实该邮件确实来自你联系的供应商。仅仅靠邮件发送者的Email地址是不足以判断这封邮件的
实际来源,因为邮件的发送者很容易被伪装 (spoofing)。

另一方面,还有这样一种可能,这封邮件确实是来自于采购方所联系的供应商。但是,供应商后来又反悔想提高采购价格,于是便谎称
“我当时根本就没发送过那封邮件”。像这样事后否认自己做过某件事情的行为,称为抵赖 (repudiation)。现代商战中,大量充斥着
这种案例。

当然,还有一种风险,就是供应商发给采购方的邮件在传输过程中,被别有用心的人篡改,将采购费改成了20万元。数字签名是一项能够
同时防止伪装、抵赖和篡改等威胁的密码技术。当供应商对邮件的内容加上数字签名之后再通过邮件一起发送,采购方则可以通过对
{\it 数字签名} (digital signature) 进行验证 (verify) 来检测出邮件是否被伪装和篡改,还能够防止供应商事后抵赖。
\stopsubsection


\startsubsection[title={伪随机数生成器}]
{\it 伪随机数生成器} (Pseudo Random Number Generator, PRNG) 用于在系统需要随机数的时候,通过一系列种子值计算
出来的伪随机数。因为生成一个真正意义上的“随机数”对于计算机来说是不可能的,伪随机数也只是尽可能地接近其应具有的随机性,
但是因为有“种子值”,所以伪随机数在一定程度上是可控可预测的。随机数在密码技术中承担了重要的职责,例如在访问HTTPS加密
站点时进行的TLS通信,会生成一个仅用于当前通信的临时密钥(即会话密钥),这个密钥就是基于伪随机数生成器产生的。如果生成的
随机数的算法不够好,窃听者就有可能推测出密钥,从而带来通信机密性下降的风险。
\stopsubsection

\stopsection
