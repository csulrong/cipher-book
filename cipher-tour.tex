% language=en macros=mkvi

\startcomponent cipher-tour

\environment cipher-environment

\startchapter[reference=sec:tour,title={环游密码世界}]

% \startintroduction

随着物联网和智能家居的兴起,网络信息安全已经渗透到我们日常生活的方方面面。密码技术作为信息安全的基石,为网络通信提供了安
全和可靠的技术手段。本章,我们先整体了解一下密码世界,看看各种密码技术如何为我们的信息安全保驾护航。

% \stopintroduction

\startsection[title={密码学中的基本概念}]
\index{concept}

信息在人与人、人与机器、机器与机器之间交互的过程中存在被第三方(人或计算机)窃取的风险,密码技术提供了信息在通信双方交互
的过程中免遭第三方窃取并破解,以及确保通信任何一方不被欺骗的一系列算法。

首先,我们来了解一下与密码技术有关的角色:

\startitemize
\item {\it 发送者} (sender):消息的发送方。
\item {\it 接收者} (receiver):消息的接收方。
\item {\it 窃听者} (eavesdropper):监听在消息传送的通道上,窃取消息的恶意攻击方。
\item {\it 破译者} (cryptanalyst):为研究密码强度而工作的密码破译人员或密码学研究者,要注意和窃听者的本质区别。
\stopitemize


\startplacefigure
  [title={消息的发送、接收和窃听},reference=fig:cipher-roles]
\midaligned{
\starttikzpicture
  [pre/.style={<-,shorten <=1pt,>=stealth',semithick},
  post/.style={->,shorten >=1pt,>=stealth',semithick}]

  % sender
  \node (sender picture) {\externalfigure[sender][type=eps,height=1cm]};
  \node [font=\Tiny, below=0.1mm of sender picture] (sender picture label) {发送者};
  \node [right=of sender picture] (sender message picture) 
    {\externalfigure[text][type=eps,height=1cm]}
    edge [pre] (sender picture);
  \node [font=\Tiny, below=0.1mm of sender message picture] (sender message label) {消息};

  % channel
  \node [shape=cylinder, aspect=0.25, draw=red!62.5!black, thick, 
    cylinder uses custom fill,
    cylinder body fill=white!62.5!black, 
    cylinder end fill=white!50!black,
    inner xsep=8mm, inner ysep=2mm, 
    right=1.5cm of sender message picture] (channel) {传输通道}
    edge [pre] (sender message picture);

  % receiver
  \node [right=1.5cm of channel] (receiver message picture) 
    {\externalfigure[text][type=eps,height=1cm]}
    edge [pre] (channel);
  \node [font=\Tiny,below=0.1mm of receiver message picture] (receiver message label) {消息};
  \node [right=of receiver message picture] (receiver picture) 
    {\externalfigure[receiver][type=eps,height=1cm]}
    edge [pre] (receiver message picture);
  \node [font=\Tiny,below=0.1mm of receiver picture] (receiver picture label) {接收者};
  
  % eavasdropper
  \node [below=2cm of channel] (intercept message picture)
    {\externalfigure[text][type=eps,height=1cm]}
    edge [pre, dashed] (channel);
  \node [font=\Tiny,below=0.1mm of intercept message picture] (intercept message label) {消息};
  \node [right=of intercept message picture] (eavasdropper picture)
    {\externalfigure[eavasdropper][type=eps,height=1cm]}
    edge [pre] (intercept message picture);
  \node [font=\Tiny,below=0.1mm of eavasdropper picture] (eavasdropper picture label) {窃听者};

  \startscope [on background layer]
    \node
    [fill=white!62.5!black, draw=red!62.5!black, very thick, inner xsep=0.6cm, 
     rounded corners, 
     fit=(sender picture) (sender picture label) (sender message picture) (sender message label)] {};
  \stopscope

  \startscope [on background layer]
    \node
    [fill=white!62.5!black, draw=red!62.5!black, very thick, inner xsep=0.6cm, 
    rounded corners, 
    fit=(receiver picture) (receiver picture label) (receiver message picture) (receiver message label)] {};
  \stopscope

  \startscope [on background layer]
    \node
    [fill=white!62.5!black, draw=red!62.5!black, very thick, inner xsep=0.6cm, 
    rounded corners, 
    fit=(eavasdropper picture) (eavasdropper picture label) (intercept message picture) (intercept message label)] {};
  \stopscope
\stoptikzpicture
}
\stopplacefigure

如\in{图}[fig:cipher-roles]所示,如果发送者不对要发送出去的消息进行任何处理,很容易被窃听者窃取并获知消息的内容。
为确保消息的机密性 (confidentiality),发送者在发送消息之前需要对其进行加密 (encryption)。消息可以是任何类型的数据,
例如,邮件、文档和交易等。通常,我们把加密前的消息称之为{\it 明文} (plaintext),加密之后的消息称之为{\it 密文}
 (ciphertext)。接收者收到密文后将其恢复回明文的过程称为解密 (decryption)。
 
因为加密和解密需要相应的密钥才能完成,当窃听者窃取到加密后的密文之后,因为没有密钥,也就无法还原出原始的明文。这就相当
于,我们在把重要的机密文件传给接收人之前,先把机密文件锁在保险柜里面,然后把保险柜交给物流公司帮忙交付给接收人,接收人
收到之后用保险柜的钥匙打开取出该机密文件。在物流运输的过程中,保险柜有可能面临被丢失的风险,如果有人偷盗了保险柜,因为
没有钥匙也无法取出里面的文件。整个过程可以用\in{图}[fig:encrypt-decrypt]来描述。

\startplacefigure
[title={消息加密和解密的过程},reference=fig:encrypt-decrypt]
\midaligned{
\starttikzpicture
  [pre/.style={<-,shorten <=1pt,>=stealth',semithick},
  post/.style={->,shorten >=1pt,>=stealth',semithick},
  node distance=0.5cm]

  % sender
  \node (sender picture) {\externalfigure[sender][type=eps,height=1cm]};
  \node [font=\Tiny, below=0.1mm of sender picture] (sender picture label) {发送者};
  \node [right=of sender picture] (sender message picture) 
    {\externalfigure[text][type=eps,height=1cm]}
    edge [pre] (sender picture);
  \node [font=\Tiny, below=0.1mm of sender message picture] (sender message label) {明文};
  \node [right=of sender message picture] (encrypt) 
    {\externalfigure[lock][type=eps,height=1cm]}
    edge [pre] (sender message picture);
  \node [font=\Tiny, below=0.1mm of encrypt] (encrypt label) {加密};

  % channel
  \node [shape=cylinder, aspect=0.25, draw=red!62.5!black, thick, 
    cylinder uses custom fill,
    cylinder body fill=white!62.5!black, 
    cylinder end fill=white!50!black,
    inner xsep=5mm, inner ysep=2mm, 
    label=above:{\Tiny 传输通道},
    right=1.5cm of encrypt] (channel) {
      \externalfigure[file-locked][type=eps,height=0.8cm]
    }
    edge [pre] (encrypt);

  % receiver
  \node [right=1.5cm of channel] (decrypt) {\externalfigure[unlock][type=eps,height=1cm]}
    edge [pre] (channel);
  \node [font=\Tiny,below=0.1mm of decrypt] (decrypt label) {解密};
  \node [right=of decrypt] (receiver message picture) 
    {\externalfigure[text][type=eps,height=1cm]}
    edge [pre] (decrypt);
  \node [font=\Tiny,below=0.1mm of receiver message picture] (receiver message label) {明文};
  \node [right=of receiver message picture] (receiver picture) 
    {\externalfigure[receiver][type=eps,height=1cm]}
    edge [pre] (receiver message picture);
  \node [font=\Tiny,below=0.1mm of receiver picture] (receiver picture label) {接收者};

  % eavasdropper
  \node [below=2cm of channel] (intercept message picture)
    {\externalfigure[file-locked][type=eps,height=1cm]}
    edge [pre, dashed] (channel);
  \node [font=\Tiny,below=0.1mm of intercept message picture] (intercept message label) {密文};
  \node [right=of intercept message picture] (eavasdropper picture)
    {\externalfigure[eavasdropper][type=eps,height=1cm]}
    edge [pre] (intercept message picture);
  \node [font=\Tiny,below=0.1mm of eavasdropper picture] (eavasdropper picture label) {窃听者};
  \node [ellipse callout, right=1mm of eavasdropper picture.north east, draw=red!62.5!black, thick, font=\Tiny,
    callout absolute pointer={(eavasdropper picture.east)}] (question) {到底是啥呢?};

  \startscope [on background layer]
    \node
    [fill=white!62.5!black, draw=red!62.5!black, very thick, inner xsep=0.2cm, 
    rounded corners, 
    fit=(sender picture) (sender picture label) (sender message picture) (sender message label) (encrypt) (encrypt label)] {};
  \stopscope

  \startscope [on background layer]
    \node
    [fill=white!62.5!black, draw=red!62.5!black, very thick, inner xsep=0.2cm, 
    rounded corners, 
    fit=(receiver picture) (receiver picture label) (receiver message picture) (receiver message label) (decrypt) (decrypt label)] {};
  \stopscope

  \startscope [on background layer]
    \node
    [fill=white!62.5!black, draw=red!62.5!black, very thick, inner xsep=0.2cm, 
    rounded corners, 
    fit=(eavasdropper picture) (eavasdropper picture label) (intercept message picture) (intercept message label) (question)] {};
  \stopscope
\stoptikzpicture
}
\stopplacefigure

\stopsection

%%%%%%%%%%%%%%%%%%%%%%%%%%%%%%

\startsection[title={对称加密算法和非对称加密算法}]
\index{crypto}

在上面的例子中,如果保险箱在运输的过程中被坏人偷盗,坏人有可能使用物理破坏的方法撬开保险箱,取出里面的机密文件。那么,
显而易见,保险箱越坚固,坏人就越难破坏保险箱拿到里面的机密文件。所以,保险箱的坚固程度就决定了里面机密文件的安全性到底
有多高。

在网络通信领域,网络传输通道是极不可靠的,信息加密后的密文在传输的过程中,存在被窃听者窃取的风险,加密算法也要确保即使
密文被窃取也不能在现实的时间内被破解,这也正是加密算法的魅力所在。

从原理上,加密算法被分成两大类,即{\it 对称加密算法} (symmetric encryption algorithm) 和
{\it 非对称加密算法} (asymmetric encryption algorithm)。它们最主要的区别在于密钥 (key) 的使用方式不同。区别于
现实生活中的“钥匙”,密码算法中的密钥是一串很长的看起来非常杂乱无章的字符序列。

\startplacefigure
[title={对称加密和非对称加密},reference=fig:encrypt-decrypt]
\startcombination[1*2]
% 对称加密
{
\midaligned{
\starttikzpicture
  [pre/.style={<-,shorten <=1pt,>=stealth',semithick},
  post/.style={->,shorten >=1pt,>=stealth',semithick},
  textnode/.style={draw=red!62.5!black, thick, fill=white!62.5!black, inner sep=2mm, font=\Tiny},
  node distance=1cm]

  \node [textnode, label={\Tiny 明文}] (sender plaintext) 
    {\vbox{\hsize 10em 你好,明天下午2点奥体中心见。}};
  \node [textnode, rounded corners, right=of sender plaintext] (encrypt) {加密}
    edge [pre] (sender plaintext);
  \node [textnode, label={\Tiny 密文}, right=of encrypt] (ciphertext) {\vbox{\hsize 10em HÇÖòU¼ÔcáGATW¶\crlf n²8以®è¤ªwålâ¶}}
    edge [pre] (encrypt);
  \node [textnode, rounded corners, right=of ciphertext] (decrypt) {解密}
    edge [pre] (ciphertext);
  \node [textnode, label={\Tiny 明文}, right=of decrypt] (receiver plaintext) {\vbox{\hsize 10em 你好,明天下午2点奥体中心见。}}
    edge [pre] (decrypt);
  \node [textnode, above=of encrypt] (encrypt key) {\vbox{\hsize 16em MIIEvwIBADANBgkqhkiG9w0BAQ\crlf FAASCBKkwggSlAgEAAoIBAQDoi}}
    edge [post] (encrypt);
  \node [font=\Tiny, above=0mm of encrypt key] (encrypt key label) {加密密钥 \externalfigure[key][type=eps, height=1em]};
  \node [textnode, above=of decrypt] (decrypt key) {\vbox{\hsize 16em MIIEvwIBADANBgkqhkiG9w0BAQ\crlf FAASCBKkwggSlAgEAAoIBAQDoi}}
    edge [post] (decrypt);
  \node [font=\Tiny, above=0mm of decrypt key] (decrypt key label) {解密密钥 \externalfigure[key][type=eps, height=1em]};
\stoptikzpicture
}
}{\Tiny\darkred 对称加密算法中,加密密钥和解密密钥相同}

% 非对称加密
{
\midaligned{
\starttikzpicture
  [pre/.style={<-,shorten <=1pt,>=stealth',semithick},
  post/.style={->,shorten >=1pt,>=stealth',semithick},
  textnode/.style={draw=red!62.5!black, thick, fill=white!62.5!black, inner sep=2mm, font=\Tiny},
  node distance=1cm]

  \node [textnode, label={\Tiny 明文}] (sender plaintext) 
    {\vbox{\hsize 10em 你好,明天下午2点奥体中心见。}};
  \node [textnode, rounded corners, right=of sender plaintext] (encrypt) {加密}
    edge [pre] (sender plaintext);
  \node [textnode, label={\Tiny 密文}, right=of encrypt] (ciphertext) {\vbox{\hsize 10em ØN,L®HÇÖòU¼AL8\crlf 7ÔE­áG¶<90>zn²ä»}}
    edge [pre] (encrypt);
  \node [textnode, rounded corners, right=of ciphertext] (decrypt) {解密}
    edge [pre] (ciphertext);
  \node [textnode, label={\Tiny 明文}, right=of decrypt] (receiver plaintext) {\vbox{\hsize 10em 你好,明天下午2点奥体中心见。}}
    edge [pre] (decrypt);
  \node [textnode, above=of encrypt] (encrypt key) {\vbox{\hsize 16em 7NTBa+6BTthK30PJolCees1hvR\crlf ph2+FZrYSt1wJcMthyk5/jVWWr}}
    edge [post] (encrypt);
  \node [font=\Tiny, above=0mm of encrypt key] (encrypt key label) {加密密钥 \externalfigure[key1][type=eps, height=1em]};
  \node [textnode, above=of decrypt] (decrypt key) {\vbox{\hsize 16em SdUxi7wL1ugq1NXn9CjEyL0C1E\crlf gop5I6iAP36WNdDKNvV4iKnBqy}}
    edge [post] (decrypt);
  \node [font=\Tiny, above=0mm of decrypt key] (decrypt key label) {解密密钥 \externalfigure[key2][type=eps, height=1em]};
\stoptikzpicture
}
}{\Tiny\darkred 非对称加密算法中,加密密钥和解密密钥不同}
\stopcombination
\stopplacefigure


对称加密算法在加密和解密时使用了相同的密钥,因为加密方和解密方使用了相同的密钥,任何人拿到了密钥就能解密加密后的消息
从而获取到原始的数据。因此,对称加密算法的密钥必须在加密方和解密方两者同时妥善保管,不能泄露给任何未经授权的第三方。

相反,非对称加密算法则在加密和解密时使用了不同的密钥。而且,在非对称加密算法下,加密密钥通常被公开,但解密密钥需要由接收
者私自妥善保管。据此特点,非对称加密算法又常常被称为{\it 公钥加密算法} (public key encryption)。

非对称加密算法是在1976年,由狄菲(Whitfield Diffie)与赫尔曼(Martin Hellman)两位学者以单向函数与单向暗门函数为
基础,提出了“非对称密码体制即公开密钥密码体制”的概念,开创了密码学研究的新方向。现代计算机和互联网中的安全体系,很大程度
上都依赖于公钥加密算法。

\stopsection

%%%%%%%%%%%%%%%%%%%%%%%%%%%%%%

\startsection[title={其他密码技术}]
\index{others}

加密算法为消息提供了机密性,但信息安全远不止于此,还有更多的问题需要解决。例如,如何保证数据的一致性,确保数据没有被恶意
篡改过;如何对信息的来源进行判断,能对伪造来源的信息进行甄别。本节,我们来初步了解一下密码学工具箱中,除加密算法以外的
的其他几种密码技术。

\startsubsection[title={单向散列函数}]

有时候,接收者希望能够验证消息在传递的过程中,没有被篡改过,即入侵者不会用假消息冒充合法消息而达到某些非法的目的。

我们经常会发现,在互联网上下载免费软件的时候,有安全意识的软件发布者会在发布软件的同时发布该软件的散列值 (hash)。散列值
就是用{\it 单向散列函数} (one-way hash function) 计算出来的。这样,下载该软件的人可以自行计算所下载文件的散列值与
发布者所发布的散列值进行比较。如果两个散列值一致,就说明下载的软件与发布者所发布的软件是相同的。软件发布者通过发布散列值
的方法,可以防止有人在软件里植入一些恶意程序来侵害下载该软件的人的计算机系统。

单向散列函数所保证的并不是机密性,而是完整性 (integrity)。散列值通常又称为哈希值、校验和 (checksum)、
指纹 (fingerprint) 或消息摘要 (message digest)。
\stopsubsection

\startsubsection[title={消息认证码}]
为了确认消息是否来源于所期望的对象,可以使用{\it 消息认证码} (message authentication code) 技术。 通过消息认证码,
不但能够确认消息是否被篡改,而且能够确认消息是否来自于所期望的通信对象。也就是说,消息认证码不仅能够保证完整性,还能够提供
认证机制。
\stopsubsection

\startsubsection[title={数字签名}]
我们先来看一个例子:供应商给采购方发来邮件,内容是“该商品的采购价格是10万元”。由于这封邮件涉及到数额巨大的交易,如果你
是采购人员,肯定会特别小心,一定要核实该邮件确实来自你联系的供应商。仅仅靠邮件发送者的Email地址是不足以判断这封邮件的
实际来源,因为邮件的发送者很容易被伪装 (spoofing)。

另一方面,还有这样一种可能,这封邮件确实是来自于采购方所联系的供应商。但是,供应商后来又反悔想提高采购价格,于是便谎称
“我当时根本就没发送过那封邮件”。像这样事后否认自己做过某件事情的行为,称为抵赖 (repudiation)。现代商战中,大量充斥着
这种案例。

当然,还有一种风险,就是供应商发给采购方的邮件在传输过程中,被别有用心的人篡改,将采购费改成了20万元。数字签名是一项能够
同时防止伪装、抵赖和篡改等威胁的密码技术。当供应商对邮件的内容加上数字签名之后再通过邮件一起发送,采购方则可以通过对
{\it 数字签名} (digital signature) 进行验证 (verify) 来检测出邮件是否被伪装和篡改,还能够防止供应商事后抵赖。
\stopsubsection


\startsubsection[title={伪随机数生成器}]
{\it 伪随机数生成器} (Pseudo Random Number Generator, PRNG) 用于在系统需要随机数的时候,通过一系列种子值计算
出来的伪随机数。因为生成一个真正意义上的“随机数”对于计算机来说是不可能的,伪随机数也只是尽可能地接近其应具有的随机性,
但是因为有“种子值”,所以伪随机数在一定程度上是可控可预测的。随机数在密码技术中承担了重要的职责,例如在访问HTTPS加密
站点时进行的TLS通信,会生成一个仅用于当前通信的临时密钥(即会话密钥),这个密钥就是基于伪随机数生成器产生的。如果生成的
随机数的算法不够好,窃听者就有可能推测出密钥,从而带来通信机密性下降的风险。
\stopsubsection

\stopsection

%%%%%%%%%%%%%%%%%%%%%%%%%%%%%%

\startsection[title={信息安全所面临的威胁及对策}]
\index{threats}

回顾一下,我们前面初步介绍了六种密码技术:

\startitemize
\item 对称加密算法
\item 非对称加密算法(公钥加密算法)
\item 单向散列函数
\item 消息认证码
\item 数字签名
\item 伪随机数生成器
\stopitemize

我们同时讨论了每种技术所解决的具体问题,这里把前面的内容再梳理一遍,用\in{图}[fig:cipher-mindmap]所示的思维导视图
总结了信息安全所面临的潜在威胁以及针对各种安全威胁所能采用的密码技术及对策。我们没有把伪随机数生成器画在图里面,是因为
它通常渗透在其他五种密码技术中使用,发挥了非常重要的作用。我们把这六种密码技术统称为{\it 密码学家的工具箱}。



\startplacefigure
  [title={信息安全所面临的威胁及其相应的密码技术对策思维导视图},
  reference=fig:cipher-mindmap]
\midaligned{
  \starttikzpicture
  \startscope[
  small mindmap,
  every node/.style={concept, circular drop shadow, execute at begin node=\hskip0pt},
  root concept/.append style={
    concept color=black,
    fill=white, line width=0.6ex,
    text=black},
  text=white,
  eavesdrop/.style={concept color=red!62.5!black,faded/.style={concept color=red!80!black}},
  tamper/.style={concept color=blue!62.5!black,faded/.style={concept color=blue!80!black}},
  spoofing/.style={concept color=orange!62.5!black,faded/.style={concept color=orange!80!black}},
  repudiation/.style={concept color=green!62.5!black,faded/.style={concept color=green!80!black}},
  grow cyclic,
  level 1/.append style={level distance=2.8cm,sibling angle=60},
  level 2/.append style={level distance=2.2cm,sibling angle=45}]
    \node [root concept] (Threats) {信息安全所面临的威胁} % root
      [clockwise from=180]
      child [eavesdrop] { node (Eavesdrop) {窃听}
        [clockwise from=180]
        child [faded] { node (symmetric encryption) {对称加密算法} }
        child [faded] { node (asymmetric encryption) {非对称加密算法} }
      }
      child [tamper] { node (Tamper) {篡改}
        [clockwise from=135]
        child [faded] { node (One-Way Hash Function) {单向散列函数} }
        child [faded] { node (Message Authentication Code) {消息认证码} }
        child [faded] { node (Digital Signature) {数字签名} }
      }
      child [repudiation] { node (Repudication) {抵赖}
        [clockwise from=45]
        child [faded] { node (Digital Signature) {数字签名} }
      }
      child [spoofing] { node (Spoofing) {伪装}
        [clockwise from=45]
        child [faded] { node (Message Authentication Code) {消息认证码} }
        child [faded] { node (Digital Signature) {数字签名} }
      };
  \stopscope
  \stoptikzpicture
}
\stopplacefigure

从\in{图}[fig:cipher-mindmap]中,我们可以看到,有些密码技术可以用来解决信息安全中的多种威胁,例如,数字签名可以
防止篡改、伪装和抵赖,但不提供保密。对于某些面临的威胁,也可能存在多种应对的密码技术,例如为了防止窃听导致信息被泄露,
可以使用对称加密算法或非对称加密算法。但是每种密码技术都有着各自的特点,适用于不同的场景。后面章节,我们会对这些密码
技术进行深入的探讨,逐个揭开它们的神秘面纱。

% \placefigure
%   [][fig:cipher-mindmap]
%   {信息安全威胁及密码技术思维导视图}
%   {\externalfigure[hacker.png]}

\stopsection


%%%%%%%%%%%%%%%%%%%%%%%%%%%%%%

\startsection[title={密码与信息安全常识}]
\index{practices}

随着信息技术的飞速发展,计算机的计算能力和存储能力正在以惊人的速度不断提升,我们所熟知的摩尔定律到目前为止仍然成立。
大数据和物联网应用正在渗透到社会组织的每一个细胞,几乎对所有行业产生颠覆性和革命性的影响。产业的发展环境逐步成熟,网络
基础设施支撑能力大幅提升,网络通信的数据量正在呈现指数级爆炸式增长。在人们的生活越来越依赖互联网的时代,信息安全在网络
通信中发挥的作用尤为重要,密码技术为保障信息安全提供了全方位的技术支持。本小节从最佳实践的角度阐述我们应该怎么合理地利
用密码技术来保障信息的安全。

\startsubsection[title={任何时候不要尝试发明新的加密算法}]

刚接触密码技术的软件开发人员,经常会出现这样的想法:我自己设计一个不对外公开的密码算法不就可以保障信息的机密性了吗?
这种想法是绝对错误的。加密系统的保密性只应建立在对密钥的保密上,不应该取决于加密算法的保密,这是密码学中的金科玉律。
任何时候,我们都不要尝试自己去发明新的加密算法,因为对加密算法的保密是困难的。对手可以用窃取、购买的方法来取得算法、
加密器件或者程序。如果得到的是加密器件或者程序,可以对它们进行反向工程而最终获得加密算法。如果只是密钥失密,那么失密
的只是和此密钥有关的情报,日后通讯的保密性可以通过更换密钥来补救;但如果是加密算法失密,而整个系统的保密性又建立在
算法的秘密性上,那么所有由此算法加密的信息就会全部暴露。

\startsubsection[title={不要使用低强度的密码}]

很多人对密码的使用有这么一个误区:就算密码强度再低也比不用密码更安全吧。其实,这种想法是非常危险的。与其使用低强度的
密码,还不如从一开始就不使用密码。这主要源于用户容易通过“密码”这个词获得一种“错误的安全感”。“信息被加密了”这一事实
并不能和信息安全划上句号。攻击者使用暴力穷举(brute-force)等攻击方法就可能破解低强度的密码。

\startsubsection[title={信息安全也是一门社会性课题}]

有了密码技术,信息安全就能完全得到保证吗?答案是否定的。密码技术只是信息安全的一部分,在信息安全的背景下,社会工程学 
(social engineering)攻击是一种操纵相关人员泄露出机密信息的攻击方法,建立在使人决断产生认知偏差的基础上,有时候
这些偏差被称为“人类硬件漏洞”。犯罪分子利用社会工程学的手法进行诱骗,使受害者不会意识到被利用来攻击网络。当人们没有
意识到他们拥有的信息的价值的时候,并不会特意地保护他们所得知的信息,社会工程学正是利用了这一点。

本书不会详细讨论社会工程学攻击,但是为了让大家提高安全意识,防患于未然,特列举以下一些流行的社会工程学攻击:

\startitemize
\item 伪装:犯罪分子通过伪装成各种角色来骗取访问权限。例如,伪装成一个看门人、雇员或者客户来获取物理访问权限;冒充
贵宾、高层经理或者其他有权或进入计算机系统并察看文件的人。
\item 偷窥:通过偷窥方式在他人输入密码时收集他的密码。甚至寻找在垃圾箱中记录密码的纸、电脑打印的文件、快递信息等,
往往也可以找到有用的信息。
\item 钓鱼:钓鱼涉及虚假邮件、聊天记录或网站设计,模拟与捕捉真正目标系统的敏感数据。比如伪造一条上来自银行或其他金融
机构的需要“验证”您登陆信息的消息,来冒充一条合法的登陆页面来骗取你的登录密码。
\item 引诱:攻击者可能使用能勾起你欲望的东西引诱你去点击,可能是一场音乐会或一部电影的下载链接,也有可能是你“中奖”
需要兑换礼品的链接,或者是商品大力打折的促销链接。一旦点击了这些链接,你的计算机设备或网络就会感染恶意软件以便于犯罪
分子进入你的系统。
\stopitemize

上面提到的这些攻击手段,都与密码的强度毫无关系。信息安全是一个复杂的系统性工程,其安全程度往往取决于系统中最薄弱的环节。
通常,最薄弱的环节不是密码,而是人类自己。“道高一尺魔高一丈”,信息安全上的漏洞和人性上脆弱的环节也不断被不法分子发掘,
我们唯有不断的增强自己的安全意识和时刻保持清醒才能更好地防患于未然。
\stopsection

%%%%%%%%%%%%%%%%%%%%%%%%%%%%%%

\startsection[title={本章小结}]
\index{summary}

本章,我们初步了解了密码世界里常用的密码技术,并介绍了使用哪种密码技术来应对信息安全中存在的威胁。我们后在后续章节中
更详细地介绍每种密码技术的细节,并为应用开发者介绍怎么在工程中使用各种密码技术。

\stopsection

\stopchapter

\stopcomponent
