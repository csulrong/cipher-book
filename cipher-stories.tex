% language=en macros=mkvi

\startcomponent cipher-tour

\environment cipher-environment

\startchapter[reference=sec:stories,title={密码的历史典故}]

% \startintro

从密码学发展历程来看,可分为古典密码和现代密码两类。古典密码有着悠久的历史,是以字符为基本加密单元的密码。而现代密码
则以信息块为基本的加密单元。古典密码和现代密码的分水岭大致就是在计算机问世的时候。本章将重点回顾古典密码的发展历史并
分享一些和密码相关的有趣典故和历史事件。

% 古典密码学主要有两大基本方法:
% 置换密码(又称易位密码):明文的字母保持相同,但顺序被打乱了。
% 代替密码:就是将明文的字符替换为密文中的另一种的字符,接收者只要对密文做反向替换就可以恢复出明文。

% \stopintro


%%%%%%%%%%%%%%%%%%%%%%%%%%%%%%


\startsection[title={中国古代人民怎么加密?}]

早期加密算法主要使用在军事中,中国历史上最早关于加密算法的记载出自于周朝兵书《六韬·龙韬》中的《阴符》和《阴书》。
其中《阴符》记载了:

\startquotation
{\it
太公曰:“主与将,有阴符,凡八等。有大胜克敌之符,长一尺。破军擒将之符,长九寸。降城得邑之符,长八寸。却敌报远之符,
长七寸。警众坚守之符,长六寸。请粮益兵之符,长五寸。败军亡将之符,长四寸。失利亡士之符,长三寸。诸奉使行符,稽留,
若符事闻,泄告者,皆诛之。八符者,主将秘闻,所以阴通言语,不泄中外相知之术。敌虽圣智,莫之能识。”
}
\stopquotation

简单来说,阴符是以八等长度的符来表达不同的消息和指令,属于密码学中的替代法(\in{图}[fig:cipher-yinfu]),在应用
中是把信息转变成敌人看不懂的符号,但知情者知道这些符号代表的含义。

\startplacefigure
  [title={阴符所蕴含的加密原理:替换法}, reference=fig:cipher-yinfu]

\midaligned{
\starttikzpicture
  [pre/.style={<-,shorten <=1pt,>=stealth',semithick},
  post/.style={->,shorten >=1pt,>=stealth',semithick},
  textnode/.style={draw=red!62.5!black, thick, fill=white!62.5!black, inner sep=2mm, font=\Tiny},
  node distance=3mm]

  \node [textnode] (dskd) {大胜克敌};
  \node [textnode, right=of dskd] (pjqj) {破军擒将};
  \node [textnode, right=of pjqj] (xcdy) {降城得邑};
  \node [textnode, right=of xcdy] (qdby) {却敌报远};
  \node [textnode, right=of qdby] (jzjs) {警众坚守};
  \node [textnode, right=of jzjs] (qlyb) {请粮益兵};
  \node [textnode, right=of qlyb] (bjwj) {败军亡将};
  \node [textnode, right=of bjwj] (slwt) {失利亡士}; 

  \node [textnode, below=1cm of dskd] (yichi)  {符长一尺} edge [pre] (dskd);
  \node [textnode, below=1cm of pjqj] (jiucun) {符长九寸} edge [pre] (pjqj);
  \node [textnode, below=1cm of xcdy] (bacun)  {符长八寸} edge [pre] (xcdy);
  \node [textnode, below=1cm of qdby] (qicun)  {符长七寸} edge [pre] (qdby);
  \node [textnode, below=1cm of jzjs] (liucun) {符长六寸} edge [pre] (jzjs);
  \node [textnode, below=1cm of qlyb] (wucun)  {符长五寸} edge [pre] (qlyb);
  \node [textnode, below=1cm of bjwj] (sicun)  {符长四寸} edge [pre] (bjwj);
  \node [textnode, below=1cm of slwt] (sancun) {符长三寸} edge [pre] (slwt);

  \startscope [on background layer]
  \node
  [fill=white!62.5!black, draw=red!62.5!black, very thick, inner sep=0.5cm, 
  rounded corners, 
  fit=(dskd) (sancun)] {};
  \stopscope
\stoptikzpicture
}
\stopplacefigure

阴符只能表述最关键的八种信号,无法表达丰富的含义和传递更具体的消息。所以,《阴书》又作了补充:

\startquotation
{
\it
武王问太公曰:“引兵深入诸侯之地,主将欲合兵,行无穷之变,图不测之利,其事烦多,符不能明;相去辽远,言语不通。为之
奈何?” 太公曰:“诸有阴事大虑,当用书,不用符。主以书遗将,将以书问主。书皆一合而再离,三发而一知。再离者,分书为
三部。三发而一知者,言三人,人操一分,相参而不相知情也。此谓阴书。敌虽圣智,莫之能识。”
}
\stopquotation

阴书作为阴符的补充,所有密谋大计,都应当用阴书,而不用阴符。国君用阴书向主将传达指示,主将用阴书向国君请示问题,这种
阴书都是一合而再离(把一封书信分为三个部分)、三发而一知(派三个人送信,每人负责其中的一部分)。阴书运用了文字拆分法
直接把一份文字拆成三份(\in{图}[fig:cipher-yinshu]),由三种渠道发送到目标方手中。敌人只有同时截获三份内容才可能
破解阴书上写的内容。

\startplacefigure
  [title={阴书所蕴含的加密原理:文字分拆法}, reference=fig:cipher-yinshu]

\midaligned{
\starttikzpicture
  [pre/.style={<-,shorten <=1pt,>=stealth',semithick},
  post/.style={->,shorten >=1pt,>=stealth',semithick},
  textnode/.style={draw=red!62.5!black, thick, fill=white!62.5!black, inner sep=2mm, font=\Tiny},
  node distance=1cm]

  \node [textnode] (mmdj) {密谋大计};
  \node [textnode, below=of mmdj]  (mmdj2) {密谋大计第2部分} edge [pre] (mmdj);
  \node [textnode, left=of mmdj2]  (mmdj1) {密谋大计第1部分} edge [pre] (mmdj);
  \node [textnode, right=of mmdj2] (mmdj3) {密谋大计第3部分} edge [pre] (mmdj);

  \startscope [on background layer]
  \node
  [fill=white!62.5!black, draw=red!62.5!black, very thick, inner sep=0.5cm, 
  rounded corners, 
  fit=(mmdj) (mmdj1) (mmdj2) (mmdj3)] {};
  \stopscope
\stoptikzpicture
}
\stopplacefigure

无论是阴符,还是阴书,都有着一定的局限性。一是有可能被对方截获而难以达到传递消息的目的,二是有可能被对方破译内容并被
对方将计就计加以利用。因此,并不是“敌虽圣智,莫之能识”。张献忠袭取襄阳就说明了这一点。

崇祯十三年七月,张献忠率领起义军突破明军防线,进入四川,杨嗣昌亦率明军十万尾随追击。面对强敌,张献忠挥师东进,于次年
二月进入湖北兴山、当阳。在东进途中,起义军活捉了由襄阳(今湖北襄樊市)回四川的杨嗣昌的军使。张献忠从其口中得知杨嗣昌
大营所在地襄阳城防空虚,决定奔袭襄阳。他杀掉使者,搜出所携带的兵符,挑选了二十八名起义军战士,换上明军的衣服,持兵符
先行。张献忠自己则亲率二千精骑,随后跟进,一昼夜急行三百里,直扑襄阳。伪装成明军的起义军士兵到达襄阳时正是夜间,他们
自称是督师杨嗣昌派来调运军械的,并出示兵符。守城明军用小筐吊上兵符,细心查验,完全吻合,才命开门放入。城门刚打开,
二十八名起义军战士一涌而入,挥刀砍杀守门明军,占领城门。张献忠率领的后续部队恰好赶到,顺利入城。一时杀声震天,明军
惊慌失措,被迫投降。起义军杀死襄王朱翊铭,降俘明军数千人,占领襄阳,杨嗣昌闻讯呕血而死。此战表明,无论是阴符还是阴书,
都不是万无一失的。

\startsection[title={凯撒密码}]
\index{caesar}

我们把视角切换到世界历史的长河中,看看古罗马时期凯撒大帝是怎么使用加密算法对军事信息进行加密的。根据罗马早期纪传体作者
盖乌斯·苏维托尼乌斯的记载,恺撒大帝的加密策略很简单,就是把字母按照字母表顺序向后移动几位,但是偏移量(offset)只有他
和将军知道,如果移动后超过了字母表中的最后一个字母(对于英文字母表而言就是\type{Z}),就回到字母表的第一个字母重新开始
下一轮。以英文字母表为例,在偏移量为3的情况下,\type{A}将会替换为\type{D},\type{B}将会被替换为\type{E},
\type{W}会被替换为\type{Z},\type{X}会被替换为\type{A};明文\type{HELLO}会被转换为密文\type{KHOOR}。这种加密
方法又被称为移位加密。


\startplacefigure
  [title={凯撒密码的加密原理:替换法}, reference=fig:caesar]

\midaligned{
\starttikzpicture
  [pre/.style={<-,shorten <=1pt,>=stealth',semithick},
  post/.style={->,shorten >=1pt,>=stealth',semithick},
  every node/.style={draw=red!62.5!black, thick, fill=white!62.5!black, minimum height=1.2em, minimum width=1.2em, font=\Tiny}]
  
  \foreach \a/\n in {A/0, B/1, C/2, D/3, E/4, F/5, G/6, H/7, I/8, 
    J/9, K/10, L/11, M/12, N/13, O/14, P/15, Q/16, R/17, S/18,
    T/19, U/20, V/21, W/22, X/23, Y/24, Z/25, A/26, B/27, C/28} {
    \ifnum \n < 26
    \node (\a-1) at (\n * 5.2mm, 2) {\a};
    \fi
    \ifnum \n > 2
    \node (\a-2) at (\n * 5.2mm, 0) {\a};
    \else
    \node [dashed] (\a-2) at (\n * 5.2mm, 0) {\a};
    \fi
  }

  \foreach \a/\b in {A/D, B/E, C/F, D/G, E/H, F/I, G/J, H/K, I/L, 
    J/M, K/N, L/O, M/P, N/Q, O/R, P/S, Q/T, R/U, S/V,
    T/W, U/X, V/Y, W/Z, X/A, Y/B, Z/C} {
    \draw [->,shorten >=1pt,>=stealth',semithick] (\a-1.south) -- (\b-2.north);
  }
\stoptikzpicture
}
\stopplacefigure


凯撒密码的解密方法也很一目了然,只需要将密文中的每个字母向相反方向平移规定的偏移量便可解密出明文。恺撒密码的加密、
解密算法还能够通过同余的数学方法进行计算。首先将字母表中的字母按顺序用数字代替,$A=0$,$B=1$,...,$Z=25$。
此时偏移量为$k$的加密算法的数学公式即为:

\startformula
E_k(x) = (x + k) \mod 26
\stopformula

解密算法的数学公式可以表示为:
\startformula
D_k(x) = (x + 26 - k) \mod 26
\stopformula

从加密和解密数学公式可以看出,当偏移量$k = 13$时(字母表内所有字母数量的一半),凯撒密码加密和解密算法的公式
完全相同,这是一种特殊的凯撒密码的变种算法,被称为ROT13。ROT13在英文网络论坛常常用作隐藏八卦、妙句、谜题解答以及
某些脏话的工具,目的是逃过版主或管理员的匆匆一瞥。因为ROT13的加密和解密计算公式完全相同,很明显,文字经过两次
ROT13加密之后,会恢复成原来的文字。

凯撒密码中的偏移量就相当于加密算法中的密钥。这个偏移量必须由发送者和接收者事先约定好。那么,当接收者以外的人
窃取到用凯撒密码加密后的密文之后,是不是就无法破解这个密文了呢?或者换句话说,凯撒密码能够被破解吗?

破解密码的复杂度很大程度上取决于{\it 密钥空间}(keyspace)的大小,所谓密钥空间是指密钥的取值范围到底有多大。凯撒
密码中的密钥是偏移量$k$,其取值范围为0至25的整数,共26种可能的取值,密钥空间非常有限。攻击者往往可以采用暴力破解
(brute-force attack)的方法就可以轻而易举地破解凯撒密码。

假设发送者和接收者之间约定的偏移量为3,那么明文\type{CRYPTOGRAPHY}加密后的密文则为\type{FUBSWRJUDSKB}。
当第三方窃听到密文之后,由于凯撒密码的密钥空间只有26种可能的取值,窃听者可以使用穷举搜索(exhausitive search)
的方法对每种可能的密钥取值尝试一遍:

\starttyping
k =  0: FUBSWRJUDSKB => FUBSWRJUDSKB
k =  1: FUBSWRJUDSKB => ETARVQITCRJA
k =  2: FUBSWRJUDSKB => DSZQUPHSBQIZ
k =  3: FUBSWRJUDSKB => CRYPTOGRAPHY
k =  4: FUBSWRJUDSKB => BQXOSNFQZOGX
k =  5: FUBSWRJUDSKB => APWNRMEPYNFW
k =  6: FUBSWRJUDSKB => ZOVMQLDOXMEV
k =  7: FUBSWRJUDSKB => YNULPKCNWLDU
k =  8: FUBSWRJUDSKB => XMTKOJBMVKCT
k =  9: FUBSWRJUDSKB => WLSJNIALUJBS
k = 10: FUBSWRJUDSKB => VKRIMHZKTIAR
k = 11: FUBSWRJUDSKB => UJQHLGYJSHZQ
k = 12: FUBSWRJUDSKB => TIPGKFXIRGYP
k = 13: FUBSWRJUDSKB => SHOFJEWHQFXO
k = 14: FUBSWRJUDSKB => RGNEIDVGPEWN
k = 15: FUBSWRJUDSKB => QFMDHCUFODVM
k = 16: FUBSWRJUDSKB => PELCGBTENCUL
k = 17: FUBSWRJUDSKB => ODKBFASDMBTK
k = 18: FUBSWRJUDSKB => NCJAEZRCLASJ
k = 19: FUBSWRJUDSKB => MBIZDYQBKZRI
k = 20: FUBSWRJUDSKB => LAHYCXPAJYQH
k = 21: FUBSWRJUDSKB => KZGXBWOZIXPG
k = 22: FUBSWRJUDSKB => JYFWAVNYHWOF
k = 23: FUBSWRJUDSKB => IXEVZUMXGVNE
k = 24: FUBSWRJUDSKB => HWDUYTLWFUMD
k = 25: FUBSWRJUDSKB => GVCTXSKVETLC
\stoptyping

纵览所有尝试的破解,就会发现只有当$k = 3$的时候,密文\type{FUBSWRJUDSKB}才可以解密出有意义的字符序列
\type{CRYPTOGRAPHY},即“密码学”的英文单词。因此,凯撒密码是一种极其不安全的加密方法,可以被攻击者在很快的时间
内破解,无法保护重要的秘密。


%%%%%%%%%%%%%%%%%%%%%%%%%%%%%%

\startsection[title={简单替换密码}]
\index{substitution}

\startsubsection[title={什么是简单替换密码}]
凯撒密码通过将明文中的每个字符按照在字符表中的顺序平移固定数量的字符数来生成密文。由于字符偏移量的取值空间极其有限,
致使凯撒密码能被轻而易举地破解。我们也提到了密钥空间这个概念,凯撒密码就是因为过小的密钥空间可以被攻击者使用暴力
破解的方法在非常快的时间内被破解。你可能意识到凯撒密码这种通过平移字符来实现字符替换的方法过于公式化,如果把这种映射
用随机化的方式打乱,是不是就完美了呢?这就是我们接下来要讨论的简单替换密码。

简单替换密码将字母表中的26个字母,分别与其他字母建立一一映射的关系,这种映射关系不像凯撒密码那样通过平移字符这种
线性化的方法,而是用一个映射表来描述明文字符和密文字符之间的映射关系,这种映射表也称为字符替换表。为了更直观地展示
字符之间的映射关系,我们把明文中的字符都用小写字母表示,密文中的字符都用大写字母表示。\in{图}[fig:substitution]
就是一个简单的字符替换表。

% OrderedDict([('a', 'D'), ('b', 'H'), ('c', 'V'), ('d', 'E'), ('e', 'J'), ('f', 'P'), ('g', 'O'), ('h', 'Y'), ('i', 'T'), ('j', 'R'), ('k', 'U'), ('l', 'Q'), ('m', 'C'), ('n', 'G'), ('o', 'W'), ('p', 'L'), ('q', 'A'), ('r', 'Z'), ('s', 'K'), ('t', 'N'), ('u', 'S'), ('v', 'B'), ('w', 'F'), ('x', 'I'), ('y', 'X'), ('z', 'M')])
\startplacefigure
  [title={简单替换密码的映射表}, reference=fig:substitution]

\midaligned{
\starttikzpicture
  [pre/.style={<-,shorten <=1pt,>=stealth',semithick},
  post/.style={->,shorten >=1pt,>=stealth',semithick},
  every node/.style={draw=red!62.5!black, thick, fill=white!62.5!black, minimum width=1.3em, minimum height=1.3em, font=\Tiny}]

  \foreach \a/\n in {a/0, b/1, c/2, d/3, e/4, f/5, g/6, h/7, i/8, 
    j/9, k/10, l/11, m/12, n/13, o/14, p/15, q/16, r/17, s/18,
    t/19, u/20, v/21, w/22, x/23, y/24, z/25} {
    \node (\a) at (\n * 5.2mm, 3) {\a};
  }

  \foreach \b/\n in {A/0, B/1, C/2, D/3, E/4, F/5, G/6, H/7, I/8, 
    J/9, K/10, L/11, M/12, N/13, O/14, P/15, Q/16, R/17, S/18,
    T/19, U/20, V/21, W/22, X/23, Y/24, Z/25} {
    \node (\b) at (\n * 5.2mm, 0) {\b};
  }

  \foreach \a/\b in {a/D, b/H, c/V, d/E, e/J, f/P, g/O, h/Y, i/T,
    j/R, k/U, l/Q, m/C, n/G, o/W, p/L, q/A, r/Z, s/K, t/N, u/S,
    v/B, w/F, x/I, y/X, z/M} {
    \draw [->,shorten >=1pt,>=stealth',semithick] (\a.south) -- (\b.north);
  }
\stoptikzpicture
}
\stopplacefigure

显然,\in{图}[fig:substitution]表示的字符替换关系不像凯撒密码那么有规律,明文字符和密文字符之间的映射看起来是
无章可循的。可以说,凯撒密码是简单字符替换密码的一个特例。为了更好地展示明文字符和密文字符之间的替换关系,我们对
\in{图}[fig:substitution]稍作转换,\in{图}[fig:substitution2],但仍然保持字符之间原来的映射关系。

\startplacefigure
  [title={变换后的简单替换密码的映射表}, reference=fig:substitution2]

\midaligned{
\starttikzpicture
  [pre/.style={<-,shorten <=1pt,>=stealth',semithick},
  post/.style={->,shorten >=1pt,>=stealth',semithick},
  every node/.style={draw=red!62.5!black, thick, fill=white!62.5!black, minimum width=1.3em, minimum height=1.3em, font=\Tiny}]

  \foreach \a/\n in {a/0, b/1, c/2, d/3, e/4, f/5, g/6, h/7, i/8, 
    j/9, k/10, l/11, m/12, n/13, o/14, p/15, q/16, r/17, s/18,
    t/19, u/20, v/21, w/22, x/23, y/24, z/25} {
    \node (\a) at (\n * 5.2mm, 3) {\a};
  }

  \foreach \b/\n in {D/0, H/1, V/2, E/3, J/4, P/5, O/6, Y/7, T/8, 
    R/9, U/10, Q/11, C/12, G/13, W/14, L/15, A/16, Z/17, K/18,
    N/19, S/20, B/21, F/22, I/23, X/24, M/25} {
    \node (\b) at (\n * 5.2mm, 0) {\b};
  }

  \foreach \a/\b in {a/D, b/H, c/V, d/E, e/J, f/P, g/O, h/Y, i/T,
    j/R, k/U, l/Q, m/C, n/G, o/W, p/L, q/A, r/Z, s/K, t/N, u/S,
    v/B, w/F, x/I, y/X, z/M} {
    \draw [->,shorten >=1pt,>=stealth',semithick] (\a.south) -- (\b.north);
  }
\stoptikzpicture
}
\stopplacefigure

凯撒密码可以用暴力破解来破译,但简单替换密码则不然。前面,我们提到,密码算法被破译的困难程度取决于密钥空间的大小。
我们来看看简单替换密码的密钥空间。明文字母中的\type{a}可以对应\type{A, B, ..., Z}这26个字母中的任意一个
(26种),\type{b}可以对应除了\type{a}所对应的字母以外的剩余25个字母中的任意一个(25种)。以此类推,我们可以
计算出简单替换密码的密钥空间大小是:

\startformula
26 \times 25 \times 24 \times \cdots \times 1 = 403291461126605635584000000
\stopformula

这个数字约等于$400 \times 10^{24}$,密钥的数量如此巨大,用暴力破解进行穷举搜索就非常困难了。我们假设以当前
(2018年11月)排名第一的Summit超级计算机在峰值性能下每秒约200亿次浮点计算的速度来遍历密钥的话,要遍历完所有
的密钥也需要花费超过6千万年的时间。这还是用我们当前最顶级的超级计算机的峰值计算速度来遍历的,普通的家用计算机
将要花费几百亿年的时间才能遍历完所有的密钥。由此可见,简单替换密码的密钥空间是足够大的。
\stopsubsection

\startsubsection[title={用频率分析的方法破解简单替换密码}]
超大的密钥空间让破译简单替换密码看起来变得不可能,但密码破译工作者发现用频率分析的密码破译方法,使破译简单替换密码
成为可能了。

在任何一种书面语言中,不同的字母或字母组合出现的频率各不相同。而且,对于以这种语言写的任意一段文本,都具有大致相同
的特征字母分布。比如,在英语中,字母\type{e}出现的频率很高,而\type{x}出现的较少。类似地,字母组合\type{st}、
\type{ng}、\type{th}以及\type{qu}等双字母组合出现的频率非常高,\type{nz}、\type{qj}组合则极少。
\in{表}[tab:freq]是人们从大量的英文文章中统计出的字母频率。

\placetable
  [here][tab:freq]
  {英文字母出现的频率表}
  \starttable[|c|c|c|c|]
  \HL
  \NC {\bf 字母} \NC {\bf 频率} \VL {\bf 字母} \NC {\bf 频率} \NC\SR
  \HL
  \NC e   \NC 11.1607\% \VL m   \NC 3.0129\% \NC\FR
  \NC a   \NC  8.4966\% \VL h   \NC 3.0034\% \NC\MR
  \NC r   \NC  7.5809\% \VL g   \NC 2.4705\% \NC\MR
  \NC i   \NC  7.5448\% \VL b   \NC 2.0720\% \NC\MR
  \NC o   \NC  7.1635\% \VL f   \NC 1.8121\% \NC\MR
  \NC t   \NC  6.9509\% \VL y   \NC 1.7779\% \NC\MR
  \NC n   \NC  6.6544\% \VL w   \NC 1.2899\% \NC\MR
  \NC s   \NC  5.7351\% \VL k   \NC 1.1016\% \NC\MR
  \NC l   \NC  5.4893\% \VL v   \NC 1.0074\% \NC\MR
  \NC c   \NC  4.5388\% \VL x   \NC 0.2902\% \NC\MR
  \NC u   \NC  3.6308\% \VL z   \NC 0.2722\% \NC\MR
  \NC d   \NC  3.3844\% \VL j   \NC 0.1965\% \NC\MR
  \NC p   \NC  3.1671\% \VL q   \NC 0.1962\% \NC\LR
  \HL
  \stoptable

简单替换密码的密钥空间如此巨大,但它的弱点也是显而易见的,就是明文中相同的字母在转换为密文后总是被同一个字母所
替换。我们参考这个英文字母频率表来实际尝试破译一段密文。现在,假设我们得到下面一段经过简单替换密码加密过后的
密文,其明文都是小写英文字母。

\starttyping
EAOQASBAEEAOQECPENFECQMRQFENABDCQECQSENFBAOZQSNBECQUNBLEAFRKKQSECQFZNBVFPBLPSSADFAKAR
ESPVQARFKASERBQASEAEPWQPSUFPVPNBFEPFQPAKESAROZQFPBLOIAGGAFNBVQBLECQUEALNQEAFZQQGBAUAS
QPBLOIPFZQQGEAFPIDQQBLECQCQPSEPTCQPBLECQECARFPBLBPERSPZFCATWFECPEKZQFCNFCQNSEAENFPTAB
FRUUPENABLQHAREZIEAOQDNFCLEALNQEAFZQQGEAFZQQGGQSTCPBTQEALSQPUPIECQSQFECQSROKASNBECPEF
ZQQGAKLQPECDCPELSQPUFUPITAUQDCQBDQCPHQFCRKKZQLAKKECNFUASEPZTANZURFEVNHQRFGPRFQECQSQFE
CQSQFGQTEECPEUPWQFTPZPUNEIAKFAZABVZNKQKASDCADARZLOQPSECQDCNGFPBLFTASBFAKENUQECAGGSQFF
ASFDSABVECQGSARLUPBFTABERUQZIECQGPBVFAKLNFGSNXLZAHQECQZPDFLQZPIECQNBFAZQBTQAKAKKNTQPB
LECQFGRSBFECPEGPENQBEUQSNEAKECRBDASECIEPWQFDCQBCQCNUFQZKUNVCECNFMRNQERFUPWQDNECPOPSQO
ALWNBDCADARZLKPSLQZFOQPSEAVSRBEPBLFDQPERBLQSPDQPSIZNKQOREECPEECQLSQPLAKFAUQECNBVPKEQS
LQPECECQRBLNFTAHQSQLTARBESIKSAUDCAFQOARSBBAESPHQZZQSSQERSBFGRXXZQFECQDNZZPBLUPWQFRFSP
ECQSOQPSECAFQNZZFDQCPHQECPBKZIEAAECQSFECPEDQWBADBAEAKECRFTABFTNQBTQLAQFUPWQTADPSLFAKR
FPZZPBLECRFECQBPENHQCRQAKSQFAZRENABNFFNTWZNQLAQSDNECECQGPZQTPFEAKECARVCEPBLQBEQSGSNFQ
FAKVSQPEGNETCPBLUAUQBEDNECECNFSQVPSLECQNSTRSSQBEFERSBPDSIPBLZAFQECQBPUQAKPTENAB
\stoptyping

首先,我们统计这段密文中各个字母出现的次数和频率,结果如\in{表}[tab:chars-freq]所示。

\placetable
  [here][tab:chars-freq]
  {密文中各英文字母出现的次数和频率}
  \starttable[|c|c|c|c|c|c|]
  \HL
  \NC {\bf 字母} \NC {\bf 次数} \NC {\bf 频率} \VL {\bf 字母} \NC {\bf 次数} \NC {\bf 频率} \NC\SR
  \HL
  \NC Q   \NC 137  \NC 12.47\% \VL K   \NC 34 \NC 3.09\% \NC\FR
  \NC E   \NC 117  \NC 10.65\% \VL D   \NC 28 \NC 2.55\% \NC\MR
  \NC A   \NC 93   \NC  8.46\% \VL U   \NC 28 \NC 2.55\% \NC\MR
  \NC P   \NC 84   \NC  7.64\% \VL T   \NC 24 \NC 2.18\% \NC\MR
  \NC F   \NC 82   \NC  7.46\% \VL G   \NC 22 \NC 2.00\% \NC\MR
  \NC C   \NC 75   \NC  6.82\% \VL O   \NC 15 \NC 1.36\% \NC\MR
  \NC S   \NC 68   \NC  6.19\% \VL I   \NC 14 \NC 1.27\% \NC\MR
  \NC B   \NC 65   \NC  5.91\% \VL V   \NC 14 \NC 1.27\% \NC\MR
  \NC N   \NC 53   \NC  4.82\% \VL W   \NC 10 \NC 0.91\% \NC\MR
  \NC L   \NC 42   \NC  3.82\% \VL H   \NC 8  \NC 0.73\% \NC\MR
  \NC Z   \NC 41   \NC  3.73\% \VL X   \NC 3  \NC 0.27\% \NC\MR
  \NC R   \NC 40   \NC  3.64\% \VL M   \NC 2  \NC 0.18\% \NC\LR
  \HL
  \stoptable

根据密码研究工作者总结出来的字母频率\in{表}[tab:freq],字母\type{e}的出现频率远高于其他字母。
经统计后,我们发现密文中字母\type{Q}的出现频率最高。我们先暂且假设字母\type{Q}就是由\type{e}变换
而来的,这样,我们把密文中的\type{Q}替换回\type{e},就得到下面的字母序列。

\starttyping
EAOeASBAEEAOeECPENFECeMReFENABDCeECeSENFBAOZeSNBECeUNBLEAFRKKeSECeFZNBVFPBLPSSADFAKAR
ESPVeARFKASERBeASEAEPWePSUFPVPNBFEPFePAKESAROZeFPBLOIAGGAFNBVeBLECeUEALNeEAFZeeGBAUAS
ePBLOIPFZeeGEAFPIDeeBLECeCePSEPTCePBLECeECARFPBLBPERSPZFCATWFECPEKZeFCNFCeNSEAENFPTAB
FRUUPENABLeHAREZIEAOeDNFCLEALNeEAFZeeGEAFZeeGGeSTCPBTeEALSePUPIECeSeFECeSROKASNBECPEF
ZeeGAKLePECDCPELSePUFUPITAUeDCeBDeCPHeFCRKKZeLAKKECNFUASEPZTANZURFEVNHeRFGPRFeECeSeFE
CeSeFGeTEECPEUPWeFTPZPUNEIAKFAZABVZNKeKASDCADARZLOePSECeDCNGFPBLFTASBFAKENUeECAGGSeFF
ASFDSABVECeGSARLUPBFTABERUeZIECeGPBVFAKLNFGSNXLZAHeECeZPDFLeZPIECeNBFAZeBTeAKAKKNTePB
LECeFGRSBFECPEGPENeBEUeSNEAKECRBDASECIEPWeFDCeBCeCNUFeZKUNVCECNFMRNeERFUPWeDNECPOPSeO
ALWNBDCADARZLKPSLeZFOePSEAVSRBEPBLFDePERBLeSPDePSIZNKeOREECPEECeLSePLAKFAUeECNBVPKEeS
LePECECeRBLNFTAHeSeLTARBESIKSAUDCAFeOARSBBAESPHeZZeSSeERSBFGRXXZeFECeDNZZPBLUPWeFRFSP
ECeSOePSECAFeNZZFDeCPHeECPBKZIEAAECeSFECPEDeWBADBAEAKECRFTABFTNeBTeLAeFUPWeTADPSLFAKR
FPZZPBLECRFECeBPENHeCReAKSeFAZRENABNFFNTWZNeLAeSDNECECeGPZeTPFEAKECARVCEPBLeBEeSGSNFe
FAKVSePEGNETCPBLUAUeBEDNECECNFSeVPSLECeNSTRSSeBEFERSBPDSIPBLZAFeECeBPUeAKPTENAB
\stoptyping


英文文章中,以字母\type{e}结尾的单词,\type{the}的出现频率极高,对上面的这段字符序列,进一步统计
发现\type{ECe}出现了27次,远远高于其他以\type{e}结尾的包含3个字母的字符串的出现次数。我们进一步
假定\type{t}被替换成了\type{E},\type{h}被替换成了\type{C},于是,我们继续将上面字符序列
中\type{E}和\type{C}分别替换回\type{t}和\type{h},得到:

\starttyping
tAOeASBAttAOethPtNFtheMReFtNABDhetheStNFBAOZeSNBtheUNBLtAFRKKeStheFZNBVFPBLPSSADFAKAR
tSPVeARFKAStRBeAStAtPWePSUFPVPNBFtPFePAKtSAROZeFPBLOIAGGAFNBVeBLtheUtALNetAFZeeGBAUAS
ePBLOIPFZeeGtAFPIDeeBLthehePStPThePBLthethARFPBLBPtRSPZFhATWFthPtKZeFhNFheNStAtNFPTAB
FRUUPtNABLeHARtZItAOeDNFhLtALNetAFZeeGtAFZeeGGeSThPBTetALSePUPItheSeFtheSROKASNBthPtF
ZeeGAKLePthDhPtLSePUFUPITAUeDheBDehPHeFhRKKZeLAKKthNFUAStPZTANZURFtVNHeRFGPRFetheSeFt
heSeFGeTtthPtUPWeFTPZPUNtIAKFAZABVZNKeKASDhADARZLOePStheDhNGFPBLFTASBFAKtNUethAGGSeFF
ASFDSABVtheGSARLUPBFTABtRUeZItheGPBVFAKLNFGSNXLZAHetheZPDFLeZPItheNBFAZeBTeAKAKKNTePB
LtheFGRSBFthPtGPtNeBtUeSNtAKthRBDASthItPWeFDheBhehNUFeZKUNVhthNFMRNetRFUPWeDNthPOPSeO
ALWNBDhADARZLKPSLeZFOePStAVSRBtPBLFDePtRBLeSPDePSIZNKeORtthPttheLSePLAKFAUethNBVPKteS
LePththeRBLNFTAHeSeLTARBtSIKSAUDhAFeOARSBBAtSPHeZZeSSetRSBFGRXXZeFtheDNZZPBLUPWeFRFSP
theSOePSthAFeNZZFDehPHethPBKZItAAtheSFthPtDeWBADBAtAKthRFTABFTNeBTeLAeFUPWeTADPSLFAKR
FPZZPBLthRFtheBPtNHehReAKSeFAZRtNABNFFNTWZNeLAeSDNththeGPZeTPFtAKthARVhtPBLeBteSGSNFe
FAKVSePtGNtThPBLUAUeBtDNththNFSeVPSLtheNSTRSSeBtFtRSBPDSIPBLZAFetheBPUeAKPTtNAB
\stoptyping

进一步分析,我们发现\type{thPt}也多次出现,英文中单词\type{that}出现的频率也是特别高的。同时,我们
发现\type{P}在这段密文中出现的频率也是极高的,我们几乎可以不假思索地猜测\type{a}被替换成了\type{P}。
把\type{P}替换回\type{a},我们得到:

\starttyping
tAOeASBAttAOethatNFtheMReFtNABDhetheStNFBAOZeSNBtheUNBLtAFRKKeStheFZNBVFaBLaSSADFAKAR
tSaVeARFKAStRBeAStAtaWeaSUFaVaNBFtaFeaAKtSAROZeFaBLOIAGGAFNBVeBLtheUtALNetAFZeeGBAUAS
eaBLOIaFZeeGtAFaIDeeBLtheheaStaTheaBLthethARFaBLBatRSaZFhATWFthatKZeFhNFheNStAtNFaTAB
FRUUatNABLeHARtZItAOeDNFhLtALNetAFZeeGtAFZeeGGeSThaBTetALSeaUaItheSeFtheSROKASNBthatF
ZeeGAKLeathDhatLSeaUFUaITAUeDheBDehaHeFhRKKZeLAKKthNFUAStaZTANZURFtVNHeRFGaRFetheSeFt
heSeFGeTtthatUaWeFTaZaUNtIAKFAZABVZNKeKASDhADARZLOeaStheDhNGFaBLFTASBFAKtNUethAGGSeFF
ASFDSABVtheGSARLUaBFTABtRUeZItheGaBVFAKLNFGSNXLZAHetheZaDFLeZaItheNBFAZeBTeAKAKKNTeaB
LtheFGRSBFthatGatNeBtUeSNtAKthRBDASthItaWeFDheBhehNUFeZKUNVhthNFMRNetRFUaWeDNthaOaSeO
ALWNBDhADARZLKaSLeZFOeaStAVSRBtaBLFDeatRBLeSaDeaSIZNKeORtthattheLSeaLAKFAUethNBVaKteS
LeaththeRBLNFTAHeSeLTARBtSIKSAUDhAFeOARSBBAtSaHeZZeSSetRSBFGRXXZeFtheDNZZaBLUaWeFRFSa
theSOeaSthAFeNZZFDehaHethaBKZItAAtheSFthatDeWBADBAtAKthRFTABFTNeBTeLAeFUaWeTADaSLFAKR
FaZZaBLthRFtheBatNHehReAKSeFAZRtNABNFFNTWZNeLAeSDNththeGaZeTaFtAKthARVhtaBLeBteSGSNFe
FAKVSeatGNtThaBLUAUeBtDNththNFSeVaSLtheNSTRSSeBtFtRSBaDSIaBLZAFetheBaUeAKaTtNAB
\stoptyping

继续猜测,\type{theSe}会不会是\type{there}呢,\type{Leath}会不会是\type{death}呢,于是,我们
用\type{r}和\type{d}分别替换回\type{S}和\type{L},得到:

\starttyping
tAOeArBAttAOethatNFtheMReFtNABDhethertNFBAOZerNBtheUNBdtAFRKKertheFZNBVFaBdarrADFAKAR
traVeARFKArtRBeArtAtaWearUFaVaNBFtaFeaAKtrAROZeFaBdOIAGGAFNBVeBdtheUtAdNetAFZeeGBAUAr
eaBdOIaFZeeGtAFaIDeeBdtheheartaTheaBdthethARFaBdBatRraZFhATWFthatKZeFhNFheNrtAtNFaTAB
FRUUatNABdeHARtZItAOeDNFhdtAdNetAFZeeGtAFZeeGGerThaBTetAdreaUaIthereFtherROKArNBthatF
ZeeGAKdeathDhatdreaUFUaITAUeDheBDehaHeFhRKKZedAKKthNFUArtaZTANZURFtVNHeRFGaRFethereFt
hereFGeTtthatUaWeFTaZaUNtIAKFAZABVZNKeKArDhADARZdOeartheDhNGFaBdFTArBFAKtNUethAGGreFF
ArFDrABVtheGrARdUaBFTABtRUeZItheGaBVFAKdNFGrNXdZAHetheZaDFdeZaItheNBFAZeBTeAKAKKNTeaB
dtheFGRrBFthatGatNeBtUerNtAKthRBDArthItaWeFDheBhehNUFeZKUNVhthNFMRNetRFUaWeDNthaOareO
AdWNBDhADARZdKardeZFOeartAVrRBtaBdFDeatRBderaDearIZNKeORtthatthedreadAKFAUethNBVaKter
deaththeRBdNFTAHeredTARBtrIKrAUDhAFeOARrBBAtraHeZZerretRrBFGRXXZeFtheDNZZaBdUaWeFRFra
therOearthAFeNZZFDehaHethaBKZItAAtherFthatDeWBADBAtAKthRFTABFTNeBTedAeFUaWeTADardFAKR
FaZZaBdthRFtheBatNHehReAKreFAZRtNABNFFNTWZNedAerDNththeGaZeTaFtAKthARVhtaBdeBterGrNFe
FAKVreatGNtThaBdUAUeBtDNththNFreVardtheNrTRrreBtFtRrBaDrIaBdZAFetheBaUeAKaTtNAB
\stoptyping

结合\type{Dhether}和\type{Dhat},我们推测\type{D}是由\type{w}替换过来的。进一步,
\type{DNth}极有可能就是\type{with},以此类推,\type{DNZZ}可能是\type{will},\type{NF}可能
是\type{is}。用\type{w}、\type{i}和\type{l}分别替换\type{D}、\type{N}和\type{Z},得到:

\starttyping
tAOeArBAttAOethatiFtheMReFtiABwhethertiFBAOleriBtheUiBdtAFRKKertheFliBVFaBdarrAwFAKAR
traVeARFKArtRBeArtAtaWearUFaVaiBFtaFeaAKtrAROleFaBdOIAGGAFiBVeBdtheUtAdietAFleeGBAUAr
eaBdOIaFleeGtAFaIweeBdtheheartaTheaBdthethARFaBdBatRralFhATWFthatKleFhiFheirtAtiFaTAB
FRUUatiABdeHARtlItAOewiFhdtAdietAFleeGtAFleeGGerThaBTetAdreaUaIthereFtherROKAriBthatF
leeGAKdeathwhatdreaUFUaITAUewheBwehaHeFhRKKledAKKthiFUArtalTAilURFtViHeRFGaRFethereFt
hereFGeTtthatUaWeFTalaUitIAKFAlABVliKeKArwhAwARldOearthewhiGFaBdFTArBFAKtiUethAGGreFF
ArFwrABVtheGrARdUaBFTABtRUelItheGaBVFAKdiFGriXdlAHethelawFdelaItheiBFAleBTeAKAKKiTeaB
dtheFGRrBFthatGatieBtUeritAKthRBwArthItaWeFwheBhehiUFelKUiVhthiFMRietRFUaWewithaOareO
AdWiBwhAwARldKardelFOeartAVrRBtaBdFweatRBderawearIliKeORtthatthedreadAKFAUethiBVaKter
deaththeRBdiFTAHeredTARBtrIKrAUwhAFeOARrBBAtraHellerretRrBFGRXXleFthewillaBdUaWeFRFra
therOearthAFeillFwehaHethaBKlItAAtherFthatweWBAwBAtAKthRFTABFTieBTedAeFUaWeTAwardFAKR
FallaBdthRFtheBatiHehReAKreFAlRtiABiFFiTWliedAerwiththeGaleTaFtAKthARVhtaBdeBterGriFe
FAKVreatGitThaBdUAUeBtwiththiFreVardtheirTRrreBtFtRrBawrIaBdlAFetheBaUeAKaTtiAB
\stoptyping

靠近句尾的地方出现了\type{withthisreVard},这个可能是\type{with this regard},也可能
是\type{with this reward}。但是我们可以立即排除后者,因为我们在上一步已经推测了\type{D}是
由\type{w}替换过来的,所以,我们推测\type{V}是由\type{g}替换过来的。现在,我们把已经推测出来的字母放到
\in{表}[tab:chars-freq1]中,如\in{表}[tab:chars-freq1]所示。

\placetable
  [here][tab:chars-freq1]
  {使用频率分析方法已经推测出来的字母}
  \starttable[|c|c|c|c|c|c|]
  \HL
  \NC {\bf 字母} \NC {\bf 次数} \NC {\bf 频率} \VL {\bf 字母} \NC {\bf 次数} \NC {\bf 频率} \NC\SR
  \HL
  \NC Q $<-$ e \NC 137  \NC 12.47\% \VL K        \NC 34 \NC 3.09\% \NC\FR
  \NC E $<-$ t \NC 117  \NC 10.65\% \VL D $<-$ w \NC 28 \NC 2.55\% \NC\MR
  \NC A        \NC 93   \NC  8.46\% \VL U        \NC 28 \NC 2.55\% \NC\MR
  \NC P $<-$ a \NC 84   \NC  7.64\% \VL T        \NC 24 \NC 2.18\% \NC\MR
  \NC F $<-$ s \NC 82   \NC  7.46\% \VL G        \NC 22 \NC 2.00\% \NC\MR
  \NC C $<-$ h \NC 75   \NC  6.82\% \VL O        \NC 15 \NC 1.36\% \NC\MR
  \NC S $<-$ r \NC 68   \NC  6.19\% \VL I        \NC 14 \NC 1.27\% \NC\MR
  \NC B        \NC 65   \NC  5.91\% \VL V        \NC 14 \NC 1.27\% \NC\MR
  \NC N $<-$ i \NC 53   \NC  4.82\% \VL W        \NC 10 \NC 0.91\% \NC\MR
  \NC L $<-$ d \NC 42   \NC  3.82\% \VL H        \NC 8  \NC 0.73\% \NC\MR
  \NC Z $<-$ l \NC 41   \NC  3.73\% \VL X        \NC 3  \NC 0.27\% \NC\MR
  \NC R        \NC 40   \NC  3.64\% \VL M        \NC 2  \NC 0.18\% \NC\LR
  \HL
  \stoptable

到此为止,排在前五的高频字母只剩下\type{A}还没有推测出来,那我们对照字母频率\in{表}[tab:freq],发现高频
字母中只有\type{o}和\type{n}还没有被反推出来。我们大胆地假设,\type{A}就是由\type{o}或者\type{n}替换
过来的,然后,我们通过\type{whA}很快排除\type{n},所以,我们推测\type{A}是由\type{o}替换过来的,继续
还原字母序列,我们得到:

\starttyping
toOeorBottoOethatiFtheMReFtioBwhethertiFBoOleriBtheUiBdtoFRKKertheFliBgFaBdarrowFoKoR
trageoRFKortRBeortotaWearUFagaiBFtaFeaoKtroROleFaBdOIoGGoFiBgeBdtheUtodietoFleeGBoUor
eaBdOIaFleeGtoFaIweeBdtheheartaTheaBdthethoRFaBdBatRralFhoTWFthatKleFhiFheirtotiFaToB
FRUUatioBdeHoRtlItoOewiFhdtodietoFleeGtoFleeGGerThaBTetodreaUaIthereFtherROKoriBthatF
leeGoKdeathwhatdreaUFUaIToUewheBwehaHeFhRKKledoKKthiFUortalToilURFtgiHeRFGaRFethereFt
hereFGeTtthatUaWeFTalaUitIoKFoloBgliKeKorwhowoRldOearthewhiGFaBdFTorBFoKtiUethoGGreFF
orFwroBgtheGroRdUaBFToBtRUelItheGaBgFoKdiFGriXdloHethelawFdelaItheiBFoleBTeoKoKKiTeaB
dtheFGRrBFthatGatieBtUeritoKthRBworthItaWeFwheBhehiUFelKUighthiFMRietRFUaWewithaOareO
odWiBwhowoRldKardelFOeartogrRBtaBdFweatRBderawearIliKeORtthatthedreadoKFoUethiBgaKter
deaththeRBdiFToHeredToRBtrIKroUwhoFeOoRrBBotraHellerretRrBFGRXXleFthewillaBdUaWeFRFra
therOearthoFeillFwehaHethaBKlItootherFthatweWBowBotoKthRFToBFTieBTedoeFUaWeTowardFoKR
FallaBdthRFtheBatiHehReoKreFolRtioBiFFiTWliedoerwiththeGaleTaFtoKthoRghtaBdeBterGriFe
FoKgreatGitThaBdUoUeBtwiththiFregardtheirTRrreBtFtRrBawrIaBdloFetheBaUeoKaTtioB
\stoptyping

接下来,我们发现开头几个单词的组合\type{toOeorBottoOethatistheMRestioB},这大概
是\type{to be or not to be that is the question}吧,其中,\type{b}$->$\type{O},
\type{n}$->$\type{B},\type{q}$->$\type{M},\type{u}$->$\type{R}。进一步将这些字母替换
进去,得到:

\starttyping
tobeornottobethatiFthequeFtionwhethertiFnoblerintheUindtoFuKKertheFlingFandarrowFoKou
trageouFKortuneortotaWearUFagainFtaFeaoKtroubleFandbIoGGoFingendtheUtodietoFleeGnoUor
eandbIaFleeGtoFaIweendtheheartaTheandthethouFandnaturalFhoTWFthatKleFhiFheirtotiFaTon
FuUUationdeHoutlItobewiFhdtodietoFleeGtoFleeGGerThanTetodreaUaIthereFtherubKorinthatF
leeGoKdeathwhatdreaUFUaIToUewhenwehaHeFhuKKledoKKthiFUortalToilUuFtgiHeuFGauFethereFt
hereFGeTtthatUaWeFTalaUitIoKFolongliKeKorwhowouldbearthewhiGFandFTornFoKtiUethoGGreFF
orFwrongtheGroudUanFTontuUelItheGangFoKdiFGriXdloHethelawFdelaItheinFolenTeoKoKKiTean
dtheFGurnFthatGatientUeritoKthunworthItaWeFwhenhehiUFelKUighthiFquietuFUaWewithabareb
odWinwhowouldKardelFbeartogruntandFweatunderawearIliKebutthatthedreadoKFoUethingaKter
deaththeundiFToHeredTountrIKroUwhoFebournnotraHellerreturnFGuXXleFthewillandUaWeFuFra
therbearthoFeillFwehaHethanKlItootherFthatweWnownotoKthuFTonFTienTedoeFUaWeTowardFoKu
FallandthuFthenatiHehueoKreFolutioniFFiTWliedoerwiththeGaleTaFtoKthoughtandenterGriFe
FoKgreatGitThandUoUentwiththiFregardtheirTurrentFturnawrIandloFethenaUeoKaTtion
\stoptyping

至此,我们的破译工作基本结束。我们基本上可以确定这段密文就是莎士比亚名著《哈姆雷特》中关于“生存和毁灭”
的名段了。我们把原著和现在已经部分破译好的字母序列对比,就能确定所有字母的替换关系
如\in{图}[fig:crack-sub]所示。

% python dict: {'a': 'P', 'c': 'T', 'b': 'O', 'e': 'Q', 'd': 'L', 'g': 'V', 'f': 'K', 'i': 'N', 'h': 'C', 'k': 'W', 'j': 'Y', 'm': 'U', 'l': 'Z', 'o': 'A', 'n': 'B', 'q': 'M', 'p': 'G', 's': 'F', 'r': 'S', 'u': 'R', 't': 'E', 'w': 'D', 'v': 'H', 'y': 'I', 'x': 'J', 'z': 'X'}
\startplacefigure
  [title={变换后的简单替换密码的映射表}, reference=fig:crack-sub]

\midaligned{
\starttikzpicture
  [pre/.style={<-,shorten <=1pt,>=stealth',semithick},
  post/.style={->,shorten >=1pt,>=stealth',semithick},
  every node/.style={draw=red!62.5!black, thick, fill=white!62.5!black, minimum width=1.3em, minimum height=1.3em, font=\Tiny}]

  \foreach \a/\n in {a/0, b/1, c/2, d/3, e/4, f/5, g/6, h/7, i/8, 
    j/9, k/10, l/11, m/12, n/13, o/14, p/15, q/16, r/17, s/18,
    t/19, u/20, v/21, w/22, x/23, y/24, z/25} {
    \node (\a) at (\n * 5.2mm, 3) {\a};
  }

  \foreach \b/\n in {P/0, O/1, T/2, L/3, Q/4, K/5, V/6, C/7, N/8, 
    Y/9, W/10, Z/11, U/12, B/13, A/14, G/15, M/16, S/17, F/18,
    E/19, R/20, H/21, D/22, J/23, I/24, X/25} {
    \node (\b) at (\n * 5.2mm, 0) {\b};
  }

  \foreach \a/\b in {a/P, c/T, b/O, e/Q, d/L, g/V, f/K, i/N, h/C,
  k/W, j/Y, m/U, l/Z, o/A, n/B, q/M, p/G, s/F,
  r/S, u/R, t/E, w/D, v/H, y/I, x/J, z/X} {
    \draw [->,shorten >=1pt,>=stealth',semithick] (\a.south) -- (\b.north);
  }
\stoptikzpicture
}
\stopplacefigure

\stopsection

明文如下:
\starttyping
tobeornottobethatisthequestionwhethertisnoblerinthemindtosuffertheslingsandarrowsofou
trageousfortuneortotakearmsagainstaseaoftroublesandbyopposingendthemtodietosleepnomor
eandbyasleeptosayweendtheheartacheandthethousandnaturalshocksthatfleshisheirtotisacon
summationdevoutlytobewishdtodietosleeptosleepperchancetodreamaytherestherubforinthats
leepofdeathwhatdreamsmaycomewhenwehaveshuffledoffthismortalcoilmustgiveuspausetherest
herespectthatmakescalamityofsolonglifeforwhowouldbearthewhipsandscornsoftimethoppress
orswrongtheproudmanscontumelythepangsofdisprizdlovethelawsdelaytheinsolenceofofficean
dthespurnsthatpatientmeritofthunworthytakeswhenhehimselfmighthisquietusmakewithabareb
odkinwhowouldfardelsbeartogruntandsweatunderawearylifebutthatthedreadofsomethingafter
deaththeundiscoveredcountryfromwhosebournnotravellerreturnspuzzlesthewillandmakesusra
therbearthoseillswehavethanflytoothersthatweknownotofthusconsciencedoesmakecowardsofu
sallandthusthenativehueofresolutionissickliedoerwiththepalecastofthoughtandenterprise
sofgreatpitchandmomentwiththisregardtheircurrentsturnawryandlosethenameofaction
\stoptyping

给明文补上空格和标点符号并断句之后,可读性就更好了:

\startalignment[middle]
\starttyping
To be, or not to be, that is the question:
Whether 'tis nobler in the mind to suffer
The slings and arrows of outrageous fortune,
Or to take arms against a sea of troubles
And by opposing end them. To die-to sleep,
No more; and by a sleep to say we end
The heart-ache and the thousand natural shocks
That flesh is heir to: 'tis a consummation
Devoutly to be wish'd. To die, to sleep;
To sleep, perchance to dream-ay, there's the rub:
For in that sleep of death what dreams may come,
When we have shuffled off this mortal coil,
Must give us pause-there's the respect
That makes calamity of so long life.
For who would bear the whips and scorns of time,
Th'oppressor's wrong, the proud man's contumely,
The pangs of dispriz'd love, the law's delay,
The insolence of office, and the spurns
That patient merit of th'unworthy takes,
When he himself might his quietus make
With a bare bodkin? Who would fardels bear,
To grunt and sweat under a weary life,
But that the dread of something after death,
The undiscovere'd country, from whose bourn
No traveller returns, puzzles the will,
And makes us rather bear those ills we have
Than fly to others that we know not of?
Thus conscience does make cowards of us all,
And thus the native hue of resolution
Is sicklied o'er with the pale cast of thought,
And enterprises of great pitch and moment
With this regard their currents turn awry
And lose the name of action.
\stoptyping
\stopalignment

通过上述破解过程,我们可以了解到利用频率分析破译简单替换密码可以从高频字母着手,同时利用高频单词查找线索。
常用的词组也可能成为线索,同时密文越长越容易破解,因为长密文统计出来的字母频率表更接近密码工作研究者们总结
出来的字母频率表。

早在公元九世纪,阿拉伯的密码破译专家就已经能够娴熟地运用统计字母出现频率的方法来破译简单替换密码,柯南·道尔在
他著名的福尔摩斯探案《跳舞的小人》里就非常详细地叙述了福尔摩斯使用频率统计法破译跳舞人形密码(也就是简单替换
密码)的过程。

%%%%%%%%%%%%%%%%%%%%%%%%%%%%%%

\startsection[title={复式替换密码:Enigma}]
\index{enigma}

Enigma这个名字在德语里是“谜”的意思,它是由德国人阿瑟·谢尔比乌斯 (Arthur Sherbius) 发明的一种能够进行
加密和解密操作的机器。在刚刚发明之际,Enigma被用在商业用途,后来到了第二次世界大战期间,纳粹德国国防军
使用Enigma并将其改良后用于军事用途。

\startsubsection[title={Enigma的构造}]
\index{enigma-constructs}

Enigma加密机的外形如\in{图}[fig:enigma]所示,它是一种由键盘、齿轮、电池和灯泡所组成的机器,通过这一台
机器就可以完成加密和解密两种操作。

\startplacefigure
[title={Engima}, reference=fig:enigma]
\startcombination[3*1]
{\externalfigure[enigma][type=jpg,height=4cm]}{\Tiny Engima加密机}
{\externalfigure[rotor][type=jpg,height=4cm]}{\Tiny 转子}
{\externalfigure[enigma-in-use][type=jpg,height=4cm]}{\Tiny 德军在法国战场使用Engima密码机}
\stopcombination
\stopplacefigure

键盘上一共有26个按键,键盘排列和广为使用的计算机键盘基本一致,只不过为了使通讯尽量地短和难以破译,空格、数字和
标点符号都被取消,而只有字母键。键盘上方就是“显示器” (Lampboard),这可不是现今的计算机屏幕显示器,只不过是
标示了同样字母的26个小灯泡。当键盘上的某个字母键被按下时,这个字母被加密后的密文字母所对应的小灯泡就亮了起来,
就是这样一种近乎原始的“显示”。在显示器的上方是三个直径6厘米的转子 (Rotor),转子是Enigma密码机最核心关键的部分。
如果转子的作用仅仅是把一个字母转换成另一个字母,那就是等同于我们前一节介绍的简单替换密码。转子的巧妙之处在于它会
旋转,每按下键盘上的一个字母键,相应加密后的字母在显示器上通过灯泡闪亮来显示,而转子就自动地转动一个字母的位置。
这样,连续多次按下同一个字母键经过加密之后的密文字母都不相同。这就是Enigma难以被破译的关键所在,这不是一种简单
替换密码。同一个字母在明文的不同位置时,可以被不同的字母替换,而密文中不同位置的同一个字母,又可以代表明文中的
不同字母,字母频率分析法在这里丝毫无用武之地了。这种加密方式在密码学上也被称为{\it 复式替换密码}。

但是如果连续键入26个字母,转子就会整整转一圈,回到原始的方向上,这时编码就和最初重复了。而在加密过程中,重复的
现象就是最大的破绽,因为这可以使破译密码的人从中发现规律。于是Enigma又增加了其他的转子,当前一个转子转动整整
一圈以后,它上面有一个齿轮拨动下一个转子,使得它的方向转动一个字母的位置。而事实上,德军使用的Enigma有3个转子
(德国防卫军版)或4个转子(德国海军M4版和德国国防军情报局版)。以Enigma密码机上配置了3个转子为例,重复的概率
就达到了$26 \times 26 \times 26 = 17576$个字母之后。

除此以外,在第一个转子之前和最后一个转子之后分别加上了一个接线板和反射器。接线板允许操作员设置各种不同的线路。
接线板上的每条线都会连接一对字母,其作用就是在电流进入转子前改变它的方向。例如,将\type{A}插口和\type{F}插口
连接起来,当操作员按下\type{A}键时,电流就会流到\type{F}插口(相当于按下了\type{F}键)再进入转子。电流进入
转子前方向被改变,增强了Enigma的保密性。接线板上最多可以同时接13条线。

反射器和转子的显著区别在于它并不转动,它仅仅将最后一个转子的其中两个触点连接起来。乍一看这么一个固定的反射器
好像没什么用处,它并不增加可以使用的编码数目,其精妙之处在于,让电流重新折回转子,把它和解密联系起来就会看出
这种设计的别具匠心了。为了解释Enigma密码机的工作原理,我们
用\in{图}[fig:enigma-firstA]和\in{图}[fig:enigma-secondA]来分别说明第一次和第二次按下\type{A}键的
时候,Enigma是怎么加密的。

\startplacefigure
[title={Enigma电路布线示意图:第一次按下\type{A}键}, reference=fig:enigma-firstA]

\midaligned{
\starttikzpicture
  [pre/.style={<-,shorten <=1pt,>=stealth',semithick},
  post/.style={->,shorten >=1pt,>=stealth',semithick},
  every node/.style={draw=red!62.5!black, thick, font=\Tiny}]
  
  % plugboard, rotors, reflectors
  \foreach \r/\c/\i in {plugboard/接线板/0, left-rotor/转子 (左)/1, mid-rotor/转子 (中)/2, right-rotor/转子 (右)/3, reflector/反射器/4} {
    \node[minimum width=2cm, minimum height=9.25cm, 
          label={below:\c}] (\r) at (1cm+\i*3cm, 4.625cm) {};
  }

  \node[minimum width=1.2em, minimum height=1.2em, above=1mm of left-rotor] {Z};
  \node[minimum width=1.2em, minimum height=1.2em, above=1mm of mid-rotor] {L};
  \node[minimum width=1.2em, minimum height=1.2em, above=1mm of right-rotor] {J};

  % pins
  \foreach \i in {0,1,...,4} {
    \foreach \j in {0,1,...,25} {
      \fill[white!62.5!black, yshift=\j*0.35cm+0.15cm, xshift=\i*3cm] (-0.15cm, 0) rectangle +(0.3cm, 0.2cm);
      \ifnum \i < 4
        \fill[white!62.5!black, yshift=\j*0.35cm+0.15cm, xshift=\i*3cm+2cm] (-0.15cm, 0) rectangle +(0.3cm, 0.2cm);
      \fi
    }
  }

  % plugboard
  \foreach \a/\b in {0/5,1/7,2/2,3/3,4/10,5/0,6/16,7/1,8/8,
  9/9,10/4,11/11,12/12,13/19,14/14,15/15,16/6,17/17,18/18,
  19/13,20/25,21/21,22/22,23/23,24/24,25/20} {
    \draw[white!62.5!black, semithick] (0.15cm, \a*0.35cm+0.25cm) -- (2cm-0.15cm, \b*0.35cm+0.25cm);
  }

  % connection between plugboard and left rotor
  \foreach \i in {0,1,...,25} {
    \draw[white!62.5!black, semithick] (2cm+0.15cm, \i*0.35cm+0.25cm) -- (3cm-0.15cm, \i*0.35cm+0.25cm);
  }
  
  % left rotor
  \foreach \a/\b in {0/17,1/10,2/9,3/5,4/3,5/11,6/6,7/15,8/1,
  9/23,10/12,11/14,12/4,13/16,14/22,15/18,16/2,17/21,18/13,
  19/20,20/7,21/0,22/24,23/19,24/8,25/25} {
    \draw[white!62.5!black, semithick] (3cm+0.15cm, \a*0.35cm+0.25cm) -- (3cm+2cm-0.15cm, \b*0.35cm+0.25cm);
  }

  % connection between left and middle rotors
  \foreach \i in {0,1,...,25} {
    \draw[white!62.5!black, semithick] (3cm+2cm+0.15cm, \i*0.35cm+0.25cm) -- (6cm-0.15cm, \i*0.35cm+0.25cm);
  }

  % middle rotor
  \foreach \a/\b in {0/19,1/0,2/20,3/15,4/10,5/3,6/21,7/16,8/13,
  9/4,10/23,11/17,12/11,13/5,14/1,15/22,16/12,17/24,18/14,19/9,
  20/2,21/8,22/7,23/25,24/18,25/6} {
    \draw[white!62.5!black, semithick] (6cm+0.15cm, \a*0.35cm+0.25cm) -- (6cm+2cm-0.15cm, \b*0.35cm+0.25cm);
  }

  % connection between middle and right rotors
  \foreach \i in {0,1,...,25} {
    \draw[white!62.5!black, semithick] (6cm+2cm+0.15cm, \i*0.35cm+0.25cm) -- (9cm-0.15cm, \i*0.35cm+0.25cm);
  }

  % right rotor
  \foreach \a/\b in {0/4,1/7,2/18,3/6,4/13,5/12,6/25,7/14,8/23,
  9/17,10/24,11/3,12/21,13/2,14/11,15/0,16/10,17/19,18/22,19/16,
  20/1,21/15,22/5,23/8,24/20,25/9} {
    \draw[white!62.5!black, semithick] (9cm+0.15cm, \a*0.35cm+0.25cm) -- (9cm+2cm-0.15cm, \b*0.35cm+0.25cm);
  }

  % connection between right rotor and reflector
  \foreach \i in {0,1,...,25} {
    \draw[white!62.5!black, semithick] (9cm+2cm+0.15cm, \i*0.35cm+0.25cm) -- (12cm-0.15cm, \i*0.35cm+0.25cm);
  }

  % reflector
  \foreach \a/\b/\n in {3/4/1,5/8/1,11/15/1,16/17/1,18/20/1,21/22/1,
  2/9/2,10/24/2,0/12/3,13/19/3,6/14/4,7/23/5,1/25/6} {
    \draw[white!62.5!black, semithick] (12cm+0.15cm, \a*0.35cm+0.25cm) -- 
      (12cm+0.15cm+\n*0.25cm, \a*0.35cm+0.25cm) --
      (12cm+0.15cm+\n*0.25cm, \b*0.35cm+0.25cm) --
      (12cm+0.15cm, \b*0.35cm+0.25cm);
  }

  % keyboard
  \foreach \c/\i in {Z/0, Y/1, X/2, W/3, V/4, U/5, T/6, S/7, R/8, 
      Q/9, P/10, O/11, N/12, M/13, L/14, K/15, J/16, I/17, H/18, 
      G/19, F/20, E/21, D/22, C/23, B/24, A/25} {
    \pgfmathmod{\i}{2}
    \let\m\pgfmathresult
    \pgfmathparse{int(\m)}
    \let\r\pgfmathresult
    \ifnum \r > 0
      \node[shape=circle, inner sep=0pt, minimum size=1em] (\c) at (-1.5cm, \i*0.35cm+0.25cm) {\c};
    \else
      \node[shape=circle, inner sep=0pt, minimum size=1em] (\c) at (-2cm, \i*0.35cm+0.25cm) {\c};
    \fi
    \draw[white!62.5!black, semithick] (\c.east) -- (-0.15cm, \i*0.35cm+0.25cm);
  }

  % highlight
  \startscope [line width=4pt, line join=round]
  \draw[green!62.5!black, draw opacity=0.5] (A.east) -- (-0.15cm, 25*0.35cm+0.25cm) -- 
  (0.15cm, 25*0.35cm+0.25cm) -- (2cm-0.15cm, 20*0.35cm+0.25cm) --
  (2cm+0.15cm, 20*0.35cm+0.25cm) -- (3cm-0.15cm, 20*0.35cm+0.25cm) --
  (3cm+0.15cm, 20*0.35cm+0.25cm) -- (3cm+2cm-0.15cm, 7*0.35cm+0.25cm) --
  (3cm+2cm+0.15cm, 7*0.35cm+0.25cm) -- (6cm-0.15cm, 7*0.35cm+0.25cm) --
  (6cm+0.15cm, 7*0.35cm+0.25cm) -- (6cm+2cm-0.15cm, 16*0.35cm+0.25cm) --
  (6cm+2cm+0.15cm, 16*0.35cm+0.25cm) -- (9cm-0.15cm, 16*0.35cm+0.25cm) --
  (9cm+0.15cm, 16*0.35cm+0.25cm) -- (9cm+2cm-0.15cm, 10*0.35cm+0.25cm) --
  (9cm+2cm+0.15cm, 10*0.35cm+0.25cm) -- (12cm-0.15cm, 10*0.35cm+0.25cm) --
  (12cm+0.15cm, 10*0.35cm+0.25cm) -- (12cm+0.15cm+2*0.25cm, 10*0.35cm+0.25cm) --
  (12cm+0.15cm+2*0.25cm, 24*0.35cm+0.25cm) -- (12cm+0.15cm, 24*0.35cm+0.25cm) --
  (12cm-0.15cm, 24*0.35cm+0.25cm) -- (9cm+2cm+0.15cm, 24*0.35cm+0.25cm) --
  (9cm+2cm-0.15cm, 24*0.35cm+0.25cm) -- (9cm+0.15cm, 10*0.35cm+0.25cm) -- 
  (9cm-0.15cm, 10*0.35cm+0.25cm) -- (6cm+2cm+0.15cm, 10*0.35cm+0.25cm) -- 
  (6cm+2cm-0.15cm, 10*0.35cm+0.25cm) -- (6cm+0.15cm, 4*0.35cm+0.25cm) -- 
  (6cm-0.15cm, 4*0.35cm+0.25cm) -- (3cm+2cm+0.15cm, 4*0.35cm+0.25cm) -- 
  (3cm+2cm-0.15cm, 4*0.35cm+0.25cm) -- (3cm+0.15cm, 12*0.35cm+0.25cm) -- 
  (3cm-0.15cm, 12*0.35cm+0.25cm) -- (2cm+0.15cm, 12*0.35cm+0.25cm) -- 
  (2cm-0.15cm, 12*0.35cm+0.25cm) -- (0.15cm, 12*0.35cm+0.25cm) -- (N.east);
  \stopscope
\stoptikzpicture
}
\stopplacefigure

\startplacefigure
[title={Enigma电路布线示意图:第二次按下\type{A}键}, reference=fig:enigma-secondA]

\midaligned{
\starttikzpicture
  [pre/.style={<-,shorten <=1pt,>=stealth',semithick},
  post/.style={->,shorten >=1pt,>=stealth',semithick},
  every node/.style={draw=red!62.5!black, thick, font=\Tiny}]
  
  % plugboard, rotors, reflectors
  \foreach \r/\c/\i in {plugboard/接线板/0, left-rotor/转子 (左)/1, mid-rotor/转子 (中)/2, right-rotor/转子 (右)/3, reflector/反射器/4} {
    \node[minimum width=2cm, minimum height=9.25cm, 
          label={below:\c}] (\r) at (1cm+\i*3cm, 4.625cm) {};
  }
  \node[minimum width=1.2em, minimum height=1.2em, above=1mm of left-rotor] {A};
  \node[minimum width=1.2em, minimum height=1.2em, above=1mm of mid-rotor] {M};
  \node[minimum width=1.2em, minimum height=1.2em, above=1mm of right-rotor] {J};

  % pins
  \foreach \i in {0,1,...,4} {
    \foreach \j in {0,1,...,25} {
      \fill[white!62.5!black, yshift=\j*0.35cm+0.15cm, xshift=\i*3cm] (-0.15cm, 0) rectangle +(0.3cm, 0.2cm);
      \ifnum \i < 4
        \fill[white!62.5!black, yshift=\j*0.35cm+0.15cm, xshift=\i*3cm+2cm] (-0.15cm, 0) rectangle +(0.3cm, 0.2cm);
      \fi
    }
  }

  % plugboard
  \foreach \a/\b in {0/5,1/7,2/2,3/3,4/10,5/0,6/16,7/1,8/8,
  9/9,10/4,11/11,12/12,13/19,14/14,15/15,16/6,17/17,18/18,
  19/13,20/25,21/21,22/22,23/23,24/24,25/20} {
    \draw[white!62.5!black, semithick] (0.15cm, \a*0.35cm+0.25cm) -- (2cm-0.15cm, \b*0.35cm+0.25cm);
  }

  % connection between plugboard and left rotor
  \foreach \i in {0,1,...,25} {
    \draw[white!62.5!black, semithick] (2cm+0.15cm, \i*0.35cm+0.25cm) -- (3cm-0.15cm, \i*0.35cm+0.25cm);
  }
  
  % left rotor
  \foreach \a/\b in {0/0,1/18,2/11,3/10,4/6,5/4,6/12,7/7,8/16,9/2,
  10/24,11/13,12/15,13/5,14/17,15/23,16/19,17/3,18/22,19/14,
  20/21,21/8,22/1,23/25,24/20,25/9} {
    \draw[white!62.5!black, semithick] (3cm+0.15cm, \a*0.35cm+0.25cm) -- (3cm+2cm-0.15cm, \b*0.35cm+0.25cm);
  }

  % connection between left and middle rotors
  \foreach \i in {0,1,...,25} {
    \draw[white!62.5!black, semithick] (3cm+2cm+0.15cm, \i*0.35cm+0.25cm) -- (6cm-0.15cm, \i*0.35cm+0.25cm);
  }

  % middle rotor
  \foreach \a/\b in {0/7,1/20,2/1,3/21,4/16,5/11,6/4,7/22,8/17,9/14,
  10/5,11/24,12/18,13/12,14/6,15/2,16/23,17/13,18/25,19/15,20/10,
  21/3,22/9,23/8,24/0,25/19} {
    \draw[white!62.5!black, semithick] (6cm+0.15cm, \a*0.35cm+0.25cm) -- (6cm+2cm-0.15cm, \b*0.35cm+0.25cm);
  }

  % connection between middle and right rotors
  \foreach \i in {0,1,...,25} {
    \draw[white!62.5!black, semithick] (6cm+2cm+0.15cm, \i*0.35cm+0.25cm) -- (9cm-0.15cm, \i*0.35cm+0.25cm);
  }

  % right rotor
  \foreach \a/\b in {0/4,1/7,2/18,3/6,4/13,5/12,6/25,7/14,8/23,
  9/17,10/24,11/3,12/21,13/2,14/11,15/0,16/10,17/19,18/22,19/16,
  20/1,21/15,22/5,23/8,24/20,25/9} {
    \draw[white!62.5!black, semithick] (9cm+0.15cm, \a*0.35cm+0.25cm) -- (9cm+2cm-0.15cm, \b*0.35cm+0.25cm);
  }

  % connection between right rotor and reflector
  \foreach \i in {0,1,...,25} {
    \draw[white!62.5!black, semithick] (9cm+2cm+0.15cm, \i*0.35cm+0.25cm) -- (12cm-0.15cm, \i*0.35cm+0.25cm);
  }

  % reflector
  \foreach \a/\b/\n in {3/4/1,5/8/1,11/15/1,16/17/1,18/20/1,21/22/1,
  2/9/2,10/24/2,0/12/3,13/19/3,6/14/4,7/23/5,1/25/6} {
    \draw[white!62.5!black, semithick] (12cm+0.15cm, \a*0.35cm+0.25cm) -- 
      (12cm+0.15cm+\n*0.25cm, \a*0.35cm+0.25cm) --
      (12cm+0.15cm+\n*0.25cm, \b*0.35cm+0.25cm) --
      (12cm+0.15cm, \b*0.35cm+0.25cm);
  }

  % keyboard
  \foreach \c/\i in {Z/0, Y/1, X/2, W/3, V/4, U/5, T/6, S/7, R/8, 
      Q/9, P/10, O/11, N/12, M/13, L/14, K/15, J/16, I/17, H/18, 
      G/19, F/20, E/21, D/22, C/23, B/24, A/25} {
    \pgfmathmod{\i}{2}
    \let\m\pgfmathresult
    \pgfmathparse{int(\m)}
    \let\r\pgfmathresult
    \ifnum \r > 0
      \node[shape=circle, inner sep=0pt, minimum size=1em] (\c) at (-1.5cm, \i*0.35cm+0.25cm) {\c};
    \else
      \node[shape=circle, inner sep=0pt, minimum size=1em] (\c) at (-2cm, \i*0.35cm+0.25cm) {\c};
    \fi
    \draw[white!62.5!black, semithick] (\c.east) -- (-0.15cm, \i*0.35cm+0.25cm);
  }

  % highlight
  \startscope [line width=4pt, line join=round]
  \draw[green!62.5!black, draw opacity=0.5] (A.east) -- (-0.15cm, 25*0.35cm+0.25cm) -- 
  (0.15cm, 25*0.35cm+0.25cm) -- (2cm-0.15cm, 20*0.35cm+0.25cm) --
  (2cm+0.15cm, 20*0.35cm+0.25cm) -- (3cm-0.15cm, 20*0.35cm+0.25cm) --
  (3cm+0.15cm, 20*0.35cm+0.25cm) -- (3cm+2cm-0.15cm, 21*0.35cm+0.25cm) --
  (3cm+2cm+0.15cm, 21*0.35cm+0.25cm) -- (6cm-0.15cm, 21*0.35cm+0.25cm) --
  (6cm+0.15cm, 21*0.35cm+0.25cm) -- (6cm+2cm-0.15cm, 3*0.35cm+0.25cm) --
  (6cm+2cm+0.15cm, 3*0.35cm+0.25cm) -- (9cm-0.15cm, 3*0.35cm+0.25cm) --
  (9cm+0.15cm, 3*0.35cm+0.25cm) -- (9cm+2cm-0.15cm, 6*0.35cm+0.25cm) --
  (9cm+2cm+0.15cm, 6*0.35cm+0.25cm) -- (12cm-0.15cm, 6*0.35cm+0.25cm) --
  (12cm+0.15cm, 6*0.35cm+0.25cm) -- (12cm+0.15cm+4*0.25cm, 6*0.35cm+0.25cm) --
  (12cm+0.15cm+4*0.25cm, 14*0.35cm+0.25cm) -- (12cm+0.15cm, 14*0.35cm+0.25cm) --
  (12cm-0.15cm, 14*0.35cm+0.25cm) -- (9cm+2cm+0.15cm, 14*0.35cm+0.25cm) --
  (9cm+2cm-0.15cm, 14*0.35cm+0.25cm) -- (9cm+0.15cm, 7*0.35cm+0.25cm) -- 
  (9cm-0.15cm, 7*0.35cm+0.25cm) -- (6cm+2cm+0.15cm, 7*0.35cm+0.25cm) -- 
  (6cm+2cm-0.15cm, 7*0.35cm+0.25cm) -- (6cm+0.15cm, 0*0.35cm+0.25cm) -- 
  (6cm-0.15cm, 0*0.35cm+0.25cm) -- (3cm+2cm+0.15cm, 0*0.35cm+0.25cm) -- 
  (3cm+2cm-0.15cm, 0*0.35cm+0.25cm) -- (3cm+0.15cm, 0*0.35cm+0.25cm) -- 
  (3cm-0.15cm, 0*0.35cm+0.25cm) -- (2cm+0.15cm, 0*0.35cm+0.25cm) -- 
  (2cm-0.15cm, 0*0.35cm+0.25cm) -- (0.15cm, 5*0.35cm+0.25cm) -- (U.east);
  \stopscope
\stoptikzpicture
}
\stopplacefigure

首先,我们假设左、中、右三个转子的位置分别对应字母\type{Z}、\type{L}、\type{J},如\in{图}[fig:enigma-firstA]
所示。字母键\type{A}按下时,电流先流到接线板上的\type{A}插口,由于接线板上\type{A}插口和\type{F}插口连接起来了,
电流方向被改变,从\type{F}插口流出后进入到左边第一个转子的\type{F}插口。之后,依次经过所有转子,每个转子都会
对电流的方向进行转换,即对字母进行替换。当电流从右边最后一个转子的\type{P}插口出来之后,经过反射器改变方向
进入最后一个转子的\type{B}插口。此后,电流沿相反方向依次经过所有转子,最后从接线板\type{N}插口出来,点亮
\type{N}灯泡。这个就是加密的整个过程。在当前的设置下,如果这时按的不是\type{A}键而是\type{N}键,那么电流
信号恰好按照前面\type{A}键被按下时的相反方向同行,最后到达\type{A}灯泡。换句话说,在这种转子的设置下,
反射器使得解密过程完全重现加密过程。


再次按下字母键\type{A},左边的转子转动一格回到字母\type{A}的位置,同时带动中间的转子转动一格到\type{M}的
位置,右边的转子不动,仍然停留在\type{J}的位置,如\in{图}[fig:enigma-secondA]所示。在此时的转子的设置下,
按下\type{A}键,\type{U}灯泡亮起。同样,如果这时按下的是\type{U}键,则点亮的是\type{A}灯泡。反射器再次
完美地使得解密过程重现了加密过程。

从数学的角度,Enigma对每个字母的加密和解密过程可以看作由多步字符替换而组合在一起的过程。我们用$P$表示接线板
的连线所对应的字符替换,$L$、$M$、$R$分别表示左、中、右3个转子所对应的字符替换,$U$表示反射器所对应的字符
替换。其中接线板和反射器对应的字符替换$P$和$U$是一经设置就不再变化的。三个转子对应的字符替换$L$、$M$、$R$则
会随着字符在明文消息中的位置不同发生变化,我们用下标$k$来表示它们在第$k$个字符的替换。另外,当电流经过反射器
后折回沿反方向经过转子和接线板的过程正好是之前字符替换的反操作,我们用上标$-1$来表示这些字符替换的反操作。
因此,明文中第$k$个字符$x_k$被加密后的字符$E_k(x)$可以用如下数学公式表示:
\startformula
E_k(x) = P^{-1} L_k^{-1} M_k^{-1} R_k^{-1} U R_k M_k L_k P x_k
\stopformula

从上述加密过程对应的数学公式可以看出,Enigma构造具有完美的对称性。解密和加密具有相同的组合过程。密文中第$k$个密文
字符$E_k(x)$被解密还原出明文字符$x_k$可以用相同的数学公式表示:
\startformula
x_k = P^{-1} L_k^{-1} M_k^{-1} R_k^{-1} U R_k M_k L_k P E_k(x)
\stopformula

\stopsubsection

\startsubsection[title={Enigma的加密过程}]

\startplacefigure[title={Enigma的加密过程}, reference=fig:enigma-encrypt]
\midaligned{
\starttikzpicture
  [pre/.style={<-,shorten <=1pt,>=stealth',semithick},
  post/.style={->,shorten >=1pt,>=stealth',semithick},
  % node distance=0.5cm,
  every node/.style={draw=red!62.5!black, thick, fill=white!62.5!black, font=\Tiny, 
    inner sep=2mm, minimum height=2em, minimum width=4.8em},
  labels/.style={draw=none,fill=none,font=\Tiny,inner sep=0}]

  \node (plaintext) {明文消息};
  \node [rounded corners=10pt, right=of plaintext] (encrypt) {(4)加密消息} edge [pre] (plaintext);
  \node [right=of encrypt] (encrypted-text) {加密后的消息} edge [pre] (encrypt);
  \node [rounded corners=10pt, right=of encrypted-text] (concat) {(5)拼接} edge [pre] (encrypted-text);
  \node [right=of concat] {密文消息} edge [pre] (concat);

  \node [above=of encrypt] (comm-passwd) {\vbox{\hsize 5em 发送者选择的通信密码}} 
    edge [post] node[labels,right]{(3)重新设置Enigma} (encrypt);
  \node [rounded corners=10pt, right=of comm-passwd] (encrypt-passwd) {(2)加密通信密码} 
    edge [pre] node[labels,above]{输入两次} (comm-passwd);
  \node [above=of concat] (encrypted-passwd) {\vbox{\hsize 4em 加密后的通信密码}} edge [pre] (encrypt-passwd) edge [post] (concat);

  \node [above=of encrypt-passwd] (daily-passwd) {\vbox{\hsize 5em 密码本中的每日密码}} 
    edge [post] node[labels,right]{(1)设置Enigma} (encrypt-passwd);

  \startscope [on background layer]
  \node
  [fill=white!62.5!black, draw=red!62.5!black, very thick, inner sep=0.6cm, 
   rounded corners, 
   fit=(encrypt) (encrypted-passwd)] {};
  \stopscope
\stoptikzpicture
}
\stopplacefigure

发送者和接收者需要各自拥有一台Enigma密码机。发送者用Enigma对明文加密,记录生成的密文并通过无线电发送给接收者。
接收者收到密文后用自己的Enigma解密,还原出明文。

发送者和接收者会事先收到一份叫做国防军密码本的册子,这个册子中记载了发送者和接收者所使用的每日密码。Enigma的加密
过程如\in{图}[fig:enigma-encrypt]所示,具体描述如下:

{\bf (1) 设置Enigma}

发送者查阅国防军密码本,找到当天的每日密码,并按照该密码设置Engima,具体来说,这个每日密码描述了如果操作
接线板上的接线并设置3个转子排列顺序和每个转子的初始位置。

{\bf (2) 加密通信密码}

接下来,发送者要想出3个字母,并将其加密。这3个字母称为通信密码。通信密码的加密也是用Enigma完成的。假设发送者
选择的的通信密码是\type{cat},则发送者需要在Enigma的键盘上输入两次该通信密码,即\type{catcat}。发送者观察
亮起的灯泡对应的字符并记录这6个字母加密后的密文,我们用大写字母来假设得到的密文字母是\type{PCVTAM}。

{\bf (3) 重新设置Enigma}

接下来,发送者根据通信密码重新设置Enigma。通信密码中的3个字母就代表了3个转子的初始位置。也就是说,左、中、右
三个转子分别转到\type{c}、\type{a}、\type{t}的位置。

{\bf (4) 加密消息}

发送者从键盘上逐字输入明文消息的字符,并从灯泡中读取所对应的字母并记录下来。

{\bf (5) 拼接}

最后,发送者将加密后的通信密码和加密后的消息拼接在一起,通过无线电发送给接收者。

\stopsubsection

\startsubsection[title={Enigma的解密过程}]

接收者收到密文消息之后,解密过程如\in{图}[fig:enigma-decrypt]所示,具体操作步骤如下:

\startplacefigure[title={Enigma的加密过程}, reference=fig:enigma-decrypt]
\midaligned{
\starttikzpicture
  [pre/.style={<-,shorten <=1pt,>=stealth',semithick},
  post/.style={->,shorten >=1pt,>=stealth',semithick},
  % node distance=0.5cm,
  every node/.style={draw=red!62.5!black, thick, fill=white!62.5!black, font=\Tiny, 
    inner sep=2mm, minimum height=2em, minimum width=4.8em},
  labels/.style={draw=none,fill=none,font=\Tiny,inner sep=0}]

  \node (ciphertext) {密文消息};
  \node [rounded corners=10pt, right=of ciphertext] (split) {(1)拆分} edge [pre] (ciphertext);
  \node [right=of split] (encrypted-text) {加密后的消息} edge [pre] (split);
  \node [rounded corners=10pt, right=of encrypted-text] (decrypt) {(5)解密消息} edge [pre] (encrypted-text);
  \node [right=of decrypt] {明文消息} edge [pre] (decrypt);

  \node [above=of split] (encrypted-passwd) {\vbox{\hsize 4em 加密后的通信密码}} edge [pre] (split);
  \node [rounded corners=10pt, right=of encrypted-passwd] (decrypt-passwd) {(3)解密通信密码} edge [pre] (encrypted-passwd);
  \node [above=of decrypt] (comm-passwd) {\vbox{\hsize 5em 发送者选择的通信密码}} 
    edge [pre] (decrypt-passwd) 
    edge [post] node[labels,right]{(4)重新设置Enigma} (decrypt);

  \node [above=of decrypt-passwd] (daily-passwd) {\vbox{\hsize 5em 密码本中的每日密码}} 
    edge [post] node[labels,right]{(2)设置Enigma} (decrypt-passwd);

  \startscope [on background layer]
  \node
  [fill=white!62.5!black, draw=red!62.5!black, very thick, inner sep=0.6cm, 
   rounded corners, 
   fit=(split) (comm-passwd)] {};
  \stopscope
\stoptikzpicture
}
\stopplacefigure


{\bf (1) 拆分}

接收者将密文消息拆分成两个部分,即开头的6个字母\type{PCVTAM}和剩下的字母序列。

{\bf (2) 设置Enigma}

像发送者一样,接收者查阅国防军密码本,找到当天的每日密码,并按照该密码设置Engima。

{\bf (3) 解密通信密码}

开头的6个字母\type{PCVTAM}即加密后的通信密码,接收者用Enigma对其进行解密,得到\type{catcat}。因为
\type{catcat}是\type{cat}重复两次的组合,这样,接收者也可以判断密文消息在通信的过程是否发生错误。

{\bf (4) 重新设置Enigma}

接收者根据解密后的通信密码\type{cat}重新设置Enigma三个转子的初始位置。

{\bf (5) 解密消息}

接收者用当前Enigma的设置,对密文消息剩下部分的字母序列进行解密,得到明文消息内容。

\stopsubsection

\startsubsection[title={每日密码和通信密码}]

通过前面对Enigma加密和解密过程的描述,我们注意到Enigma中出现了每日密码和通信密码这两种不同的密钥。在Enigma中,
每日密码被用来加密通信密码,而不是用来加密消息的,消息是用通信密码加密的。也就是说,每日密码是一种用来加密密钥的
密钥。这样的密钥,被称为{\it 密钥加密密钥} (Key Encrypting Key, KEK)。KEK在现代加密算法中依然被广泛使用。
后面,我们在介绍混合密码系统时还会多次遇到这一概念。

\stopsubsection

\startsubsection[title={Enigma的弱点}]

我们已经了解了Enigma的加密和解密过程,相比较于简单替换密码,Engima的确要复杂得多。但我们仍然能找到Enigma的一些
弱点。

{\bf 明文中的字母被Enigma加密之后永远不会被替换成该字母本身。}
Enigma反射器在电流重新进入转子之前,改变了电流方向,无论接线板怎么接线以及三个转子的顺序和每个转子的旋转
位置如何改变,输入的字母都绝对不会被替换成该字母本身。第二次世界大战中,英国军队的密码破译者截获了一段
Enigma的密文,他们发现在密文中字母\type{L}从未出现。密码破译者根据这一事实推测出明文是一段只有
字母\type{L}的文字。发送者的目的是将毫无意义的明文加密发送以干扰密码破译者。发送者本想干扰密码破译者,
没想到却反而为破译者提供了线索。

{\bf 通信密码的弱点。}
通信密码太短,被加密后只有6个字母。密码破译者可以知道,密文开头的6个字母就是通信密码被连续输入两次而加密的。
而且,Enigma在加密通信密码这一重要步骤中,绝大部分情况下只有最左边的转子会旋转,只有当左边的转子设置
到\type{U}之后的字母时,才可能带动中间的转子旋转。这个特点也可能被密码破译者利用。

{\bf 国防军的每日密码本也是一个弱点。}
国防军的每日密码本是使用Enigma的必要操作手册。因为发送者和接收者都得使用这个密码本,如果这个密码本落到
敌人手里,就必须作废这个已经派发到全军的密码本,而不得不重新制作新的密码本。同时,如何安全地把这个密码本
配送到全军中也是一个问题。这个话题,就是我们今后要在介绍现代密码通信时要详细探讨的{\it 密钥配送问题}。

\stopsubsection

\startsubsection[title={Enigma的破译}]

Enigma在当时被认为是一种无法破译的密码机,德军的一份对Enigma的评估写道:“即使敌人获取了一台同样的机器,
它仍旧能够保证其加密系统的保密性。”Engima的设计并不依赖Enigma的构造(相当于加密算法),只要不知道Enigma的
设置(相当于密钥),就无法破译密码。Enigma的这个设计理念已经契合了现代密码体系的思想,即加密系统的保密性只应
建立在对密钥的保密上,不应该取决于加密算法的保密。Enigma的设置由每日密码所决定,具体表现为3个转子的排列
顺序、每个转子的初始位置、以及接线板连线的状况。我们先来看看要暴力破解,需要实验多少种可能性:

\startitemize[1,packed,broad]
\item 3个转子的排列顺序存在6种可能性;
\item 3个转子初始位置存在$26 \times 26 \times 26 = 17,576$种可能性;
\item 接线板上两两交换6对字母的可能性则异常庞大,有$100,391,791,500$种。
\stopitemize

于是一共有$17576 \times 6 \times 100,391,791,500$,其结果大约为$10,000,000,000,000,000$!即一亿亿种
可能性!这样庞大的可能性,换言之,即便能动员大量的人力物力,要想靠暴力破解法来逐一试验可能性,那几乎是不可能的。

1931年11月8日,法国情报人员通过间谍活动搞到了Engima的操作和内部线路的资料,但是法国还是无法破译它,因为Enigma的
设计要求就是要在机器被缴获后仍具有高度的保密性。当时的法军认为,由于凡尔赛条约限制了德军的发展,也就没有花费人力
物理去破译它。与法国不同,第一次世界大战中新独立的波兰的处境却很危险,西边的德国根据凡尔赛条约割让给了波兰大片领土,
德国人对此怀恨在心,而东边的苏联也在垂涎着波兰的领土。所以波兰需要时刻了解这两个国家的内部信息。在科学的其他领域,
我们说失败乃成功之母;而在密码分析领域,我们则应该说恐惧乃成功之母。这种险峻的形势造就了波兰一大批优秀的密码学家。
Enigma最终由波兰密码学家马里安·雷杰夫斯基 (Marian Rejewski) 破译。

雷杰夫斯基深知“重复乃密码大敌”。在Enigma密码中,最明显的重复莫过于每条电文最开始的那六个字母,它由三个字母的密钥
重复两次加密而成。德国人没有想到这里会是看似固若金汤的Enigma防线的弱点。雷杰夫斯基每天都会收到一大堆截获的德国电报,
所以一天中可以得到许多这样的六个字母串,它们都由同一个当日密钥加密而成。通过分析这些电文的前六个字母串,雷杰夫斯基
总结出Enigma的数量巨大的密钥主要是由接线板来提供的,如果只考虑转子的排列顺序和它们的初始位置,
只有$6 \times 17576 = 105,456$种可能性。虽然这还是一个很大的数字,但是把所有的可能性都试验一遍,已经是一件可以
做到的事情了。雷杰夫斯基和同事根据情报复制出了Enigma样机,并在Enigma的基础上设计了一台能自动验证所有
$26 \times 26 \times 26 = 17,576$个转子位置的机器,为了同时试验三个转子的所有可能的排列顺序,就需要6台同样的
机器,这样就可以试遍所有的$6 \times 17576 = 105,456$种转子排列顺序和初始位置。所有这6台Enigma和为使它们协作的
其他器材组成了一整个大约一米高的机器,能在两小时内找出当日密钥。

\stopsubsection

\stopsection

%%%%%%%%%%%%%%%%%%%%%%%%%%%%%%

\startsection[title={本章小结}]
\index{summary}

我们在本章回顾了历史上一些经典的加密算法及其典故,从中我们知道,古典密码多使用替换法进行加密和解密,并详细介绍了简单
替换和复式替换两种加密方法。同时,我们还讨论了这些古典密码所面临的问题,并尝试用暴力破解和频率分析的方法分别破译了
凯撒密码和简单替换密码。通过本章的学习,我们了解到:

\startitemize[n,packed,broad]
\item 暴力破解适用于破译密钥空间较小的加密算法,频率分析可以用于破解简单的单表替换密码。
\item 加密系统的保密性只应建立在对密钥的保密上,不应该取决于加密算法的保密。
\stopitemize

\stopsection

\stopchapter

\stopcomponent



% https://www.sohu.com/a/192352784_490113
